%
%  exercise-4.tex
%  artificial intelligence
%
%  Created by Illya Starikov on 01/21/18.
%  Copyright 2018. Illya Starikov. All rights reserved.
%

\RequirePackage[l2tabu, orthodox]{nag}
\documentclass[12pt]{scrartcl}

\newcommand{\exercisenumber}{4}
\newcommand{\duedate}{February 12\textsuperscript{th}, 2018}

\usepackage[normalem]{ulem}
\newcommand\redout{\bgroup\markoverwith{\textcolor{red}{\rule[0.5ex]{2pt}{0.8pt}}}\ULon}

\usetheme[logo=template-presentation/logo,faculty=ped]{fibeamer}

\usepackage{amssymb,amsmath,verbatim,graphicx,pdfpages,microtype,units,booktabs,upquote,xcolor,siunitx,csquotes,fancyvrb,newverbs,wrapfig,multicol,tikz,textcomp,wrapfig,cutwin}
\usepackage{fontawesome,setspace,rotchiffre,lipsum,listings,animate,listings}
\usepackage[xspace]{ellipsis}

\hypersetup{%
            colorlinks = true,
            linkcolor = orange,
            urlcolor  = orange,
            citecolor = orange,
            anchorcolor = orange}

\newcommand{\hugeslide}[1]{%
\begin{frame}[plain,c]
    \centering {\usebeamerfont*{frametitle} \usebeamercolor[fg]{frametitle}{\fontsize{40}{50}\selectfont\textit{#1}}}
\end{frame}
}

\newcommand{\presentaddcount}[1]{\addtocounter{#1}{1}\Roman{#1}}
\newcommand{\presentcount}[1]{\Roman{#1}}
\newcommand{\shellcmd}[1]{\texttt{\colorbox{gray!30}{#1}}}

\lstdefinelanguage{swift}
{%
  morekeywords={%
    func,if,then,else,for,in,while,do,switch,case,default,where,break,continue,fallthrough,return,
    typealias,struct,class,enum,protocol,var,func,let,get,set,willSet,didSet,inout,init,deinit,extension,
    subscript,prefix,operator,infix,postfix,precedence,associativity,left,right,none,convenience,dynamic,
    final,lazy,mutating,nonmutating,optional,override,required,static,unowned,safe,weak,internal,
    private,public,is,as,self,unsafe,dynamicType,true,false,nil,Type,Protocol,print
  },
  morecomment=[l]{//}, % l is for line comment
  morecomment=[s]{/*}{*/}, % s is for start and end delimiter
  morestring=[b]" % defines that strings are enclosed in double quotes
}

\definecolor{keyword}{HTML}{BA2CA3}
\definecolor{string}{HTML}{D12F1B}
\definecolor{comment}{HTML}{008400}
\definecolor{type}{HTML}{66B9AA}

\lstdefinestyle{Swift}{%
  language=swift,
  basicstyle=\ttfamily,
  showstringspaces=false, % lets spaces in strings appear as real spaces
  columns=fixed,
  keepspaces=true,
  keywordstyle=\color{keyword},
  stringstyle=\color{string},
  commentstyle=\color{comment},
  emph={Int,Character,Double,Float,Unsigned},
  emphstyle={\color{type}},
  morestring=[b]",
  escapeinside={(*}{*)}
}

\newcommand\syntaxbox[2][fill=orange!80]{%
    \tikz[baseline]\node[%
        inner ysep=0pt,
        inner xsep=2pt,
        anchor=text,
        rectangle,
        rounded corners=1mm,
        #1] {\strut#2};%
}



\begin{document}
\maketitle

\begin{statement}
    On page 90, we mentioned iterative lengthening search, an iterative analog of uniform cost search. The idea is to use increasing limits on path cost. If a node is generated whose path cost exceeds the current limit, it is immediately discarded. For each new iteration, the limit is set to the lowest path cost of any node discarded in the previous iteration.
\end{statement}

\section{Iterative Lengthening Optimality}
\begin{statement}
    Show that this algorithm is optimal for general path costs.
\end{statement}

Because the nodes are discovered in order of path cost, naturally the node with the lowest path cost will be discovered first.

\section{Iterative Lengthening Performance}
\begin{statement}
    Consider a uniform tree with branching factor $b$, solution depth $d$, and unit step costs. How many iterations will iterative lengthening require?
\end{statement}

The runtime will be $\bigO{b^d}$.

\section{Iterative Lengthening Performance II}
\begin{statement}
    Now consider step costs drawn from the continuous range $[\epsilon, 1]$, where $0 < \epsilon < 1$. How many iterations are required in the worst case?
\end{statement}

The runtime will be $\bigO{\frac{d}{\epsilon}}$.


\end{document}
