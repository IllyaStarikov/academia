%
%  exercise-5.tex
%  artificial intelligence
%
%  Created by Illya Starikov on 01/21/18.
%  Copyright 2018. Illya Starikov. All rights reserved.
%

\RequirePackage[l2tabu, orthodox]{nag}
\documentclass[12pt]{scrartcl}

\newcommand{\exercisenumber}{5}
\newcommand{\duedate}{February 19\textsuperscript{th}, 2018}

\usepackage[normalem]{ulem}
\newcommand\redout{\bgroup\markoverwith{\textcolor{red}{\rule[0.5ex]{2pt}{0.8pt}}}\ULon}

\usetheme[logo=template-presentation/logo,faculty=ped]{fibeamer}

\usepackage{amssymb,amsmath,verbatim,graphicx,pdfpages,microtype,units,booktabs,upquote,xcolor,siunitx,csquotes,fancyvrb,newverbs,wrapfig,multicol,tikz,textcomp,wrapfig,cutwin}
\usepackage{fontawesome,setspace,rotchiffre,lipsum,listings,animate,listings}
\usepackage[xspace]{ellipsis}

\hypersetup{%
            colorlinks = true,
            linkcolor = orange,
            urlcolor  = orange,
            citecolor = orange,
            anchorcolor = orange}

\newcommand{\hugeslide}[1]{%
\begin{frame}[plain,c]
    \centering {\usebeamerfont*{frametitle} \usebeamercolor[fg]{frametitle}{\fontsize{40}{50}\selectfont\textit{#1}}}
\end{frame}
}

\newcommand{\presentaddcount}[1]{\addtocounter{#1}{1}\Roman{#1}}
\newcommand{\presentcount}[1]{\Roman{#1}}
\newcommand{\shellcmd}[1]{\texttt{\colorbox{gray!30}{#1}}}

\lstdefinelanguage{swift}
{%
  morekeywords={%
    func,if,then,else,for,in,while,do,switch,case,default,where,break,continue,fallthrough,return,
    typealias,struct,class,enum,protocol,var,func,let,get,set,willSet,didSet,inout,init,deinit,extension,
    subscript,prefix,operator,infix,postfix,precedence,associativity,left,right,none,convenience,dynamic,
    final,lazy,mutating,nonmutating,optional,override,required,static,unowned,safe,weak,internal,
    private,public,is,as,self,unsafe,dynamicType,true,false,nil,Type,Protocol,print
  },
  morecomment=[l]{//}, % l is for line comment
  morecomment=[s]{/*}{*/}, % s is for start and end delimiter
  morestring=[b]" % defines that strings are enclosed in double quotes
}

\definecolor{keyword}{HTML}{BA2CA3}
\definecolor{string}{HTML}{D12F1B}
\definecolor{comment}{HTML}{008400}
\definecolor{type}{HTML}{66B9AA}

\lstdefinestyle{Swift}{%
  language=swift,
  basicstyle=\ttfamily,
  showstringspaces=false, % lets spaces in strings appear as real spaces
  columns=fixed,
  keepspaces=true,
  keywordstyle=\color{keyword},
  stringstyle=\color{string},
  commentstyle=\color{comment},
  emph={Int,Character,Double,Float,Unsigned},
  emphstyle={\color{type}},
  morestring=[b]",
  escapeinside={(*}{*)}
}

\newcommand\syntaxbox[2][fill=orange!80]{%
    \tikz[baseline]\node[%
        inner ysep=0pt,
        inner xsep=2pt,
        anchor=text,
        rectangle,
        rounded corners=1mm,
        #1] {\strut#2};%
}



\begin{document}
\maketitle

\section{$A^*$ vs. DFS}
\begin{statement}
    True/False: Depth-first search always expands at least as many nodes as $A^∗$ search with an admissible heuristic.
\end{statement}

\textbf{False.} DFS has the best case of $\bigTheta{d}$ (for the case where there is no backtracking to find a solution). $A^*$ can make no such guarantee, because it ranges by heuristic.

\section{$h(n) = 0$ For 8-Puzzle}
\begin{statement}
    True/False: $h(n) = 0$ is an admissible heuristic for the 8-puzzle.
\end{statement}

\textbf{True.} For positive path costs, $h(n) = 0$ will always be an admissible heuristic.

\begin{equation*}
    \forall \epsilon \in \mathbb{R}^+, h(n) = 0 \leq C^*(n) = \epsilon
\end{equation*}

\section{$A^*$ In Robotics}
\begin{statement}
    True/False: $A∗$ is of no use in robotics because percepts, states, and actions are continuous.
\end{statement}

\textbf{True.} $A^*$ operates on discrete states. However, there is always the possibility to create discrete states from continuous states.

\section{Breadth-First Search Completeness}
\begin{statement}
    True/False: Breadth-first search is complete even if zero step costs are allowed.
\end{statement}

\textbf{True.} Per our definition of completeness:

\begin{quote}
    A search algorithm is complete iff it will find a goal when one exists.
\end{quote}

\section{Rook Admissibility}
\begin{statement}
    True/False: Assume that a rook can move on a chessboard any number of squares in a straight line, vertically or horizontally, but cannot jump over other pieces. Manhattan distance is an admissible heuristic for the problem of moving the rook from square A to square B in the smallest number of moves.
\end{statement}

\textbf{False.} Using the chess depicted in Figure~\ref{fig:chessboard}, the Manhattan Distance from $A1$ to $A8$ would be \num{8}; however, a rook could get there in one move.

\begin{figure}[h]
    \centering
    \includegraphics[width=.5\linewidth]{assets/chessboard.jpg}
    \caption{A Standard Chessboard}
    \label{fig:chessboard}
\end{figure}

\end{document}
