%
%  exercise-6.tex
%  artificial intelligence
%
%  Created by Illya Starikov on 3/13/18.
%  Copyright 2018. Illya Starikov. All rights reserved.
%

\RequirePackage[l2tabu, orthodox]{nag}
\documentclass[12pt]{scrartcl}

\newcommand{\exercisenumber}{6}
\newcommand{\duedate}{March 13\textsuperscript{th}, 2018}

\usetheme[logo=template-presentation/logo,faculty=ped]{fibeamer}

\usepackage{amssymb,amsmath,verbatim,graphicx,pdfpages,microtype,units,booktabs,upquote,xcolor,siunitx,csquotes,fancyvrb,newverbs,wrapfig,multicol,tikz,textcomp,wrapfig,cutwin}
\usepackage{fontawesome,setspace,rotchiffre,lipsum,listings,animate,listings}
\usepackage[xspace]{ellipsis}

\hypersetup{%
            colorlinks = true,
            linkcolor = orange,
            urlcolor  = orange,
            citecolor = orange,
            anchorcolor = orange}

\newcommand{\hugeslide}[1]{%
\begin{frame}[plain,c]
    \centering {\usebeamerfont*{frametitle} \usebeamercolor[fg]{frametitle}{\fontsize{40}{50}\selectfont\textit{#1}}}
\end{frame}
}

\newcommand{\presentaddcount}[1]{\addtocounter{#1}{1}\Roman{#1}}
\newcommand{\presentcount}[1]{\Roman{#1}}
\newcommand{\shellcmd}[1]{\texttt{\colorbox{gray!30}{#1}}}

\lstdefinelanguage{swift}
{%
  morekeywords={%
    func,if,then,else,for,in,while,do,switch,case,default,where,break,continue,fallthrough,return,
    typealias,struct,class,enum,protocol,var,func,let,get,set,willSet,didSet,inout,init,deinit,extension,
    subscript,prefix,operator,infix,postfix,precedence,associativity,left,right,none,convenience,dynamic,
    final,lazy,mutating,nonmutating,optional,override,required,static,unowned,safe,weak,internal,
    private,public,is,as,self,unsafe,dynamicType,true,false,nil,Type,Protocol,print
  },
  morecomment=[l]{//}, % l is for line comment
  morecomment=[s]{/*}{*/}, % s is for start and end delimiter
  morestring=[b]" % defines that strings are enclosed in double quotes
}

\definecolor{keyword}{HTML}{BA2CA3}
\definecolor{string}{HTML}{D12F1B}
\definecolor{comment}{HTML}{008400}
\definecolor{type}{HTML}{66B9AA}

\lstdefinestyle{Swift}{%
  language=swift,
  basicstyle=\ttfamily,
  showstringspaces=false, % lets spaces in strings appear as real spaces
  columns=fixed,
  keepspaces=true,
  keywordstyle=\color{keyword},
  stringstyle=\color{string},
  commentstyle=\color{comment},
  emph={Int,Character,Double,Float,Unsigned},
  emphstyle={\color{type}},
  morestring=[b]",
  escapeinside={(*}{*)}
}

\newcommand\syntaxbox[2][fill=orange!80]{%
    \tikz[baseline]\node[%
        inner ysep=0pt,
        inner xsep=2pt,
        anchor=text,
        rectangle,
        rounded corners=1mm,
        #1] {\strut#2};%
}



\begin{document}
\maketitle

\section{Suboptimal Min vs Optimal Min}
\begin{statement}
    Prove the following assertion: For every game tree, the utility obtained by \textit{MAX} using minimax decisions against a suboptimal \textit{MIN} will be never be lower than the utility obtained playing against an optimal \textit{MIN}. Can you come up with a game tree in which \textit{MAX} can do still better using a suboptimal strategy against a suboptimal \textit{MIN}?
\end{statement}

We will do so by induction.

\begin{description}
    \item[Basis Case] The base case is a \textit{MIN} node with all terminals. Suppose we have the following leaf such that for any node $h_x$, they are ordered in ascending order based on the utility value $h_1 \leq h_2 \leq \cdots \leq h_{n - 1} \leq h_n$:
        \begin{figure}[H]
            \centering

            \begin{forest}
                [MIN [
                    [$h_1$]
                    [$h_2$]
                    [$\cdots$]
                    [$h_{n - 1}$]
                    [$h_n$]
                ]]
            \end{forest}
        \end{figure}

        An optimal \textit{MIN} would always pick $h_1$, while a suboptimal would pick from $\{h_1, h_2, \ldots, h_{n - 1}, h_n\}$. From this, we can conclude
        \begin{equation}
            \text{Optimal \textit{MIN}'s Utility} \leq \text{Suboptimal \textit{MIN} Utility}
        \end{equation}

    \item[Inductive Case] The recursive case is for \textit{MIN} with parents and children.
        \begin{figure}[H]
            \centering

            \begin{forest}
                [MAX
                    [MIN [[\vdots][\vdots][\vdots]]]
                    [MIN [
                        [$h_1$       ]
                        [$h_2$       ]
                        [$\cdots$    ]
                        [$h_{n - 1}$ ]
                        [$h_n$       ]
                    ]]
                    [MIN [[\vdots][\vdots][\vdots]]]
                ]
            \end{forest}
        \end{figure}

        However, because the game tree can be reduced to a \textit{MIN} node with all terminals (such as the base case), our inductive step holds.
\end{description}

For a game tree of a suboptimal \textit{MAX} vs a \textit{MIN}, we can produce the following chess game:

\begin{figure}[H]
    \centering

    \begin{forest}
        [MAX[
            [[MIN
                [
                    [[1]]
                    [[1]]
                    [[1]]
                    [[$\cdots$]]
                    [[\textcolor{red}{-1}]]
                    [[1]]
                ]
            ]]
            [[MIN
                [
                    [[\nicefrac{1}{2}]]
                    [[\nicefrac{1}{2}]]
                    [[\nicefrac{1}{2}]]
                    [[$\cdots$]]
                    [[\nicefrac{1}{2}]]
                    [[\nicefrac{1}{2}]]
                ]
            ]]
        ]]
    \end{forest}
\end{figure}

An optimal \textit{MAX} would pick the rightmost \textit{MIN}, which would result in only a draw; however, a suboptimal \textit{MIN} would only pick the leftmost \textit{MIN}, resulting in in almost all wins, with only one possibility of losing.

\end{document}
