%
%  exercise-3.tex
%  artificial intelligence
%
%  Created by Illya Starikov on 01/21/18.
%  Copyright 2018. Illya Starikov. All rights reserved.
%

\RequirePackage[l2tabu, orthodox]{nag}
\documentclass[12pt]{scrartcl}

\newcommand{\exercisenumber}{3}
\newcommand{\duedate}{February 5\textsuperscript{th}, 2018}

\usepackage{amssymb,amsmath,verbatim,graphicx,microtype,upquote,units,booktabs,akkwidepage}

\newcommand{\chapterNumber}[1]{
    \setcounter{section}{#1}
    \addtocounter{section}{-1}
}

\begin{document}
\maketitle

\section{Negative Path Optimality}
\begin{statement}
    Suppose that actions can have arbitrarily large negative costs; explain why this possibility would force any optimal algorithm to explore the entire state space.
\end{statement}

Even if for any path we have an arbitrary path $P_\max$ with largest path cost $N_\max$,

\begin{equation*}
    N_\text{max} = \sum_{\text{vertex $v$} \in P_\text{max}} \textsc{path-cost}\left(v\right)
\end{equation*}

There might be a singular path $P$ after it,

\begin{equation*}
    N = \sum_{\text{vertex $v$} \in P} \textsc{path-cost}\left(v\right)
\end{equation*}

Such that

\begin{equation*}
    N_\text{max} + N = \text{Minimum Path Cost}
\end{equation*}

Therefore, any algorithm that is to find the optimal path cost must search the entire state space.

Also, there is the possibility of a negative cycle in a graph, which can only be discovered by exploring the entire state space.

\section{Bounds on Negative Path Cost}
\begin{statement}
    Does it help if we insist that step costs must be greater than or equal to some negative constant $c$? Consider both trees and graphs.
\end{statement}

For trees, it can be helpful in limiting the search space, because there can only be so much gained from negative paths.

For graphs, there is the aforementioned issue of negative cycles, where the lowest path cost could be $-\infty$ (i.e., keep repeating the negative cycle infinitely).

\section{Loops In Negative Path Cost}
\begin{statement}
    Suppose that a set of actions forms a loop in the state space such that executing the set in some order results in no net change to the state. If all of these actions have negative cost, what does this imply about the optimal behavior for an agent in such an environment?
\end{statement}

This implies that the optimal behavior in this environment is to loop around the states in the negative cycle.


\end{document}
