%
%  bonus-1.tex
%  bonus
%
%  Created by Illya Starikov on 02/18/18.
%  Copyright 2018. Illya Starikov. All rights reserved.
%

\RequirePackage[l2tabu, orthodox]{nag}
\documentclass[12pt]{scrartcl}

\usepackage{amssymb,amsmath,verbatim,graphicx,microtype,upquote,units,booktabs,siunitx,xcolor,float}

\title{Bonus \#1}
\date{Due Date: Sunday, February 18\textsuperscript{th}, 2018}
\author{Illya Starikov}

\begin{document}
\maketitle

We know the definition of branching factor $b^*$ to be defined as

\begin{equation*}
    N = b^* + {(b^*)}^2 + \cdots + {(b^*)}^d
\end{equation*}

With $N$ being the total number of nodes being generated, and $d$ being the solution depth. From this, we can get an estimate for the effective branching factor as follows:

\begin{equation*}
    b^* = N^{\frac{1}{d}}
\end{equation*}

Note this is not a perfect equation, and has the possibility for high error tolerances; for our purposes, it should be okay. The two heuristics for this assignment are as follows:
\begin{align*}
    h_1 &= \text{Score Difference} \\
        &= \frac{\text{Quota $-$ Current Score}}{\text{Max Swaps $-$ Current Swaps}} \\
    h_2 &= \text{Homogeneous Board} \\
        &= \text{Standard Deviation of the} \\
        &= \text{Number of Devices Types on the Board}
\end{align*}


Results can be found in Table~\ref{tab:results}. Essentially, the branching factors are similar, but the Homogeneous Board heuristic has a tendency to explore more of the board.

\begin{table}[H]
    \centering
    \caption{The computations for the effective branching factor.}
    \label{tab:results}
    \resizebox{\textwidth}{!}{%
    \begin{tabular}{llccc}
        \toprule
        \textbf{Input} & \textbf{Heuristic} & \textbf{Nodes Generated ($N$)} & \textbf{Solution Depth ($d$)} & \textbf{Branching Factor ($b^*$)} \\
        Puzzle \#1 & Score Difference  & \num{4}   & \num{1}  & \num{4} \\
        {}         & Homogeneous Board & \num{4}   & \num{1}  & \num{4} \\

        Puzzle \#2 & Score Difference  & \num{99}  & \num{11} & \num{1.52} \\
        {}         & Homogeneous Board & \num{144} & \num{10} & \num{1.64} \\

        Puzzle \#3 & Score Difference  & \num{94}  & \num{12} & \num{1.46} \\
        {}         & Homogeneous Board & \num{890} & \num{25} & \num{1.31} \\
        \bottomrule
    \end{tabular}
    }
\end{table}

\end{document}
