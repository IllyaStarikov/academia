%
%  test-1.tex
%  artificial intelligence
%
%  Created by Illya Starikov on Saturday, July 15, 2018.
%  Copyright 2018. Illya Starikov. All rights reserved.
%

\RequirePackage[l2tabu, orthodox]{nag}
\documentclass[12pt]{scrartcl}

\newcommand{\exercisenumber}{2}
\newcommand{\duedate}{July 15\textsuperscript{th}, 2018}
\usepackage{amssymb,amsmath,verbatim,graphicx,microtype,upquote,units,booktabs,akkwidepage}

\newcommand{\chapterNumber}[1]{
    \setcounter{section}{#1}
    \addtocounter{section}{-1}
}
\usepackage{multicol}
\usepackage[shortlabels]{enumerate}

\title{Test \#1}

\begin{document}
\maketitle

\section*{Multiple Choice}
\begin{enumerate}
    \item \textbf{e. None of the above}
    \item \textbf{c. Remove any attribute that has missing values.}
    \item \textbf{b. $\frac{1}{2}$}
    \item \textbf{b. wt}
    \item \textbf{d. Spearman’s rank correlation coefficient}
    \item \textbf{c. Healthland}
    \item \textbf{b. slice for Time = Q1}
    \item \textbf{d. roll up on Location = Beijing or Tokyo  (i.e., from city to country)}
    \item \textbf{c. drill down on Time = Q1  (i.e., from quarter to month)}
    \item \textbf{a. dice for (location = Beijing or Tokyo) and (product = Chain or bracelet) and (time = Q1 or Q2)}
\end{enumerate}

\setcounter{section}{10}
\section{Short Answer}
Method \#1 is the most accurate, because the true positive ($y$-axis) correctly identified the values, while the false positive ($x$-axis) incorrectly identified the values. Method \#1 had the fastest growing function (with respect to $y$).

\section{1-R Method}
\begin{table}[H]
    \resizebox{\columnwidth}{!}{%
        \begin{tabular}{l|l|l|l|l|l}
            \hline
            \multicolumn{1}{|l|}{\textbf{Attribute}} & \textbf{Attribute Value} & \textbf{\# Rows With Attribute Value} & \textbf{Most Frequent Value For sportPref} & \textbf{Errors} & \multicolumn{1}{l|}{\textbf{Total Errors}} \\ \hline
            \multicolumn{1}{|l|}{ageGroup}           & youngAdult               & 3                                   & football (2)                           & 1               & \multicolumn{1}{l|}{3}                     \\ \hline
                                         & middleAge                & 3                                   & football/hockey/baseball (1/1/1)       & 2               &                                            \\ \cline{2-5}
                                         & senior                   & 2                                   & baseball (2)                           & 0               &                                            \\ \hline
            \multicolumn{1}{|l|}{gender}             & M                        & 5                                   & baseball/football (2/2)                & 3               & \multicolumn{1}{l|}{5}                     \\ \hline
                                         & F                        & 3                                   & football/hockey/baseball (1/1/1)       & 2               &                                            \\ \hline
            \multicolumn{1}{|l|}{petPreference}      & dog                      & 5                                   & football (3)                           & 2               & \multicolumn{1}{l|}{3}                     \\ \hline
                                         & cat                      & 3                                   & baseball (2)                           & 1               &                                            \\ \cline{2-5}
        \end{tabular}
        }
\end{table}

The rules are as follows:
\begin{alignat*}{3}
    &\text{ageGroup $=$ \textbf{youngAdult}} &&\implies \text{football} \\
    &\text{ageGroup $=$ \textbf{middleAge}}  &&\implies \text{football} \\
    &\text{ageGroup $=$ \textbf{senior}}     &&\implies \text{baseball} \\
\end{alignat*}


\section{Prism}
For football, we get the following table:

\begin{table}[H]
    \centering
    \begin{tabular}{|c|c|c|c|}
        \hline
        gender & pet & drink & sport    \\\hline
        M      & dog & beer  & football \\
        F      & dog & beer  & football \\\hline
    \end{tabular}
\end{table}

For our P and T values:


\begin{table}[H]
    \centering
    \begin{tabular}{|c|ccc|}
        \hline
                     & T & P   & T/P \\\hline
        gender = M   & 3 & 1   & 1/3 \\
        gender = F   & 4 & 1   & 1/4 \\
        pet = dog    & 3 & 2   & 2/3 \\
        drink = beer & 3 & 2   & 3/4 \\\hline
    \end{tabular}
\end{table}

Seeing as not T/P values are 1, we must add a clause. We choose pet = dog as the base.

\begin{table}[H]
    \centering
    \begin{tabular}{|c|ccc|}
        \hline
                     & T & P   & T/P \\\hline
        gender = M   & 1 & 0 & 0 \\
        gender = F   & 1 & 0 & 0 \\
        drink = beer & 2 & 2 & 1 \\\hline
    \end{tabular}
\end{table}

\begin{alignat*}{3}
    &\text{pet $=$ \textbf{dog} and drink $=$ \textbf{beer}} &&\implies \text{football} \\
\end{alignat*}

\section{Statistical Modeling}
The likelihood would be as follows:

\begin{equation*}
    \text{likelihood} =
    \nicefrac{4}{9} \times
    \nicefrac{2}{9} \times
    \nicefrac{6}{9} \times % lol
    \nicefrac{3}{9} \times
    \nicefrac{9}{14}
\end{equation*}

\section{Entropy}
\begin{enumerate}[(a)]
    \item entropyBeforeSplit would be as follows:
        \begin{equation*}
            -\nicefrac{1}{6}\,\log_2\left(\nicefrac{1}{6}\right)
            -\nicefrac{2}{6}\,\log_2\left(\nicefrac{2}{6}\right)
            -\nicefrac{3}{6}\,\log_2\left(\nicefrac{3}{6}\right)
        \end{equation*}

    \item entropyPoor would be as follows:
        \begin{equation*}
            -\nicefrac{2}{4}\,\log_2\left(\nicefrac{2}{4}\right)
            -\nicefrac{2}{4}\,\log_2\left(\nicefrac{2}{4}\right)
        \end{equation*}

    \item infoGain would be determined as follows:
        \begin{align*}
            \text{entropyAfterSplit} &=
            \nicefrac{3}{6}\, \text{entropyShort} +
            \nicefrac{2}{6}\, \text{entropyMed} +
            \nicefrac{1}{6}\, \text{entropyLong} \\
            \text{infoGain} &= \text{entropyBeforeSplit} - \text{entropyAfterSplit} \\
        \end{align*}
\end{enumerate}

\section{Rule Induction}
\begin{enumerate}[(a)]
    \item The partitions would be as follows:
        \begin{align*}
            {\{d\}}^*      & = \{ \{x_1\}, \{x_2, x_3\}, \{x_5\}, \{x_5\} \}     \\
            {\{e\}}^*      & = \{ \{x_1, x_2, x_5\}, \{x_3, x_4\} \}             \\
            {\{d,\, e\}}^* & = \{ \{x_1\}, \{x_2\}, \{x_3\}, \{x_4\}, \{x_5\} \} \\
        \end{align*}

    \item The coverings are as follows:
        \begin{itemize}
            \item ${\{d\}}^*$ would not work, because every block in the partition is not a subset of a block in ${\{f\}}^*$.
            \item ${\{d, e\}}^*$ would work, because every block in the partition is a subset of a block in ${\{f\}}^*$.
            \item ${\{a, d, e\}}^*$ would not work, because although every block in the partition is a subset of a block in ${\{f\}}^*$, it is not minimal.
        \end{itemize}

    \item The rules would be as follows:
        \begin{alignat*}{3}
            &\text{$d$ = X and $e = 4$}  &&\implies \text{$f = T$} \\
            &\text{$d$ = S and $e = 4$}  &&\implies \text{$f = T$} \\
            &\text{$d$ = S and $e = 3$}  &&\implies \text{$f = F$} \\
            &\text{$d$ = H and $e = 3$}  &&\implies \text{$f = F$} \\
            &\text{$d$ = M and $e = 4$}  &&\implies \text{$f = F$} \\
        \end{alignat*}
\end{enumerate}

\section{KD-Tree}
Sorting, we get the following: [(2, 10), (4, 20), (6, 10), (8, 20), (10, 30)].

With a median of 6...

\begin{itemize}
    \item $x < 6$ group: [(2, 10), (4, 20)]
    \item $x \geq 6$ group: [(6, 10), (8, 20), (10, 30)]
\end{itemize}

Sorting, we get the following: [(2, 10), (4, 20)] [(6, 10), (8, 20), (10, 30)]

With a median of 15 for the first group:

\begin{itemize}
    \item $y < 15$ group: [(2, 10)]
    \item $y \geq 15$ group: [(4, 20)]
\end{itemize}

With a median of 20 for the second group:

\begin{itemize}
    \item $y < 20$ group: (6, 10)
    \item $y \geq 20$ group: [(8, 20), (10, 30)]
\end{itemize}

(Using a shortcut for the final block), Sorting, and using a median of 9, our last block looks like as follows:

\begin{itemize}
    \item $x < 9$ group: [(8, 20)]
    \item $y \geq 9$ group: [(10, 30)]
\end{itemize}

\begin{center}
    \begin{forest}
        for tree={%
            l sep=2cm,
        s sep=0.1cm,
        minimum height=0.8cm,
        minimum width=2cm
        }
        [x
            [y, edge label={node[midway,fill=white]{$< 6$}}
                [{$(2,\, 10)$}, edge label={node[midway,fill=white]{$< 15$}}]
                [{$(4,\, 20)$}, edge label={node[midway,fill=white]{$\geq 15$}}]
            ]
            [y, edge label={node[midway,fill=white]{$\geq 6$}}
                [{$(6,\, 10)$}, edge label={node[midway,fill=white]{$< 20$}}]
                [x, edge label={node[midway,fill=white]{$\geq 20$}}
                    [{$(8,\, 20)$}, edge label={node[midway,fill=white]{$< 9$}}]
                    [{$(10,\, 30)$}, edge label={node[midway,fill=white]{$\geq 9$}}]
                ]
            ]
        ]
    \end{forest}
\end{center}

\section{Clustering}

\begin{table}[H]
    \centering
    \begin{tabular}{|c|c|c|c|c|}
        \hline
        x & y  & distance to (2, 4) & distance to (5, 6) & distance to (8, 1) \\\hline
        2 & 4  & 0                  & 5                  & 9 \\
        5 & 6  & 5                  & 0                  & 8 \\
        8 & 1  & 9                  & 8                  & 0 \\
        7 & 3  & 6                  & 5                  & 3 \\
        4 & 10 & 8                  & 5                  & 13 \\
        3 & 0  & 5                  & 8                  & 6 \\
        9 & 8  & 11                 & 6                  & 8 \\
        \hline
    \end{tabular}
\end{table}

Our clusters would be as follows:

\begin{description}
    \item[Cluster Center (2, 4)] (2, 4), (3, 0)
    \item[Cluster Center (5, 6)] (5, 6), (4, 10), (9, 8)
    \item[Cluster Center (8, 1)] (8, 1), (7, 3)
\end{description}

With means as follows:

\begin{description}
    \item[Cluster Mean of (2, 4), (3, 0)] (2.5, 2) $\approx$ (3, 2)
    \item[Cluster Mean of (5, 6), (4, 10), (9, 8)] (6, 8)
    \item[Cluster Center of (8, 1), (7, 3)] (7.5, 2) $\approx$ (8, 2)
\end{description}

\begin{table}[H]
    \centering
    \begin{tabular}{|c|c|c|c|c|}
        \hline
        x & y  & distance to (3, 2) & distance to (6, 8) & distance to (8, 2) \\\hline
        2 & 4  & 3                  & 8                  & 8 \\
        5 & 6  & 6                  & 3                  & 7 \\
        8 & 1  & 6                  & 9                  & 1 \\
        7 & 3  & 5                  & 6                  & 2 \\
        4 & 10 & 9                  & 4                  & 12 \\
        3 & 0  & 2                  & 11                 & 7 \\
        9 & 8  & 12                 & 3                  & 7 \\
        \hline
    \end{tabular}
\end{table}

\begin{description}
    \item[Cluster Center (3, 2)] (2, 4), (3, 0)
    \item[Cluster Center (6, 8)] (5, 6), (4, 10), (9, 8)
    \item[Cluster Center (8, 2)] (8, 1), (7, 3)
\end{description}

\textit{Clusters haven't changed!} Final cluster centers and instances are as follows:

\begin{description}
    \item[Cluster Center (3, 2)] (2, 4, 11, yes), (3, 0, 3, yes)
    \item[Cluster Center (6, 8)] (5, 6, 5, no), (4, 10, 8, yes), (9, 8, 1, no)
    \item[Cluster Center (8, 2)] (8, 1, 7, no), (7, 3, 4, yes)
\end{description}

\section{Confusion Table}
\begin{enumerate}[(a)]
    \item For a randomly produced results, there were 8 values that we predicted to be B, when they were actually G.
    \item For a classifier produced results, there were 30 values that we predicted to be B, and were actually B.
    \item The non-random classifier, 90 were predicted correctly. For the random classifier, 39 were predicted correctly. Therefore, 51 more were predicted correctly.
    \item Kappa Statistic would be

        \begin{equation*}
            \frac{
                \text{Non-Random Correct - Random Correct}
            }{
                \text{Total}
            }
        \end{equation*}

        Which would be as follows:

        \begin{equation*}
            \frac{90 - 39}{100}
        \end{equation*}
\end{enumerate}


\end{document}
