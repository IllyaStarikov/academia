\documentclass{article}
\usepackage{amsmath}
\usepackage{fancyvrb}
\usepackage{forest}
\begin{document}

\fvset{
fontsize=\small,
numbers=left
}

\title{Homework \#2}
\author{Illya Starikov}
\date{Date Due: February 15, 2016}
\maketitle

\section{Programming Assignment}
Code is attached. Binary search implementation is as follows.

\begin{Verbatim}
extension Array where Element: Comparable {
    func binarySearch(key: Element) -> Int? {
        return binarySearch(key, low: 0, high: self.count - 1)
    }

    private func binarySearch(key: Element, low: Int, high: Int) -> Int? {
        if low > high { return nil }

        let middle = Int((low + high) / 2)

        if key == self[middle] {
            return middle
        } else if key > self[middle] {
            return binarySearch(key, low: middle + 1, high: high)
        } else {
            return binarySearch(key, low: low, high: middle - 1)
        }
    }
}
\end{Verbatim}

\section{Recurrence Equation}
\subsection{Tower of Hanoi}
\begin{forest}
[c
    [$\frac{c - 1}{2}$
        [$\frac{c - 1}{4}$
            [\vdots
                [$T(1)$]
                [$T(1)$]
            ]
            [\vdots]
        ]
        [$\frac{c - 1}{4}$
            [\vdots]
            [\vdots]
        ]
    ]
    [$\frac{c - 1}{2}$
        [$\frac{c - 1}{4}$
            [\vdots]
            [\vdots]
        ]
        [$\frac{c - 1}{4}$
            [\vdots]
            [\vdots]
        ]
    ]
]
\end{forest}


The complexity of this tree is $O(\lg n)$, or alternatively, $O(2^n)$
\subsection{Merge Sort}
\begin{equation}
T(n) = 2 \ T(\frac{n}{2}) + n
\end{equation}

From the equation, we can see that

\begin{equation}
a = 2, b = 2, c = 1, f(n) = n
\end{equation}

We observe that

\begin{equation}
f(n) \in \Theta(n^{\log _b a}) = \Theta (n ^{log _2 2}) = \Theta(n)
\end{equation}

It follows from Case \#2 of the Master Theorem

\begin{eqnarray}
T(n) & \in & \Theta(n^{\log _b a} \lg n) \\
& = & \Theta (n^{\log _2 2} \lg n) \\
& = & \Theta (n^1 \lg n) \\
& = & \Theta (n \lg n)
\end{eqnarray}

Thus the recurrence relation $T(n)$ is in $\Theta (n \log n)$. QED.

\section{Loop Invariant}
\begin{Verbatim}
Merge(A, p, q, r)
    leftIndex = q - p + 1
    rightIndex = r - q
    let L[1..left + 1] and R[1..right + 1] be new arrays
    for i = 1 to leftIndex
        L[i] = A[p + i - 1]
    for j = 1 to rightIndex
        R[j] = A[q + j]
    L[leftIndex + 1] = sentinel
    R[rightIndex + 1] = sentinel
    i = 1
    j = 1
    for k = p to r
        if L[i] <= R[j]
            A[k] = L[i]
            i = i + 1
        else A[k] = R[j]
            j = j + 1
\end{Verbatim}

\noindent
\textit{At the end of the for loops on 4-5, 6-7, new arrays will be created, holding the left and right half of the data in the passed array.}
\begin{description}
\item [Initialization] Initially we have two arrays, the left and right side of the original passed array. This holds trivially.
\item [Maintenance] During each iteration, we copy over the array's data.
\item [Termination] Upon termination, the Left array has the data from [$p + i - 1$] and Right array has the data of [$q + j$].
\end{description}

\noindent
\textit{During lines 12 - 17, we merge the arrays to get a properly sorted array.}
\begin{description}
\item [Initialization] Initially we have the left and right arrays, unsorted, and the original array. This holds trivially.
\item [Maintenance] During the iterations of the array, we replace the data of the originally passed array with the smaller of the Left and Right array.
\item [Termination] Upon termination, we have a fully sorted array from $q...r$.
\end{description}

\section{Quicksort}
Let $\_$ signify the pivot, $||$ signify the wall.
\begin{enumerate}
\item $||$ 4 2 7 6 3 5 \underline{1}
\item 1 $||$ $||$ 4 2 7 6 3 \underline{5}
\item 1 $||$ 4 $||$ 2 7 6 3 \underline{5}
\item 1 $||$ 4 2 $||$ 7 6 3 \underline{5}
\item 1 $||$ 4 2 3 $||$ 7 6 \underline{5}
\item 1 $||$ 4 2 \underline{3} $||$ 5 $||$ 7 6
\item 1 $||$ $||$ 4 \underline{3} $||$ 5 $||$ 7 6
\item 1 $||$ 2 $||$ 4 \underline{3} $||$ 5 $||$ 7 6
\item 1 $||$ 2 $||$ 3 $||$ 4 $||$ 5 $||$ 7 \underline{6}
\item 1 $||$ 2 $||$ 3 $||$ 4 $||$ 5 $||$ $||$ 7 \underline{6}
\item 1 $||$ 2 $||$ 3 $||$ 4 $||$ 5 $||$ 6 $||$ \underline{7}
\item 1 $||$ 2 $||$ 3 $||$ 4 $||$ 5 $||$ 6 $||$ 7

\end{enumerate}

\section{Heapsort}
\subsection{Heap Representation}
\begin{forest}
[\textbf{7}
    [4
        [1]
        [2]
    ]
    [5
        [3]
    ]
]
\end{forest}

\noindent
\begin{forest}
[\textbf{5}
    [4
        [1]
        [2]
    ]
    [3]
]
\end{forest}

\noindent
\begin{forest}
[\textbf{4}
    [2
        [1]
    ]
    [3]
]
\end{forest}

\noindent
\begin{forest}
[\textbf{3}
    [2]
    [1]
]
\end{forest}

\noindent
\begin{forest}
[\textbf{2}
    [1]
]
\end{forest}

\noindent
\begin{forest}
[\textbf{1}]
\end{forest}

\subsection{Enumerated Steps}
\begin{enumerate}
\item $7$
\item $7 \ 5$
\item $7 \ 5 \ 4$
\item $7 \ 5 \ 4 \ 3$
\item $7 \ 5 \ 4 \ 3 \ 2$
\item $7 \ 5 \ 4 \ 3 \ 2 \ 1$
\end{enumerate}
\end{document}