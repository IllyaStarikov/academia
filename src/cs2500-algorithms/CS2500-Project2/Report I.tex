\documentclass{article}
\usepackage[colorlinks = true,
            linkcolor = blue,
            urlcolor  = blue,
            citecolor = blue,
            anchorcolor = blue]{hyperref}

\title{Programming Project I, First Report}
\author{Illya Starikov, Claire Trebing, Timothy Ott}
\date{Due Date: April 22, 2016}

\begin{document}
\maketitle

\section{Abstract}
Social Networks have revolutionized the way we communicate, meet others, consume information, and essentially influence our day-to-day lives. To show Social Network's prominence, \href{http://www.pewinternet.org/fact-sheets/social-networking-fact-sheet/}{here are the percentages of online adults who use social media}:

\begin{description}
    \item [Facebook]: 71\% Adults
    \item [Twitter]: 23\% Adults
    \item [Instagram]: 26\% Adults
    \item [Pinterest]: 28\% Adults
    \item [LinkedIn]: 28\% Adults
\end{description}

This is unprecedented. To say that if one understands the network, they understand the community is an understatement. In this experiment we would like to put test this theory.

\section{Introduction and Motivation}
As stated above, social networks play a dominant role in our lives. They are many reasons for this to be valuable.

\section{Proposed Solutions}

\section{Plan of Experiments}

\section{Team Roles}
\begin{itemize}
    \item Illya Starikov
    \begin{itemize}
        \item Project Manager
        \item Official Write-up
    \end{itemize}
    \item Timothy Ott
    \begin{itemize}
        \item Pseudocode Write-up
        \item Algorithm Analysis
    \end{itemize}
    \item Claire Trebing
    \begin{itemize}
        \item Pseudocode Write-up
        \item Algorithm Analysis
    \end{itemize}

\end{itemize}

\end{document}