\documentclass{article}
\usepackage[colorlinks = true,
            linkcolor = blue,
            urlcolor  = blue,
            citecolor = blue,
            anchorcolor = blue]{hyperref}

\title{Programming Project II, First Report}
\author{Illya Starikov, Claire Trebing, Timothy Ott}
\date{Due Date: April 22, 2016}

\usepackage{amsmath}
\usepackage{amssymb}

\usepackage{xcolor}
\newcommand{\shellcmd}[1]{\texttt{\colorbox{gray!30}{#1}}}

\begin{document}
\maketitle

\section{Abstract}
Social Networks have revolutionized the way we communicate, meet others, consume information, and essentially influence our day-to-day lives. To show Social Network's prominence, \href{http://www.pewinternet.org/fact-sheets/social-networking-fact-sheet/}{here are the percentages of online adults who use social media}:

\begin{description}
    \item [Facebook]: 71\% Adults
    \item [Twitter]: 23\% Adults
    \item [Instagram]: 26\% Adults
    \item [Pinterest]: 28\% Adults
    \item [LinkedIn]: 28\% Adults
\end{description}

This is unprecedented. Seeing as an overwhelming majority of adults have some sort of social media account, this can be used to model real world relationships --- through social graphs.

In this experiment we would like to examine the social graph through the follower-followee relationship.

\section{Introduction and Motivation}
As stated above, social networks play a dominant role in our lives. Through social graph's, we can examine real world relationships.

There are many reasons for this to be valuable, such as common interest, community detection, influence amongst followers, etc. For this experiment, we would just like to test on the following measures:

\begin{description}
    \item [Degree Distribution] Test the direct amount of followers compared to followees a person has.
    \item [Shortest Path Distribution] Test the degree of separation between a follower-followee.
    \item [Graph Diameter] Test the breadth of a community.
    \item [Closeness Centrality] Test the dependence of a degree of separation on a singular person.
    \item [Betweenness Centrality Distribution] Same as previous.
    \item [Community Detection] Based on Closeness Centrality, find the diameter of a community.
\end{description}

This will give us a reasonable dataset for the relationship of a community in a social graph.

\section{Proposed Solutions}
\subsection{Shortest Path}
\begin{verbatim}
V = the number of vertices
distance = new array[V][V]
Initialize distance to -1   //since there are no negative weights this will
                            //represent infinity
Floyd(V):
    for each vertex v:
        distance[v][v] = 0
    for each edge (u,v)
        distance[u][v] = w(u,v) //the weight of the edge (u,v)
    for k from 1 to V
        for i from 1 to V
            for j from 1 to V
              if distance[i][j] > distance[i][k] + distance[k][j]
                distance[i][j] = distance[i][k] + distance[k][j]

shortest = new array[100]
initialize shortest to 0
for i from 1 to V
    for j from 1 to V
        if distance[i][j] > 0
          if distance[i][j] > shortest.length
              temp = new array[distance[i][j]+10]
              for i from 0 to shortest.length
                  temp[i] = distance[i]
              delete distance
              distance = temp
          shortest[distance[i][j]]++

for i from 1 to shortest.length
    if shortest[i] > 0
        output shortest[i]
\end{verbatim}

\subsubsection{Complexity Analysis}
As we know, the Floyd-Warshall algorithm operates on $n^3$ time. Combining this with the algorithm to iterate over the resulting V x V matrix results in a $n^2 + n^3$ complexity --- or, $\mathcal{O}(n^3)$ time.

\subsection{Closeness Centrality}
\begin{verbatim}
sum = 0
closeness = new array[100]
initialize closeness to 0
for i from 0 to V
    for j from 1 to V
        sum += distance[i][j]
    closeness[i] = 1/sum
    output closeness[i]
    sum = 0
\end{verbatim}

\subsubsection{Complexity Analysis}
Because we are iterating over the entire matrix resulting from the Floyd-Warshall algorithm of $V \times V$ size. We arrive at a complexity of $\mathcal{O}(n^2)$.

\subsection{Community Detection}
\begin{verbatim}
UWBetween[] = Unweighted Betweenness Centrality Edges
V = Original Network

For k from 0 to 4:
    DescendSort(UWBetween)
    UWBetween[0] = x
  while (UWBetween[0] == x)
      V.remove(UWBetween[i])
      UWBetween.remove(UWBetween[0])
  Floyd(V)            //Run Floyd-Warshall Algorithm on revised data set
  Diameter(V)         // Calculate and output max diameter based on new matrix from above
  Betweenness(V)      // Calculate Betweenness centrality based on
                      // new matrix giving us a new UWBetween array
\end{verbatim}

\subsubsection{Complexity Analysis}
Because the Community detection algorithm calls the algorithms to find all shortest paths, the diameter and the unweighted betweenness centrality edges within it and since we are executing this algorithm a total of five times, the complexity of this algorithm is five times the sum of the complexities of these algorithms.

\section{Plan of Experiments}
The major purpose of our experiment is to dissect and examine networks, so we propose the following plan of experiments:

\begin{enumerate}
    \item Extrapolate data and store in a relevant data structure --- in our case, a sorted \texttt{map<int, vector<pair<int, double>>>}
    \begin{itemize}
        \item The \texttt{int} is the key to be used, signifying the origin.
        \item The \texttt{vector<pair>} holds all the adjacent edges.
    \end{itemize}
    \item Iterate over the entirety of the data structure to determine degree distribution.
    \begin{itemize}
        \item For \emph{weighted and unweighted out degree}, it is simply counting the the \texttt{vector} size with respect to each key.
        \item For \emph{weighted and unweighted in degree}, making an efficient algorithm is still difficult --- not only is by default $\mathcal{O}(n)$ but there is an efficient way of finding where the edges lead to. We accomplish this by a \href{https://en.wikipedia.org/wiki/Binary_search_tree}{Binary Search Tree}. By iterating over every vertex and storing their destination in a binary search tree, we have achieved a sufficient algorithm.
    \end{itemize}
    \item Detect the shortest path via the \href{https://en.wikipedia.org/wiki/Floyd–Warshall_algorithm}{Floyd-Warshall algorithm} for directed graphs.
        \begin{itemize}
            \item Make the graph undirected by making it symmetric about the $a _{i, i}$ elements $\forall i \in \text{edges}$.
            \item Test again.
        \end{itemize}
    \item Detect closeness centrality and betweenness.
    \item Implement community detection.
    \item Output results to user.
\end{enumerate}

\section{Team Roles}
\begin{itemize}
    \item Illya Starikov
    \begin{itemize}
        \item Project Manager
        \item Official Write-up
    \end{itemize}
    \item Timothy Ott
    \begin{itemize}
        \item Pseudocode Write-up
        \item Algorithm Analysis
    \end{itemize}
    \item Claire Trebing
    \begin{itemize}
        \item Pseudocode Write-up
        \item Algorithm Analysis
    \end{itemize}
\end{itemize}

\end{document}