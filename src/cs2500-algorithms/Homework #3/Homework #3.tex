\documentclass{article}
\title{Homework \#3}
\author{Illya Starikov}
\date{Due Date: February 28, 2016}

\usepackage{forest,amsmath,amssymb}

\let\emptyset\varnothing

\begin{document}

\maketitle

\section{Matrix Multiplication}
\begin{tabular}{ c|c|c|c|c| }
  & 1 & 2 & 3 & 4 \\ \hline
1 & 0 & $20 ^{[1]}$ & $35^{[2]}$ & $65^{[3]}$  \\ \hline
2 & X & 0 & $12 ^{[2]}$ & $28^{[3]}$ \\ \hline
3 & X & X & 0 & $6^{[3]}$ \\ \hline
4 & X & X & X & 0  \\ \hline
\end{tabular}

\noindent
Therefore, the minimum scalar operations can be achieved via $(((A_1 \times A_2) \times A_3) \times A_4)$

\section{Longest Common Subsequence}
\subsection{Recursion Tree}
\begin{forest}
[$\binom{\text{AMBE}}{\text{ACME}}$ \textbf{E}
    [$\binom{\text{AMB}}{\text{ACM}}$
        [$\binom{\text{AM}}{\text{ACM}}$ \textbf{M}
            [$\binom{\text{A}}{\text{AC}}$
                [$\binom{\emptyset}{\text{AC}}$
                    [$\binom{\emptyset}{\text{A}}$
                        [$\emptyset$]
                    ]
                ]
                [$\binom{\text{A}}{\text{A}}$ \textbf{A}
                    [$\emptyset$]
                ]
            ]
        ]
        [$\binom{\text{AMB}}{\text{AC}}$
            [$\binom{\text{AM}}{\text{AC}}$
                [$\binom{\text{A}}{\text{AC}}$
                    [$\binom{\emptyset}{\text{AC}}$
                        [$\binom{\emptyset}{\text{A}}$
                            [$\emptyset$]
                        ]
                    ]
                    [$\binom{\text{A}}{\text{A}}$ \textbf{A}
                        [$\emptyset$]
                    ]
                ]
                [$\binom{\text{AM}}{\text{A}}$
                    [$\binom{\text{A}}{\text{A}}$ \textbf{A}
                        [$\emptyset$]
                    ]
                    [$\binom{\text{AM}}{\emptyset}$
                        [$\binom{\text{A}}{\emptyset}$
                            [$\emptyset$]
                        ]
                    ]
                ]
            ]
            [$\binom{\text{AMB}}{\text{A}}$
                [$\binom{\text{AM}}{\text{A}}$
                    [$\binom{\text{A}}{\text{A}}$ \textbf{A}
                        [$\emptyset$]
                    ]
                    [$\binom{\text{AM}}{\emptyset}$
                        [$\binom{\text{A}}{\emptyset}$
                            [$\emptyset$]
                        ]
                    ]
                ]
                [$\binom{\text{AMB}}{\emptyset}$
                    [$\binom{\text{AM}}{\emptyset}$
                        [$\binom{\text{A}}{\emptyset}$
                            [$\emptyset$]
                        ]
                    ]
                ]
            ]
        ]
    ]
]
\end{forest}

\noindent
Therefore, the longest common substring is AME.

\subsubsection{Trace Table}
\begin{tabular}{ |c|c|c|c|c|c| }
 \hline
 & $\emptyset$ & A & M & B & E \\ \hline
$\emptyset$ & 0 & \textbf{0} & 0 & 0 & 0 \\ \hline
A & 0 & 1 & \textbf{1} & 1 & 1 \\ \hline
C & 0 & 1 & \textbf{2} & 1 & 1 \\ \hline
M & 0 & 1 & \textbf{2} & \textbf{2} & 2 \\ \hline
E & 0 & 1 & 2 & 2 & \textbf{3} \\ \hline
\end{tabular}

\noindent
Therefore, the longest common substring is AME.
\end{document}