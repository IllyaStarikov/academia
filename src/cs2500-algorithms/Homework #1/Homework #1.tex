\documentclass{article}
\usepackage{amsmath}
\usepackage{amssymb}
\usepackage[english]{babel}
\usepackage{amsthm}
\begin{document}

\author{Illya Starikov}
\title{Homework \#1}
\date{Due Date: February 3, 2016}
\maketitle

\section{Insertion Sort}
\subsection{Operations of Insertion Sort}
\begin{eqnarray*}
\textbf{3} \ \underline{41} \ 52 \ 26 \ 38 \ 57 \ 9 \ 49  & & \text{1 Comparison(s).} \\
\textbf{3 41} \ \underline{52} \ 26 \ 38 \ 57 \ 9 \ 49  & & \text{1 Comparison(s).} \\
\textbf{3 41 52} \ \underline{26} \ 38 \ 57 \ 9 \ 49  & & \text{3 Comparison(s).} \\
\textbf{3 26 41 52} \ \underline{38} \ 57 \ 9 \ 49  & & \text{3 Comparison(s).} \\
\textbf{3 26 38 41 52} \ \underline{57} \ 9 \ 49  & & \text{1 Comparison(s).} \\
\textbf{3 26 38 41 52 57} \ \underline{9} \ 49  & & \text{6 Comparison(s).} \\
\textbf{3 9 26 38 41 52 57} \ \underline{49}  & & \text{3 Comparison(s).} \\
\textbf{3 9 26 38 41 49 52 57} & & \text{0 Comparison(s).}
\end{eqnarray*}

\subsection{Number of Comparisons}
18 total operations.

\section{Question 2}
\begin{center}
\begin{tabular}{| l | l | c | c | c | c | c |}
 \hline
B & A & O & o & $\Omega$ & $\omega$ & $\Theta$ \\ \hline
$2^n$ & $2^\frac{n}{2}$ & no & no & yes & yes & no \\ \hline
$n^{\lg c}$ & $c^{\lg n}$ & yes & no & yes & no & yes \\ \hline
\end{tabular}
\end{center}

\subsection{Justification}
To determine larger asymptotic growth, take the limit of one function over the other. Arbitrarily choosing $2^n$ for the numerator, we see that:

\begin{equation}
\lim _{n \rightarrow \infty} \frac{2^n}{2^{\frac{n}{2}}} = \lim _{n \rightarrow \infty} 2^{\frac{n}{2}} = \infty
\end{equation}

Alternatively, if we chose $2^n$ as the denominator, we notice that $\lim _{n \rightarrow \infty} \frac{2^{\frac{n}{2}}}{2^n} = 0$, so we know that the condition $2^n > 2^{\frac{n}{2}}$ holds, and for certain conditions the functions are equal (take $n = 0, 2^{0} = 2^{\frac{0}{2}}$).

$\therefore$ $2^n$ is $\omega(2^\frac{n}{2})$ and $\Omega(2^\frac{n}{2})$. QED.

\subsection{Justification II}
By the properties of logarithms, we will show that $n^{\lg c} = c^{\lg n}$.

\begin{eqnarray}
n^{\lg c} & = & c^{\lg n} \\
& = & c^{\log _2 n} \\
& = & c^\frac{\ln n}{\ln 2} \\
& = & e^{\ln c \times \frac{\ln n}{\ln 2}} \\
& = & e^{\frac{\ln c}{\ln 2} \times \ln n} \\
& = & e^{\lg c \ln n} \\
& = & n^{\lg c}
\end{eqnarray}

$\therefore n^{\lg c}$ is $O(c^{\lg n}), \ \Omega(c^{\lg n})$ and $\Theta(c^{\lg n})$. QED.

\section{Big-$O$ Implies Big-$\Omega$}
\textbf{Theorem: } Let f(n) and g(n) be asymptotically positive functions, $O(g(n))$ be the set $\{ f(n) : \exists c, n_0 \in \mathbb{R}^+, \forall n, n_0 \in \mathbb{R}, 0 \leq f(n) \leq c \ g(n) \wedge n > n_0  \}$, and $\Omega(g(n))$ be the set $\{ f(n) : \exists c, n_0 \in \mathbb{R}^+, \forall n, n_0 \in \mathbb{R}^+, 0 \leq cg(n) \leq f(n) \wedge n > n_0 \}$ Then,

\begin{equation}
f(n) = O(g(n)) \Rightarrow g(n) = \Omega (f(n))
\end{equation}

[\textit{We will prove so by contradiction.}]

\noindent
\textbf{Proof: } Suppose not. That is, suppose

\begin{eqnarray}
\sim [f(n) = O(g(n)) & \Rightarrow & g(n) = \Omega (f(n))] \\
f(n) = O(g(n)) & \wedge & \sim [g(n) = \Omega (f(n))]
\end{eqnarray}

Using the formal definition of $\Omega(g(n))$, we can see the negation is as follows:

\begin{equation}
\sim \Omega(g(n)) = \{ f(n) : \forall c, n_0 \in \mathbb{R}^+, \exists n, n_0 \in \mathbb{R}^+ 0, > cg(n) > f(n) \wedge n \leq n_0 \}
\end{equation}

This is a contradiction, for there are no $c, n_0$ such that $0 \leq c g(n) \leq f(n) \wedge 0 > cf(n) > g(n)$ because $c, n_0$ are defined as \textit{positive} constants, and $f(n), g(n)$ are defined as \textit{positive} functions. Because no such positive function and positive constants exists to satisfy

\begin{equation}
0 > cf(n) > g(n)
\end{equation}

This has led us to a contradiction. QED.

\section{Asymptotic Proofs}
\subsection{Proof I}
Assuming $n > 1$,

\begin{eqnarray}
n^2 & \leq & 20n^2 + 2n + 5 \\
& \leq & 20n^2 + 2n^2 + 5 \\
& \leq & 27n^2
\end{eqnarray}

$\therefore$ For $C = 27, n_0 = 1$, $20n^2 + 2n + 5 = O(n^2)$. QED.

\subsection{Proof II}
Assume $C = 1, n_0 = 1$. $\therefore $This satisfies the condition that
$c$ and $n_0$ are positive constants such that $0 \leq Cn^2 \leq 5n^2 - 15n + 100 \ \forall n \geq n_0$. QED.

\subsection{Proof III}
\subsubsection{Lower Bound}
Assume $C = 1, n_0 = 1$. $\therefore $This satisfies the condition that
$c$ and $n_0$ are positive constants such that $0 \leq Cn^2 \leq 5n^2 + 2n \ \forall n \geq n_0$. QED.

\subsubsection{Upper Bound}
Assuming $n > 1$,

\begin{eqnarray}
n^2 & \leq & 5n^2 + 2 \\
& \leq & 5n^2 + 2n^2 \\
& \leq & 7n^2
\end{eqnarray}

$\therefore C = 7, n_0 = 1, 5n^2 + 2n = \Theta(n^2)$. QED.

\subsection{Proof IV}
To prove that $5n+7 = o(n^2)$ we must show that $\exists c, n_0 \in \mathbb{R}^+, 0 \leq f(n) < Cg(n) \wedge n > n_0$. We do so by showing that $\lim _{n \rightarrow \infty} \frac{f(n)}{g(n)} = 0$. [\textit{with the help of L'H\^{o}pital's Rule}]

\begin{eqnarray}
& & \lim _{n \rightarrow \infty} \frac{5n+ 1}{n^2} \\
& = & \lim _{n \rightarrow \infty} \frac{5}{2n} \\
& = & 0
\end{eqnarray}

QED.
\end{document}