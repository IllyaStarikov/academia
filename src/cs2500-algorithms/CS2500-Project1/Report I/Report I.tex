\documentclass{article}

\title{Programming Project I, First Report}
\author{Illya Starikov, Claire Trebing, \& Timothy Ott}
\date{Due Date: March 07, 2016}

\usepackage{amsmath, amssymb}
\usepackage{hyperref}
\hypersetup{
    colorlinks=true,
}

\usepackage{listings}
\usepackage{courier}
\lstset{
  basicstyle=\ttfamily,
  columns=fullflexible,
  keepspaces=true,
}


\begin{document}
\maketitle

\section{Abstract}
Smartphone users launch many apps everyday, however one of the most fundamental things a smartphone does is abstracted away: memory management.

Although smartphones have advanced significantly in many ways compared to their first predecessors (in terms of RAM, architecture, processors), deactivation, (the process of ``the operating system needing to choose and remove some apps from the memory'', a subproblem of \href{https://en.wikipedia.org/wiki/Memory_management}{memory management}) is a solution that is often less-than-perfect. Although Java's \href{http://www.oracle.com/webfolder/technetwork/tutorials/obe/java/gc01/index.html}{Garbage Collection} and Swift's \href{https://developer.apple.com/library/ios/documentation/Swift/Conceptual/Swift_Programming_Language/AutomaticReferenceCounting.html}{Automatic Reference Counting} (ARC) have sufficed, there are other methods.

In this project I propose to solve this problem using three techniques:

\begin{itemize}
    \item Brute Force
    \item Dynamic Programming
    \item Greedy Solution
\end{itemize}

\section{Introduction and Motivation}
As stated previously, memory management is solved in a less-than-perfect manner. Although current technology suffices, we would like to compare algorithms to show the significant gains via three different approaches (Brute Force, Dynamics Programming, and Greedy).

\section{Proposed Solution}
For our project we decided to take a more \href{https://en.wikipedia.org/wiki/Skeuomorph}{skeuomorphic} and object oriented approach, modeling objects after their real world counterparts, such as \texttt{Application} or \texttt{Smartphone}. As for the approaches, we have the following solutions:

\subsection{Brute Force}
For the brute force method, we knew that we had to check every possible subset (and for a set of size $n$, we know there to be \href{http://mathworld.wolfram.com/Subset.html}{$2^n$ subsets}). We used this to our advantage, creating a lookup table modeled after a truth table. As an example, suppose we have a three items in our knapsack:

\begin{center}
    \begin{tabular}{|c|c|c|c|}
        \hline
        Items A & Item B & Item C & Item Number  \\ \hline
        0 & 0 & 0 & 0\\
        0 & 0 & 1 & 1\\
        0 & 1 & 0 & 2\\
        0 & 1 & 1 & 3\\
        1 & 0 & 0 & 4\\
        1 & 0 & 1 & 5\\
        1 & 1 & 0 & 6\\
        1 & 1 & 1 & 7\\
        \hline
    \end{tabular}
\end{center}

We notice that the knapsack combination can be directly summarized by the $n$-bit binary representation. Using the previous example of three item knapsack:

\begin{eqnarray}
    \text{Subset \#0}: 0_{10} = 000_{2} = \sim(ABC) & = & \text{No Items} \\
    \text{Subset \#5}: 5_{10} = 101_{2} = A\sim(B)C & = & \text{Items A, B} \\
    \text{Subset \#7}: 7_{10} = 111_{2} = ABC & = & \text{Items A, B, C}
\end{eqnarray}

Now that we have have a represenation of every subset, we can multiply the columns value by the knapsack value to get the item's value into the lookup table. Extending the example, suppose we have the following table:

\begin{center}
    \begin{tabular}{|c|c|c|}
        \hline
        App & Memory & Cost \\ \hline
        A & 2 & 4 \\
        B & 3 & 6 \\
        C & 7 & 9 \\
        \hline
    \end{tabular}
\end{center}

Multiplying the columns by the values produces the following results:

\begin{center}

    \begin{tabular}{|c|c|c|}
        \hline
        App A & App B & App C   \\ \hline
        $0 \times 4$ & $0 \times 6$ & $0 \times 9$ \\
        $0 \times 4$ & $0 \times 6$ & $1 \times 9$ \\
        $0 \times 4$ & $1 \times 6$ & $0 \times 9$ \\
        $0 \times 4$ & $1 \times 6$ & $1 \times 9$ \\
        $1 \times 4$ & $0 \times 6$ & $0 \times 9$ \\
        $1 \times 4$ & $0 \times 6$ & $1 \times 9$ \\
        $1 \times 4$ & $1 \times 6$ & $0 \times 9$ \\
        $1 \times 4$ & $1 \times 6$ & $1 \times 9$ \\
        \hline
    \end{tabular} \quad
    = \quad
    \begin{tabular}{|c|c|c|}
        \hline
        App A & App B & App C \\ \hline
        0 & 0 & 0\\
        0 & 0 & 9\\
        0 & 6 & 0\\
        0 & 6 & 9\\
        4 & 0 & 0\\
        4 & 0 & 9\\
        4 & 6 & 0\\
        4 & 6 & 9\\
        \hline
    \end{tabular}
\end{center}

Adding the columns across we produce the following, and factoring the weight:
\begin{center}
    \begin{tabular}{|c|c|c|c|c|}
        \hline
        Items A & Item B & Item C & Total Cost & Total Memory \\ \hline
        0 & 0 & 0 & 0 & 0\\
        0 & 0 & 9 & 9 & 7\\
        0 & 6 & 0 & 6 & 3\\
        0 & 6 & 9 & 15 & 10\\
        4 & 0 & 0 & 4 & 2\\
        4 & 0 & 9 & 12 & 9\\
        4 & 6 & 0 & 10 & 5\\
        4 & 6 & 9 & 19 & 12\\
        \hline
    \end{tabular}
\end{center}

Now finding the minimal value is straightforward, find the minimal value is traversing down the additive value while looking at the size it frees up. Suppose we have to free up 10 blocks of memory, items $A, B$ would be the most efficient choice.

The pseudocode is as follows, with a few notes:

\begin{itemize}
    \item Comparison is done while ``generating'' a table.
    \item A table is not generate, but the binary representation is used instead.
\end{itemize}

\subsubsection{Pseudocode}
\begin{lstlisting}[mathescape]
knapsackBrute(items, knapsackSize)
    minumum = 0
    for i = 0 to $2^n$ - 1
        subset = binaryToInteger(i)
        sum = 0

        for i to subset.length
            sum = sum + item[i].benefit * subset[i]
            size = size + item[i].weight * subset[i]

        if size <= knapsackSize && sum < minumum
            minumum = sum
            greatestSubset = i

    subset = binaryToInteger(greatestSubset)
    for i = 0 to subset.size
        if subset[i] == 1
            optimalSolution.append(item[i])

    return optimalSolution
\end{lstlisting}

\subsubsection{Time Complexity}
Analyzing this algorithm, we can see the complexity:

\begin{equation}
    \sum _{k = 1} ^{n} 2^k \times \sum _{k = 1} ^{n} c = \mathcal{O}(n 2^n)
\end{equation}

We run in $\mathcal{O}(n 2^n)$ time.

\subsection{Dynamic Programming}
The approach to dynamic programming came from creating a table of the possible memory and cost combinations. The table would contain rows for the values from 0 to M, the largest possible memory we would have to free up. The program starts by finding the lowest amount of memory an app can take up and the lowest cost. This will act as a base case. For example, if the lowest memory amount is 3 and the cost is 1, then if the user needed to clear out 1, 2, or 3 memory spaces then the phone would close that app. Amounts greater than 4 would be calculated when the program started.

\subsubsection{Pseudocode}
\begin{lstlisting}[mathescape]
Create_Delete_Table(numberOfApps, numberToDelete, Memory[], Cost[]){
	Let d[1 ... M, 3] be new table as shown below
	Memory To Delete | Nr. of Apps | Cost    |  List of Apps Used
		0
		1
		2
		3
		...
		M

	int low = findMin(d[],Memory[], Cost[]);
    //This function will find a base case for an app with
    the lowest memory and lowest cost
	for (j = 1 to low)
	{
		d[j, 0] = d[low, 0];
		d[j, 1] = d[low, 1];
		d[j, 2] = d[low, 2];
		d[j, 3] = d[low, 3];
	}

	for(j = low to M)
	{
		for( i = 1 to N){
			if(Mi == j && Ci < d[j-low,2] + d[low,2]){
				d[j, 1] = 1;
				d[j, 2] = Ci;
				d[j, 3] = i;
			}

			else{
				d[j,1] = d[j-low,1]++;
				d[j,2] = d[j-low,2] + d[low,2];
				//add app to the list
			}
		}
	}
	return;
}
\end{lstlisting}

\subsubsection{Time Complexity}
We can see that this algorithm is $\mathcal{O}(n^2)$.

\subsection{Greedy Solution}
After considering several greedy algorithm solutions we settled on selecting based on a ratio of cost and memory freed by deactivating a particular app. Once each of the apps’ ratio has been evaluated (and the ratio has been stored as a member variable) we sort the array of apps in an ascending order according to this new value. Because the apps are now in order in ascending cost per unit of memory we can now simply add apps to the solution set in the order they appear in the array until our memory requirements are met. For example, if we take our example set of three items:

\begin{center}
    \begin{tabular}{|c|c|c|}
        \hline
        App & Memory & Cost \\ \hline
        A & 2 & 4 \\
        B & 3 & 6 \\
        C & 7 & 9 \\
        \hline
    \end{tabular}
\end{center}

We first take the ratio of each item’s cost over the capacity of memory it takes up and sort the elements accordingly. Doing this gives us the following updated table:

\begin{center}
    \begin{tabular}{|c|c|c|c|}
        \hline
        App & Memory & Cost & Ratio \\ \hline
        C & 7 & 9 & 1.29 \\
        A & 2 & 4 & 2 \\
        B & 3 & 6 & 2 \\
        \hline
    \end{tabular}
\end{center}

And now we simply add apps until our memory requirement has been met or exceeded. Suppose we need to free up 9 units of memory, according to this approach the optimal solution would be items C and A.

It is worth noting, however, that this solution doesn’t always produce the absolute optimal solution. If we use the previous example of freeing up 10 units of memory this algorithm produces a solution of all the apps, A, B and C as opposed to the actual optimal solution of app C and B.  This inaccuracy is a sacrifice in order to benefit from a much lower runtime than that of the other solutions.

We could perhaps introduce a second greedy choice or to further refine our algorithm to improve the accuracy but we would do so at the cost of performance and the solutions chosen by our initial solution are accurate enough for our purposes.

\subsubsection{Pseudocode}
\begin{lstlisting}[mathescape]
knapsackGreedy()
    A[] = Array of Apps
    S[] = Array of solution set
    M = Desired amount of memory to be freed
    sm = Amount of memory freed by solution set

    n = A.length

    for 0 to n
      A[i].ratio = A[i].cost/A[i].memory

    mergesort(A)

    GreedKnap(A, S)
      S[0] = A[0]
      sm = A[0].memory
      i = 1
      while (sm < M AND i < n)
        S[i] = A[i]
        sm += A[i].memory
\end{lstlisting}

\subsubsection{Time Complexity}
Analyzing this algorithm, including both going over the array to produce the ratio that will guide our greedy choices and choosing the actual solution set, we come up with the complexity of:

\begin{equation}
    \sum _{k = 1} ^{n} k + \sum _{k = 1} ^{n} k = \mathcal{O}(2n)
\end{equation}

This algorithm therefore runs in $\mathcal{O}(2n)$ time.

\section{Plan of Experiments}
The main objective of our experiments is to extrapolate time and efficiency from the algorithms for sufficiently large input, so our plan of experiments is thus:

\begin{enumerate}
    \item Generate a random array of apps, and a smartphone with $n$ memory.
    \item Take a date stamp.
    \item Run the Brute Force algorithm for the random array of apps and the smartphone.
    \item Take another date stamp, record the difference.
    \item Return to step 2 and run the Dynamic Programming algorithm.
    \item Likewise, return to step 2 and run the Greedy algorithm.
    \item Record the results from the experiment.
    \item Return to step 1, repeat 9 additional times, take the average.
    \item Run the algorithm again for a scale factor of $n$ (the input size).
\end{enumerate}


\section{Team Roles}
\begin{description}
    \item [Illya Starikov] Project Management, Development (Brute Force)
    \item [Timothy Ott] Development (Greedy), Architecture
    \item [Claire Trebing] Development (Dynamic), Quality Assurance,  Documentation
\end{description}
\end{document}