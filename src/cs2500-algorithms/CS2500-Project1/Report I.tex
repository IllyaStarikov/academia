\documentclass{article}

\title{Programming Project I, First Report}
\author{Illya Starikov, Claire Trebing, \& Timothy Ott}
\date{Due Date: March 07, 2016}

\usepackage{hyperref}
\hypersetup{
    colorlinks=true,
}

\usepackage{listings}
\usepackage{courier}
\lstset{
  basicstyle=\ttfamily,
  columns=fullflexible,
  keepspaces=true,
}


\begin{document}
\maketitle

\section{Abstract}
Smartphone users launch many apps everyday, however one of the most fundamental things a smartphone does is abstracted away: memory management.

Although smartphones have advance significantly (RAM, architecture, processors) compared to their first predecessor's, deactivation \footnote{The process of ``the operating system needing to choose and remove some apps from the memory'', a subproblem of \href{https://en.wikipedia.org/wiki/Memory_management}{memory management}.} is a solution that is often less-than-perfect. Although Java's \href{http://www.oracle.com/webfolder/technetwork/tutorials/obe/java/gc01/index.html}{Garbage Collection} and Swift's \href{https://developer.apple.com/library/ios/documentation/Swift/Conceptual/Swift_Programming_Language/AutomaticReferenceCounting.html}{Automatic Reference Counting} (ARC) have sufficed, there are other methods.

In this project I propose to solve this problem three techniques:

\begin{itemize}
    \item Brute Force
    \item Dynamic Programming
    \item Greedy Solution
\end{itemize}

\section{Introduction and Motivation}
As stated previously, memory management is solved in a less-than-perfect manner. Although current technology suffices, we would like to compare algorithms to show the significant gains via three different approaches (Brute Force, Dynamics Programming, and Greedy).

\section{Proposed Solution}
For our project we decided to take a more \href{https://en.wikipedia.org/wiki/Skeuomorph}{skeuomorphic} and object oriented approach, modeling objects after their real world counterparts, such as \texttt{Application} or \texttt{Smartphone}. As for the approaches, we have the following solutions:

\subsection{Brute Force}
For the brute force method, we knew that we have to check every possible subset (and for a set of size$n$ )

\begin{lstlisting}[mathescape]
knapsackBrute(items, napsackSize)
    max = 0
    for i = 0 to $2^n$ - 1
        subset = binaryToInteger(i)
        sum = 0

        for i to subset.length
            sum = sum + item[i].benefit * subset[i]
            size = size + item[i].weight * subset[i]

        if size <= napsackSize && sum > max
            max = sum
            greatestSubset = i

    subset = binaryToInteger(greatestSubset)
    for i = 0 to subset.size
        if subset[i] == 1
            optimalSolution.append(item[i])

    return optimalSolution
\end{lstlisting}
\subsection{Dynamic Programming}
\subsection{Greedy Solution}


\section{Plan of Experiments}


\section{Team Roles}
\begin{description}
    \item [Illya Starikov] Project Management, Development
    \item [Timothy Ott] Development (Lead), Architecture
    \item [Claire Trebing] Development, Quality Assurance,  Documentation
\end{description}

\end{document}