\documentclass{article}

\title{Programming Project I, First Report}
\author{Illya Starikov \& Timothy Ott}
\date{Due Date: March 07, 2016}

\usepackage{hyperref}
\hypersetup{
    colorlinks=true,
}

\begin{document}

\maketitle

\section{Abstract}
Smartphone users launch many apps everyday, however one of the most fundamental things a smartphone does is abstracted away: memory management.

Although smartphones have advance significantly (RAM, architecture, processors) compared to their first predecessor's, deactivation \footnote{The process of ``the operating system needing to choose and remove some apps from the memory'', a subproblem of \href{https://en.wikipedia.org/wiki/Memory_management}{memory management}.} is a solution that is often less-than-perfect. Although Java's \href{http://www.oracle.com/webfolder/technetwork/tutorials/obe/java/gc01/index.html}{Garbage Collection} and Swift's \href{https://developer.apple.com/library/ios/documentation/Swift/Conceptual/Swift_Programming_Language/AutomaticReferenceCounting.html}{Automatic Reference Counting} (ARC) have sufficed, there are other methods.

In this project I propose to solve this problem three techniques:

\begin{itemize}
    \item Brute Force
    \item Dynamic Programming
    \item Greedy Solution
\end{itemize}

\section{Introduction and Motivation}
\section{Proposed Solution}
\section{Plan of Experiments}
\section{Team Roles}

\end{document}