%
%  homework-1.tex
%  cs5200-analysis-of-algorithms
%
%  Created by Illya Starikov on 08/25/17.
%  Copyright 2017. Illya Starikov. All rights reserved.
%

\RequirePackage[l2tabu, orthodox]{nag}
\documentclass[12pt]{scrartcl}


\newcommand{\homeworknumber}{1}
\newcommand{\homeworkdue}{September 1\textsuperscript{st}, 2017}
\usetheme[logo=template-presentation/logo,faculty=ped]{fibeamer}

\usepackage{amssymb,amsmath,verbatim,graphicx,pdfpages,microtype,units,booktabs,upquote,xcolor,siunitx,csquotes,fancyvrb,newverbs,wrapfig,multicol,tikz,textcomp,wrapfig,cutwin}
\usepackage{fontawesome,setspace,rotchiffre,lipsum,listings,animate,listings}
\usepackage[xspace]{ellipsis}

\hypersetup{%
            colorlinks = true,
            linkcolor = orange,
            urlcolor  = orange,
            citecolor = orange,
            anchorcolor = orange}

\newcommand{\hugeslide}[1]{%
\begin{frame}[plain,c]
    \centering {\usebeamerfont*{frametitle} \usebeamercolor[fg]{frametitle}{\fontsize{40}{50}\selectfont\textit{#1}}}
\end{frame}
}

\newcommand{\presentaddcount}[1]{\addtocounter{#1}{1}\Roman{#1}}
\newcommand{\presentcount}[1]{\Roman{#1}}
\newcommand{\shellcmd}[1]{\texttt{\colorbox{gray!30}{#1}}}

\lstdefinelanguage{swift}
{%
  morekeywords={%
    func,if,then,else,for,in,while,do,switch,case,default,where,break,continue,fallthrough,return,
    typealias,struct,class,enum,protocol,var,func,let,get,set,willSet,didSet,inout,init,deinit,extension,
    subscript,prefix,operator,infix,postfix,precedence,associativity,left,right,none,convenience,dynamic,
    final,lazy,mutating,nonmutating,optional,override,required,static,unowned,safe,weak,internal,
    private,public,is,as,self,unsafe,dynamicType,true,false,nil,Type,Protocol,print
  },
  morecomment=[l]{//}, % l is for line comment
  morecomment=[s]{/*}{*/}, % s is for start and end delimiter
  morestring=[b]" % defines that strings are enclosed in double quotes
}

\definecolor{keyword}{HTML}{BA2CA3}
\definecolor{string}{HTML}{D12F1B}
\definecolor{comment}{HTML}{008400}
\definecolor{type}{HTML}{66B9AA}

\lstdefinestyle{Swift}{%
  language=swift,
  basicstyle=\ttfamily,
  showstringspaces=false, % lets spaces in strings appear as real spaces
  columns=fixed,
  keepspaces=true,
  keywordstyle=\color{keyword},
  stringstyle=\color{string},
  commentstyle=\color{comment},
  emph={Int,Character,Double,Float,Unsigned},
  emphstyle={\color{type}},
  morestring=[b]",
  escapeinside={(*}{*)}
}

\newcommand\syntaxbox[2][fill=orange!80]{%
    \tikz[baseline]\node[%
        inner ysep=0pt,
        inner xsep=2pt,
        anchor=text,
        rectangle,
        rounded corners=1mm,
        #1] {\strut#2};%
}



\begin{document}
\maketitle


\problem{}
\begin{enumerate}[label= (\alph*)]
    \item \num{4.54} $\pm$ \num{0.05} billion years (\url{https://en.wikipedia.org/wiki/Age_of_the_Earth})
    \item \num{5} billion years (\url{http://earthguide.ucsd.edu/virtualmuseum/ita/05_3.shtml})
    \item \num{13.6} billion years $\pm$ \num{800} million years (\url{https://www.space.com/263-milky-age-narrowed.html})
    \item \num{13.82} billion years (\url{http://www.slate.com/blogs/bad_astronomy/2013/03/21/age_of_the_universe_planck_results_show_universe_is_13_82_billion_years.html})
    \item \num{7.79} billion years (\url{https://www.livescience.com/39775-how-long-can-earth-support-life.html})
    \item \num{3.25} billion years. Guaranteed existential risks, human prevention not possible.
        \begin{itemize}
            \item Asteroid impact
            \item Extraterrestrial invasion (i.e.~, aliens taking over the world)
            \item Geomagnetic reversal of the North and South pole
        \end{itemize}

        Prevention possible.
        \begin{itemize}
            \item Artificial super intelligence enslavement of the man race. Steps are unclear, besides keeping all experiments in a Faraday cage.
            \item Biotechnology engineering. A bio-engineered virus or disease that gets loose can cause a global pandemic. Steps to prevent are clear.
            \item Nuclear holocaust. Steps to prevent are clear.
        \end{itemize}

        Extinction prevention matters are to become a space-ferrying species, attempt some sort of peace treaty amongst all nations, and to try to prevent global pandemics. (\url{https://en.wikipedia.org/wiki/Global_catastrophic_risk#Asteroid_impact})

    \item \num{10}-\num{12} billion years. The sun will become a Red Giant, no longer being able to provide fuel. Will consume several planets along the way. (\url{https://www.forbes.com/sites/startswithabang/2017/01/27/how-our-solar-system-will-end-in-the-far-future/#20ea714f4e5d})

            \begin{quote}
                Eventually, about 5--7 billion years down the line, we’ll run out of nuclear fuel in the Sun’s core, which will cause our parent star to become a Red Giant, engulfing Mercury and Venus in the process. Due to the particulars of stellar evolution, the Earth/Moon system will probably be pushed outwards, and be spared the fiery fate of our inner neighbors.
            \end{quote}

        \item Either the universe will stop contracting, reach an influx in the gravitational pull on all cosmic bodies, and come back to a singularity; or, heat death, where entropy reaches its max, and all energy is evenly dissipated throughout the universe, freezing everything. The first theory has \num{1}-\num{100} trillion years, the second has no estimate. (\url{http://www.bbc.com/earth/story/20150602-how-will-the-universe-end})

        \item \num{584.6} billion years. Between \num{0.585}\%--\num{58.5}\%.
\end{enumerate}

\problem{}
\begin{enumerate}[label= (\alph*)]
    \item A standard, 12pt \LaTeX{} document can fit roughly \num{39} lines of text.

        \begin{equation*}
            \frac{9 \times 10^8}{39} \approx 2.31 \times 10^8\,\text{sheets}
        \end{equation*}

    \item \SI{1500}{sheets} is approximately \$14.99, before tax. (\url{https://www.amazon.com/Georgia-Pacific-Spectrum-Standard-Multipurpose-998606/dp/B00BB5DJU6/ref=sr_1_2?s=office-products&ie=UTF8&qid=1503934559&sr=1-2&keywords=printer+paper})
        \begin{equation*}
            \left(2.31 \times 10^8\,\text{sheets}\right) \times \frac{\$14.99}{\SI{1500}{sheets}} \approx \$2.31 \times 10^6
        \end{equation*}

        Assuming a standard filing cabinet can store roughly \SI{15}{reams}, with \SI{100}{sheets} in a ream,

        \begin{equation*}
            2.31 \times 10^8\,\text{sheets} \times \frac{\SI{1}{reams}}{\SI{100}{sheets}} \times \frac{\SI{1}{cabinet}}{\SI{15}{reams}} = \num{154000}
        \end{equation*}

    \item The monks expected the project to take \textit{15,000 years} by hand, and \textit{3 months} by modern technology. Assuming an average human lifespan of \SI{79}{years}, approximately \SI{50}{years} could be used writing.

        \begin{equation*}
            \frac{\SI{15000}{years}}{\SI{50}{years}} = \SI{300}{people}
        \end{equation*}

    \item The time allocated to the computer was \textit{3 months}. To print all 3 billion, it would have to print approximately \SI{1120.07}{names} per second.

    \item I enjoyed the story, and the first thing I did was a mental check to see if the numbers added up. Naturally, they did not, but that did not detract from the story.
\end{enumerate}

\problem{}


\begin{lstlisting}
def alternating_difference(numbers):
    if not numbers:
        return 0

    return numbers[0] - alternating_difference(numbers[1:])
\end{lstlisting}

\problem{}
\begin{lstlisting}
def f53q(n):
    if  n < 8:
        return ValueError('Value less than 8')
    elif n % 5 == 0:
        return (0, n // 5)
    elif n % 5 == 1:
        return (2, (n - 5) // 5)
    elif n % 5 == 2:
        return (4, (n - 12) // 5)
    elif n % 5 == 3:
        return (1, n // 5)
    else:
        return (3, (n - 9) // 5)
\end{lstlisting}

\problem{}
\begin{align*}
    f(1) &= 91 \\
    f(-6) &= 91 \\
    f(200) &= 190 \\
    f(27) &= 91 \\
\end{align*}

A more appropriate, non-recursive function would be as follows:

\begin{lstlisting}
def f(n):
    if n < 101:
        return 91
    else:
        return n - 10
\end{lstlisting}

This will always produce the same output. Let us write $f(n)$ as a piecewise-defined function.

\[ f(x) =
    \begin{cases}
        n - 10 & 100 < n < \infty \\
        f(f(n + 11)) & -\infty < n \leq 100
    \end{cases}
\]

We see that, for values less than $100$, $n$ will grow until it hits the first case. For values $90\ldots200$, we see that $n$ will oscillate until it reaches $101$, upon which one final $f(x)$ will be applied.

A more simple explanation would be to insert $n - 10$ into $f(f(n + 11))$ to see that $f(x) = f(f(n+1))$ until it reaches $101$, $1$ past the bounds, and gets $n - 10$ applied to it one more time.

\problem{}
The function is copy and pasted from implementation.

\begin{lstlisting}
def A(x, y):
    if x == 0:
        return y + 1
    elif y == 0:
        return A(x - 1, 1)
    else:
        return A(x - 1, A(x, y - 1))
\end{lstlisting}

The maximum value of my system (Late-2013 MacBook Pro) is roughly \verb|A(3, 5)|. For \verb|A(2, 2)|, there should be \num{42438} function calls.

\problem{}
\begin{lstlisting}
def gcd2(a, b):
    if a == 0:
        return (b, 0, 1)
    else:
        g, s, t = gcd2(b % a, a)
        return [g, t - (b // a) * s, s]
\end{lstlisting}

\problem{}
\begin{lstlisting}
# defined as list_ to not stomp out the typename _list_
def super_reverse(list_):
    if len(list_) < 2:
        return list_
    else:
        last_element = list_[-1]
        first_element = list_[0]

        # if is a list or a tuple, sort that as well
        if isinstance(first_element, list):
            first_element = super_reverse(first_element)
        if isinstance(last_element, list):
            last_element = super_reverse(last_element)

        return [last_element] + super_reverse(list_[1:-1]) + [first_element]
\end{lstlisting}

\problem{}
\begin{lstlisting}
def anagram(string_):
    if string_ == "":
        return [""]

    result = []

    for partial_anagram in anagram(string_[1:]):
        for position in range(len(partial_anagram) + 1) :
            right_side = partial_anagram[position:]
            left_side = partial_anagram[:position]
            original_element = string_[0]

            result += [left_side + original_element + right_side]

    return result
\end{lstlisting}

\begin{table}[!ht]
    \centering
    \caption{Results (units in milliseconds)}
    \resizebox{\columnwidth}{!}{%
    \begin{tabular}{crrrc}
        \textbf{String Size} & \textbf{One Recursive Call} & \textbf{$n$ Recursive Calls} & \textbf{Absolute Difference} & \textbf{Percent Difference} \\
        1 &      0.006 &           0.004 &          -0.002 &         -56.250\% \\
        2 &      0.009 &           0.009 &          -0.000 &          -2.703\% \\
        3 &      0.015 &           0.026 &           0.011 &          42.202\% \\
        4 &      0.036 &           0.104 &           0.068 &          65.367\% \\
        5 &      0.139 &           0.528 &           0.389 &          73.668\% \\
        6 &      0.672 &           3.124 &           2.452 &          78.486\% \\
        7 &      5.312 &          25.050 &          19.738 &          78.794\% \\
        8 &     39.671 &         204.850 &         165.179 &          80.634\% \\
        9 &    332.021 &        1936.445 &        1604.424 &          82.854\% \\
        10&   3471.336 &       21813.828 &       18342.492 &          84.087\% \\
        11&  43223.737 &      267635.574 &      224411.837 &          83.850\%
    \end{tabular}
    }
\end{table}

\end{document}

