%
%  homework-3.tex
%  cs5200-analysis-of-algorithms
%
%  Created by Illya Starikov on 08/25/17.
%  Copyright 2017. Illya Starikov. All rights reserved.
%


\RequirePackage[l2tabu, orthodox]{nag}
\documentclass[12pt]{scrartcl}


\newcommand{\homeworknumber}{3}
\newcommand{\homeworkdue}{September 19\textsuperscript{th}, 2017}
\usepackage{amssymb,amsmath,verbatim,graphicx,microtype,upquote,units,booktabs,akkwidepage}

\newcommand{\chapterNumber}[1]{
    \setcounter{section}{#1}
    \addtocounter{section}{-1}
}

\begin{document}
\maketitle

\problem{}
\begin{lstlisting}
def fn(n):
    if n > 100:
        return n - 10
    return ((n - 101) % (c - 10)) + 91
\end{lstlisting}


\problem{}
The following function tests random values up to in $\{ \num{-1000000}, \ldots, \num{1000000} \}$. \texttt{is\_gcd\_fast} is a faster alternative because $\gcd \in \mathcal{O}\left( \log_{\frac{a}{a\mod b}} (a \times b) \right)$ while $\gcd_{\text{fast}} \in \mathcal{O}\left( c \right)$.

\begin{lstlisting}
from fractions import gcd
import random


def is_gcd(gcd_value, a, b):
    return gcd_value == gcd(a, b)


def is_gcd_fast(gcd_value, a, b):
    return a % gcd_value == b % gcd_value == 0

for i in range(1000):
    a = random.randint(-1000000, 1000000)
    b = random.randint(-1000000, 1000000)
    g = gcd(a, b) if random.choice([True, False]) else random.randint(-1000000, 1000000)

    print(is_gcd_fast(g, a, b) == is_gcd(g, a, b))
\end{lstlisting}


\problem{}
\subsection{All \texttt{Sum}s}
\begin{lstlisting}
# Warning, for all lists (with the exclusion of [1], [0] and []) this function does not stop
def generate_all_sums(l):
    for element in l:
        if element == 0:
            return [0] + generate_all_sums(element[1:])
        else:
            return [element] + [generate_all_sums([element*element])]

    return l
\end{lstlisting}

\subsection{$Sum(T) \not\in \{\text{\textit{all odds}}\}$}
Take $Sum$ to be the function that maps all elements to all of their multiples.

\begin{theorem}
    There is not set $T$ such that $Sum(T) = \{ \text{all odd integers} \}$.
\end{theorem}

We prove so by proof of contradiction.

\begin{proof}
    Suppose not. That is, suppose that is a set $T$ such that $Sum(T) = \{ \text{all odd integers} \}$. Set $T$, at a minimum, must have a single element (for the empty set $\emptyset$ cannot produce any integers). We take an arbitrary element from this set $T$ and name it $q$. There are two forms for $q$ ($\exists m \in \mathbb{Z}$),

\[
    q = \left.
        \begin{cases}
            2m & \Leftrightarrow q\ \text{is even} \\
            2m + 1 & \Leftrightarrow q\ \text{is odd} \\
        \end{cases}
        \right.
\]

If $q$ is even, there is already a contradiction. If $q$ is odd, then we have the following contradiction. Supposing we sum $q$ twice, we have the following form:

\begin{equation*}
    q = a(2m + 1) + b(2m + 1)
\end{equation*}

Because $a$ and $b$ are arbitrary, we take them to be $1$.

\begin{align*}
    q &= (2m + 1) + (2m + 1) \\
      &= 4m + 2 \\
      &= 2(2m + 1)
\end{align*}

Because we know the sum of integers and multiplication of integers to be integers, $2(2m + 1) \equiv 2n, \exists n \in \mathbb{Z}$, which has the form of an even integer.

This has lead us to a contradiction. Therefore, our hypothesis is false, concluding there is not set $T$ such that $Sum(T) = \{ \text{all odd integers} \}$.

\end{proof}

\subsection{Fast Algorithm For $S$}
To get $d$ from an arbitrary set $S$, we simply take the greatest common divisor from all the sets.

\begin{lstlisting}
from functools import reduce

def gcd(numbers):
    def gcd_single(a, b):
        while b:
            a, b = b, a % b
        return a

    return reduce(gcd_single, numbers)
\end{lstlisting}

\subsection{$d \in \{ 540051690381, 5404079462298, 3485942644184 \}$}
$547$.


\problem{}
\begin{theorem}
    $\forall i \in \mathbb{Z}^+$,

    \begin{equation*}\label{eq:1}
        \sum _{i = 1}^n {(-1)}^{i - 1} i^3 = -\frac{1}{8} {(-1)}^n \left( 4n^3 + 6n^2 - 1 \right) - \frac{1}{8}
    \end{equation*}
\end{theorem}

\begin{proof}
    \begin{description}
        \item[Step \#1] We wish to prove that for all natural numbers $f(n) = g(n)$, where $f$ and $g$ are as defined as follows:

            \begin{align*}
                f(x) &=  \sum _{i = 1}^n  {(-1)}^{i - 1} i^3 \\
                g(x) &= -\frac{1}{8} {(-1)}^n \left( 4n^3 + 6n^2 - 1 \right) - \frac{1}{8}
            \end{align*}

            Let $D$ be the set of natural numbers (i.e., $D = \mathbb{Z}^+$). $D$ includes the stopping value $1$.

        \item[Step \#2] Checking two values is trivial; take $2$ and $3$.

            \begin{align*}
                \sum _{i = 1}^2 {(-1)}^{i - 1} i^3 = -\frac{1}{8} {(-1)}^2 \left( 4 * 2^3 + 6 * ^2 - 1 \right) - \frac{1}{8} = -7 \\
                \sum _{i = 1}^3 {(-1)}^{i - 1} i^3 = -\frac{1}{8} {(-1)}^3 \left( 4 * 3^3 + 6 * 3^2 - 1 \right) - \frac{1}{8} = 20
            \end{align*}

            To check the stopping value, we check the first value. We clearly see that $f(x) = g(x) = 1$.

        \item[Step \#3] If $n$ in $\mathbb{Z}^+$ triggers a recursive call, then $n > 0$. The only value used in the call is $n - 1$, which is in $\mathbb{Z}$ and greater than or equal to $0$, because it is an integer and $n − 1 \geq 0$ since $n > 0$.

        \item[Step \#4] We use the integer $n$ as the counter. When recursion is called the function is called with the value $n - 1$. The counter strictly decreases and the recursion halts.

        \item[Step \#5] To prove recursion stops, we take the following:

            \begin{align}
                f(x) &= f(n - 1) + f(n) \\
                     &= f(n - 1) + {(-1)}^{n-1} n^3 \\
                     &= -\frac{1}{8} {(-1)}^{(n - 1)} \left( 4{(n - 1)}^3 + 6{(n - 1)}^2 - 1 \right) - \frac{1}{8} + {(-1)}^{n-1} n^3 \\
                     &= -\frac{1}{8} {(-1)}^n \left( 4n^3 + 6n^2 - 1 \right) - \frac{1}{8} \\
                     &= g(x)
            \end{align}
            Because $f(n) = g(n)$, the property is inherited recursively.

        \item[Step \#6]  Since Steps 1--5 have been verified, it follows from the Principle of Recursion that $P$ holds for all values in $\mathbb{Z^+}$, i.e., $f(n) = g(n), \forall n \in \mathbb{Z}^+$
    \end{description}
\end{proof}

\begin{table}[H]
    \centering
    \caption{The results from the explicit form, the series form, and the difference for the first \num{40} values.}
    \begin{tabular}{rrc}
        \toprule
        \textbf{Sum Value}    & \textbf{Explicit Values} & \textbf{Difference}  \\\midrule
            1      & 1      & 0 \\
            -7     & -7     & 0 \\
            20     & 20     & 0 \\
            -44    & -44    & 0 \\
            81     & 81     & 0 \\
            -135   & -135   & 0 \\
            208    & 208    & 0 \\
            -304   & -304   & 0 \\
            425    & 425    & 0 \\
            -575   & -575   & 0 \\
            756    & 756    & 0 \\
            -972   & -972   & 0 \\
            1225   & 1225   & 0 \\
            -1519  & -1519  & 0 \\
            1856   & 1856   & 0 \\
            -2240  & -2240  & 0 \\
            2673   & 2673   & 0 \\
            -3159  & -3159  & 0 \\
            3700   & 3700   & 0 \\
            -4300  & -4300  & 0 \\
            4961   & 4961   & 0 \\
            -5687  & -5687  & 0 \\
            6480   & 6480   & 0 \\
            -7344  & -7344  & 0 \\
            8281   & 8281   & 0 \\
            -9295  & -9295  & 0 \\
            10388  & 10388  & 0 \\
            -11564 & -11564 & 0 \\
            12825  & 12825  & 0 \\
            -14175 & -14175 & 0 \\
            15616  & 15616  & 0 \\
            -17152 & -17152 & 0 \\
            18785  & 18785  & 0 \\
            -20519 & -20519 & 0 \\
            22356  & 22356  & 0 \\
            -24300 & -24300 & 0 \\
            26353  & 26353  & 0 \\
            -28519 & -28519 & 0 \\
            30800  & 30800  & 0 \\ \bottomrule
    \end{tabular}
\end{table}

\begin{lstlisting}
def sum_solution(n):
    if n <= 1:
        return 1
    else:
        return (-1)**(n - 1) * n**3 + sum_solution(n - 1)


def explicit_solution(n):
    return -int(
        (1.0 / 8.0) *
        ((-1)**n) *
        (4 * n**3 + 6 * n**2 - 1) +
        1.0 / 8.0)


def main():
    for n in range(1, 40):
        print(sum_solution(n), explicit_solution(n), sum_solution(n) - explicit_solution(n))


if __name__ == "__main__":
    main()
\end{lstlisting}

\end{document}
