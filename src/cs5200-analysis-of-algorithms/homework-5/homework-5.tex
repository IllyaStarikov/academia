%
%  homework-5.tex
%  cs5200-analysis-of-algorithms
%
%  Created by Illya Starikov on 08/25/17.
%  Copyright 2017. Illya Starikov. All rights reserved.
%

foobar

\RequirePackage[l2tabu, orthodox]{nag}
\documentclass[12pt]{scrartcl}

\newcommand{\homeworknumber}{5}
\newcommand{\homeworkdue}{October 4\textsuperscript{th}, 2017}
\usepackage{amssymb,amsmath,verbatim,graphicx,microtype,upquote,units,booktabs,akkwidepage}

\newcommand{\chapterNumber}[1]{
    \setcounter{section}{#1}
    \addtocounter{section}{-1}
}

\begin{document}
\maketitle

\problem{}
\begin{theorem}
    If $L \geq 2$, then every binary tree with $L$ leaves contains a subtree having between $\frac{L}{3}$ and $\frac{2L}{3}$ leaves, inclusive.
\end{theorem}

\begin{proof}
        Suppose not. That is, suppose that there exists a binary tree with more than $1$ leaves with all subtrees containing leaves in the range $\frac{L}{3} > n > \frac{2L}{3}$. The root contains $L$ leaves, and all bottommost leaves contain $1$ leaves.

        This implies that two subtrees with size less than $\frac{L}{3}$ (because our hypothesis states all leaves must in the range $\frac{L}{3} > n > \frac{2L}{3}$) combined to make a tree of size greater than $\frac{2L}{3}$. This is a contradiction

        \begin{equation*}
            \frac{L}{3} + \frac{L}{3} \not > \frac{2L}{3}
        \end{equation*}

        Because we have reached a contradiction, our hypothesis does not hold. Therefore, binary tree with $L$ leaves contains a subtree having between $\frac{L}{3}$ and $\frac{2L}{3}$ leaves, inclusive.
\end{proof}


\problem{}
\begin{theorem}
    For every binary tree $T$, let us associate a ``weight'' $w(q) = 2^{-depth(q)}$. For all leaves $q \in T$,

    \begin{equation*}
        \sum _q w(q) \leq 1
    \end{equation*}
\end{theorem}

\begin{proof}
    We prove so by induction.

    (Base Case) For the root,

    \begin{equation*}
        \sum _{x \in T} w(x) = 2^{-0} = 1
    \end{equation*}

    (Inductive Hypothesis) Suppose for any tree of $n$ nodes, $\sum _{x \in T} ^n w(q) \leq 1$.

    (Inductive Step) Suppose the binary tree $T_0$ to have $n - 1$ nodes. We must show that $\sum _{x \in T_0} ^{n-1} w(x) \leq 1$. Suppose we take the left subtree $T_L$ or the right subtree $T_R$. By the inductive hypothesis, the sum of the weights are less than or equal to $1$; however, since the depth of a node in $T_0$ is one greater than the depth of $T_L$ or $T_R$, the respective weights of ever leaf in $T_0$ is halved. Thus,

    \begin{equation*}
        \sum _{x \in T_L} w(x) \leq \nicefrac{1}{2} \qquad \sum _{x \in T_R} w(x) \leq \nicefrac{1}{2}
    \end{equation*}

    Because both of these are bounded by $\nicefrac{1}{2}$, their sum forms $T_0$, which is bounded by $1$.

    By the principle of mathematical induction, $\sum _q w(q) \leq 1$ for any tree.
\end{proof}


\problem{}
\begin{theorem}
    Any planar graph can be colored with six or fewer colors.
\end{theorem}

\begin{proof}
    We prove so by induction. We take $n$ to be an arbitrary number of nodes in an arbitrary graph $G$.

    (Base Case) For $n$ vertices, where $1 \leq n \leq 6$, we can color the graph $G$ with $6$ or less colors (for every vertices maps to a single, unique color).

    (Inductive Hypothesis) Suppose for $n - 1$ vertices, where $n > 1$, the graph $G$ can be colored with $6$ or less colors.

    (Inductive Step) We now prove that $G$ can be colored with $n$ vertices can be colored with $6$ or fewer colors. Recall all connected, simple planar graph contains a maximum of $5$ degrees. Suppose that the any vertex $v \in G$ has this max degree $5$.

    We remove this vertex, name it $v_0$, with all incident edges. We now have less than $n$ vertices, and by our inductive hypothesis, this graph now can be colored with $6$ or less colors. Adding this vertex back, we see that there are $5$ incident vertexes, hence $5$ colors. We use the $6$\textsuperscript{th} color for $v_0$.

    Hence, the inductive step holds. By the principle of mathematical induction, any planar graph can be colored with six or fewer colors.
\end{proof}


\problem{}
\begin{theorem}
    Any point $x$ in a graph is a cut-point iff there exists two vertices in the graph, $a$ and $b$, such that every path between $a$ and $b$ has to pass through $x$.
\end{theorem}

\begin{proof}
    Assume $G$ to be a connected graph, and $\exists \text{ vertices } a, b \in G$, where every path $a\longleftrightarrow b$ path passes through $x$. Also assume there exists no path $a\longleftrightarrow b \in G - \{ x \}$. Therefore, $G - \{ x \}$ is disconnected, for $a$ and $b$ are in two different components of $G - \{ x \}$. This proves that $x$ is a cut-point.
\end{proof}


\problem{}
\begin{description}
    \item[Euler Path] A path that uses every edge of a graph exactly once.
    \item[Hamiltonian Cycle] A path through a graph that starts and ends at the same vertex and includes every other vertex exactly once.
\end{description}

Yes, all \textbf{cyclic graphs} meet both these conditions.


\end{document}
