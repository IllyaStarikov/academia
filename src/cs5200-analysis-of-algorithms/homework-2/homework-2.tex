%
%  homework-2.tex
%  cs5200-analysis-of-algorithms
%
%  Created by Illya Starikov on 08/25/17.
%  Copyright 2017. Illya Starikov. All rights reserved.
%

\RequirePackage[l2tabu, orthodox]{nag}
\documentclass[12pt]{scrartcl}


\newcommand{\homeworknumber}{2}
\newcommand{\homeworkdue}{September 11\textsuperscript{th}, 2017}
\usetheme[logo=template-presentation/logo,faculty=ped]{fibeamer}

\usepackage{amssymb,amsmath,verbatim,graphicx,pdfpages,microtype,units,booktabs,upquote,xcolor,siunitx,csquotes,fancyvrb,newverbs,wrapfig,multicol,tikz,textcomp,wrapfig,cutwin}
\usepackage{fontawesome,setspace,rotchiffre,lipsum,listings,animate,listings}
\usepackage[xspace]{ellipsis}

\hypersetup{%
            colorlinks = true,
            linkcolor = orange,
            urlcolor  = orange,
            citecolor = orange,
            anchorcolor = orange}

\newcommand{\hugeslide}[1]{%
\begin{frame}[plain,c]
    \centering {\usebeamerfont*{frametitle} \usebeamercolor[fg]{frametitle}{\fontsize{40}{50}\selectfont\textit{#1}}}
\end{frame}
}

\newcommand{\presentaddcount}[1]{\addtocounter{#1}{1}\Roman{#1}}
\newcommand{\presentcount}[1]{\Roman{#1}}
\newcommand{\shellcmd}[1]{\texttt{\colorbox{gray!30}{#1}}}

\lstdefinelanguage{swift}
{%
  morekeywords={%
    func,if,then,else,for,in,while,do,switch,case,default,where,break,continue,fallthrough,return,
    typealias,struct,class,enum,protocol,var,func,let,get,set,willSet,didSet,inout,init,deinit,extension,
    subscript,prefix,operator,infix,postfix,precedence,associativity,left,right,none,convenience,dynamic,
    final,lazy,mutating,nonmutating,optional,override,required,static,unowned,safe,weak,internal,
    private,public,is,as,self,unsafe,dynamicType,true,false,nil,Type,Protocol,print
  },
  morecomment=[l]{//}, % l is for line comment
  morecomment=[s]{/*}{*/}, % s is for start and end delimiter
  morestring=[b]" % defines that strings are enclosed in double quotes
}

\definecolor{keyword}{HTML}{BA2CA3}
\definecolor{string}{HTML}{D12F1B}
\definecolor{comment}{HTML}{008400}
\definecolor{type}{HTML}{66B9AA}

\lstdefinestyle{Swift}{%
  language=swift,
  basicstyle=\ttfamily,
  showstringspaces=false, % lets spaces in strings appear as real spaces
  columns=fixed,
  keepspaces=true,
  keywordstyle=\color{keyword},
  stringstyle=\color{string},
  commentstyle=\color{comment},
  emph={Int,Character,Double,Float,Unsigned},
  emphstyle={\color{type}},
  morestring=[b]",
  escapeinside={(*}{*)}
}

\newcommand\syntaxbox[2][fill=orange!80]{%
    \tikz[baseline]\node[%
        inner ysep=0pt,
        inner xsep=2pt,
        anchor=text,
        rectangle,
        rounded corners=1mm,
        #1] {\strut#2};%
}



\begin{document}
\maketitle

\problem{}
We define $Q = \{q | q + n = n + k \}$, and $n \in Z^+$. Given that $n + 0 = n = 0 + n$

\begin{align*}
    q + (k + 1) &= (q + k) + 1 \\
                &= (k + q) + 1 \\
                &= k + (1 + k) \\
                &= (k + 1) + m \\
                &\therefore q \in Q \\
                &\therefore n + q = q
\end{align*}

\problem{}
\begin{theorem}
    $\forall i \in \mathbb{Z}^+$,

    \begin{equation}\label{eq:1}
        \sum_{i = 1} ^{n} \frac{i^2}{(2i -1)(2i + 1)} = \frac{n(n + 1)}{2(2n + 1)}
    \end{equation}
\end{theorem}

\begin{proof}
    We will use the principle of recursion to solve this problem.

    \begin{description}
        \item[Step \#1] Our problem states for an arbitrary integer $i$, summed to an arbitrary integer $n$, we have the summation (we denote by $f(x)$) and explicit forms ($g(x)$) described by Equation~\ref{eq:1}. We take the stopping value to be $0$.
        \item[Step \#2] Checking two values is trivial; take $1$ and $2$.

            \begin{align*}
                \sum_{i = 1} ^{1} \frac{i^2}{(2i -1)(2i + 1)} &= \frac{1(1 + 1)}{2(2*1 + 1)} = \frac{1}{3} \\
                \sum_{i = 1} ^{n} \frac{i^2}{(2i -1)(2i + 1)} &= \frac{2*(2 + 1)}{2(2*2 + 1)} = \frac{3}{5}
            \end{align*}

            To check the stopping value, we check the $0$th value. We clearly see that $f(x) = g(x) = 0$.

        \item[Step \#3] If $n$ in $\mathbb{Z}^+$ triggers a recursive call, then $n > 0$. The only value used in the call is $n - 1$, which is in $\mathbb{Z}$ and greater than or equal to $0$, because it is an integer and $n − 1 \geq 0$ since $n > 0$.

        \item[Step \#4] We use the integer $n$ as the counter. When recursion is called the function is called with the value $n - 1$. The counter strictly decreases and the recursion halts.

        \item[Step \#5] To prove recursion stops, we take the following:

            \begin{alignat*}{2}
                f(n) &= \sum_{i = 1} ^{n} \frac{i^2}{(2i -1)(2i + 1)} &\text{definition}\\
                     &= f(n-1) + \frac{n^2}{(2n -1)(2n + 1)}          &\text{recursive structure}\\
                     &= g(n-1) + \frac{n^2}{(2n -1)(2n + 1)}          &\text{by our assumption}\\
                     &= \frac{(n - 1)((n-1) + 1)}{2(2(n - 1) + 1)} + \frac{n^2}{(2n -1)(2n + 1)} & \text{replacing $g(n)$ with definition} \\
                     &= \frac{n(n + 1)}{2(2(2n + 1))}                 &\text{algebraic simplification} \\
                     &= g(n)
            \end{alignat*}

            Because $f(n) = g(n)$, the property is inherited recursively.

        \item[Step \#6]  Since Steps 1--5 have been verified, it follows from the Principle of Recursion that $P$ holds for all values in $\mathbb{Z^+}$, i.e., $f(n) = g(n), \forall n \in \mathbb{Z}, n \geq 0$
    \end{description}
\end{proof}

\begin{table}[H]
    \centering
    \caption{The results from the explicit form, the series form, and the difference for the first \num{40} values.}
    \begin{tabular}{ccc}
        \textbf{Sum Value}    & \textbf{Explicit Values} & \textbf{Difference}  \\
        \num{0.333333333333} &  \num{0.333333333333} & \num{0.000000000000000} \\
        \num{0.600000000000} &  \num{0.600000000000} & \num{0.000000000000000} \\
        \num{0.857142857143} &  \num{0.857142857143} & \num{0.000000000000000} \\
        \num{1.111111111111} &  \num{1.111111111111} & \num{0.000000000000000} \\
        \num{1.363636363636} &  \num{1.363636363636} & \num{0.000000000000000} \\
        \num{1.615384615385} &  \num{1.615384615385} & \num{0.000000000000000} \\
        \num{1.866666666667} &  \num{1.866666666667} & \num{0.000000000000000} \\
        \num{2.117647058824} &  \num{2.117647058824} & \num{0.000000000000000} \\
        \num{2.368421052632} &  \num{2.368421052632} & \num{0.000000000000000} \\
        \num{2.619047619048} &  \num{2.619047619048} & \num{0.000000000000000} \\
        \num{2.869565217391} &  \num{2.869565217391} & \num{0.000000000000000} \\
        \num{3.120000000000} &  \num{3.120000000000} & \num{0.000000000000001} \\
        \num{3.370370370370} &  \num{3.370370370370} & \num{0.000000000000001} \\
        \num{3.620689655172} &  \num{3.620689655172} & \num{0.000000000000001} \\
        \num{3.870967741935} &  \num{3.870967741935} & \num{0.000000000000001} \\
        \num{4.121212121212} &  \num{4.121212121212} & \num{0.000000000000001} \\
        \num{4.371428571429} &  \num{4.371428571429} & \num{0.000000000000001} \\
        \num{4.621621621622} &  \num{4.621621621622} & \num{0.000000000000002} \\
        \num{4.871794871795} &  \num{4.871794871795} & \num{0.000000000000001} \\
        \num{5.121951219512} &  \num{5.121951219512} & \num{0.000000000000002} \\
        \num{5.372093023256} &  \num{5.372093023256} & \num{0.000000000000002} \\
        \num{5.622222222222} &  \num{5.622222222222} & \num{0.000000000000002} \\
        \num{5.872340425532} &  \num{5.872340425532} & \num{0.000000000000003} \\
        \num{6.122448979592} &  \num{6.122448979592} & \num{0.000000000000003} \\
        \num{6.372549019608} &  \num{6.372549019608} & \num{0.000000000000003} \\
        \num{6.622641509434} &  \num{6.622641509434} & \num{0.000000000000004} \\
        \num{6.872727272727} &  \num{6.872727272727} & \num{0.000000000000004} \\
        \num{7.122807017544} &  \num{7.122807017544} & \num{0.000000000000004} \\
        \num{7.372881355932} &  \num{7.372881355932} & \num{0.000000000000004} \\
        \num{7.622950819672} &  \num{7.622950819672} & \num{0.000000000000004} \\
        \num{7.873015873016} &  \num{7.873015873016} & \num{0.000000000000003} \\
        \num{8.123076923077} &  \num{8.123076923077} & \num{0.000000000000004} \\
        \num{8.373134328358} &  \num{8.373134328358} & \num{0.000000000000004} \\
        \num{8.623188405797} &  \num{8.623188405797} & \num{0.000000000000004} \\
        \num{8.873239436620} &  \num{8.873239436620} & \num{0.000000000000004} \\
        \num{9.123287671233} &  \num{9.123287671233} & \num{0.000000000000004} \\
        \num{9.373333333333} &  \num{9.373333333333} & \num{0.000000000000004} \\
        \num{9.623376623377} &  \num{9.623376623377} & \num{0.000000000000004} \\
        \num{9.873417721519} &  \num{9.873417721519} & \num{0.000000000000004} \\
    \end{tabular}
\end{table}

\problem{}
\begin{theorem}
    For an arbitrary $n \in Z$, such that $n \geq 0$,

    \begin{equation*}
        2^n \leq fib(n) \leq 2^\frac{n}{2}
    \end{equation*}
\end{theorem}

\begin{proof}
    \begin{description}
        \item[Step \#1] Prove by induction that for $n > 0$

            \begin{equation*}
                2^n \leq fib(n) \leq 2^\frac{n}{2}
            \end{equation*}


        \item[Step \#2] Check the base case and two other values,

           \begin{align*}
               2^0 \leq fib(0) \leq \leq 2^0
               2^1 \leq fib(1) \leq \leq 2^\frac{1}{2}
               2^2 \leq fib(2) \leq \leq 2^1
           \end{align*}

       \item[Step \#3] To prove the recursive case, we take the following.

           \begin{alignat*}{2}
                2^n \leq fib(n) \leq 2^\frac{n}{2} \\
                2^{n - 1} \leq fib(n) + fib(n-1) \leq 2^\frac{n - 1}{2} \\
                \frac{2^n}{2} \leq fib(n) + fib(n-1) \leq \frac{2^\frac{n}{2}}{\sqrt{2}} \\
           \end{alignat*}

           Because we know $ \frac{2^n}{2} \leq 2^n$ and $\frac{2^{\frac{n}{2}}}{\sqrt{2}} \leq 2^\frac{n}{2}$, we know that $f(n)$ must be bounded by those function.

       \item[Step \#4]  Since the hypotheses of Simple Induction are true, the conclusion follows, namely, for all natural numbers, $f(n) \mod 133 = 0$.
    \end{description}
\end{proof}

\problem{}
Upon inspection, it's quite apparent that the series is the sum of all natural numbers, with every other number being negative. More formally,

\begin{equation*}
    \sum_{q=1}^n {(-1)}^{q - 1} q^2 = {(-1)}^{n - 1} \frac{n*(n + 1)}{2}
\end{equation*}

The results can be summarized below.

\begin{table}[H]
    \centering
    \caption{The results from the explicit form, the series form, and the difference for the first \num{40} values.}
    \begin{tabular}{ccc}
        \textbf{Sum Value}    & \textbf{Explicit Values} & \textbf{Sum of Numbers}  \\
        1.0000000000    & 1.0000000000    & 1.000000000000000 \\
        -3.0000000000   & -3.0000000000   & 3.000000000000000 \\
        6.0000000000    & 6.0000000000    & 6.000000000000000 \\
        -10.0000000000  & -10.0000000000  & 10.000000000000000 \\
        15.0000000000   & 15.0000000000   & 15.000000000000000 \\
        -21.0000000000  & -21.0000000000  & 21.000000000000000 \\
        28.0000000000   & 28.0000000000   & 28.000000000000000 \\
        -36.0000000000  & -36.0000000000  & 36.000000000000000 \\
        45.0000000000   & 45.0000000000   & 45.000000000000000 \\
        -55.0000000000  & -55.0000000000  & 55.000000000000000 \\
        66.0000000000   & 66.0000000000   & 66.000000000000000 \\
        -78.0000000000  & -78.0000000000  & 78.000000000000000 \\
        91.0000000000   & 91.0000000000   & 91.000000000000000 \\
        -105.0000000000 & -105.0000000000 & 105.000000000000000 \\
        120.0000000000  & 120.0000000000  & 120.000000000000000 \\
        -136.0000000000 & -136.0000000000 & 136.000000000000000 \\
        153.0000000000  & 153.0000000000  & 153.000000000000000 \\
        -171.0000000000 & -171.0000000000 & 171.000000000000000 \\
        190.0000000000  & 190.0000000000  & 190.000000000000000 \\
        -210.0000000000 & -210.0000000000 & 210.000000000000000 \\
        231.0000000000  & 231.0000000000  & 231.000000000000000 \\
        -253.0000000000 & -253.0000000000 & 253.000000000000000 \\
        276.0000000000  & 276.0000000000  & 276.000000000000000 \\
        -300.0000000000 & -300.0000000000 & 300.000000000000000 \\
        325.0000000000  & 325.0000000000  & 325.000000000000000 \\
        -351.0000000000 & -351.0000000000 & 351.000000000000000 \\
        378.0000000000  & 378.0000000000  & 378.000000000000000 \\
        -406.0000000000 & -406.0000000000 & 406.000000000000000 \\
        435.0000000000  & 435.0000000000  & 435.000000000000000 \\
        -465.0000000000 & -465.0000000000 & 465.000000000000000 \\
        496.0000000000  & 496.0000000000  & 496.000000000000000 \\
        -528.0000000000 & -528.0000000000 & 528.000000000000000 \\
        561.0000000000  & 561.0000000000  & 561.000000000000000 \\
        -595.0000000000 & -595.0000000000 & 595.000000000000000 \\
        630.0000000000  & 630.0000000000  & 630.000000000000000 \\
        -666.0000000000 & -666.0000000000 & 666.000000000000000 \\
        703.0000000000  & 703.0000000000  & 703.000000000000000 \\
        -741.0000000000 & -741.0000000000 & 741.000000000000000 \\
        780.0000000000  & 780.0000000000  & 780.000000000000000 \\
    \end{tabular}
\end{table}

\problem{}
\problem{}

\begin{lstlisting}
def depth(n):
    if n < 2:
        return 1
    if n % 2 == 1:
        return 1 + depth(3 * n + 1)
    else:
        return 1 + depth(n // 2)


def main():
    for i in range(101):
        print("%3.0f & %5.0f\\\\" % (i, depth(i)))


if __name__ == "__main__":
    main()
\end{lstlisting}


\begin{minipage}[t]{.30\textwidth}
    \centering
    \begin{tabular}{cl}
        \textbf{Iteration} & \textbf{Depth} \\
        0 &     1\\
        1 &     1\\
        2 &     2\\
        3 &     8\\
        4 &     3\\
        5 &     6\\
        6 &     9\\
        7 &    17\\
        8 &     4\\
        9 &    20\\
        10 &     7\\
        11 &    15\\
        12 &    10\\
        13 &    10\\
        14 &    18\\
        15 &    18\\
        16 &     5\\
        17 &    13\\
        18 &    21\\
        19 &    21\\
        20 &     8\\
        21 &     8\\
        22 &    16\\
        23 &    16\\
        24 &    11\\
        25 &    24\\
        26 &    11\\
        27 &   112\\
        28 &    19\\
        29 &    19\\
        30 &    19\\
        31 &   107\\
        32 &     6\\
        33 &    27\\
    \end{tabular}
\end{minipage}\hfil
\begin{minipage}[t]{.30\textwidth}
    \begin{tabular}{cl}
        34 &    14\\
        35 &    14\\
        36 &    22\\
        37 &    22\\
        38 &    22\\
        39 &    35\\
        40 &     9\\
        41 &   110\\
        42 &     9\\
        43 &    30\\
        44 &    17\\
        45 &    17\\
        46 &    17\\
        47 &   105\\
        48 &    12\\
        49 &    25\\
        50 &    25\\
        51 &    25\\
        52 &    12\\
        53 &    12\\
        54 &   113\\
        55 &   113\\
        56 &    20\\
        57 &    33\\
        58 &    20\\
        59 &    33\\
        60 &    20\\
        61 &    20\\
        62 &   108\\
        63 &   108\\
        64 &     7\\
        65 &    28\\
        66 &    28\\
        67 &    28\\
    \end{tabular}
\end{minipage}\hfil
\begin{minipage}[t]{.30\textwidth}
    \begin{tabular}{cl}
        68 &    15\\
        69 &    15\\
        70 &    15\\
        71 &   103\\
        72 &    23\\
        73 &   116\\
        74 &    23\\
        75 &    15\\
        76 &    23\\
        77 &    23\\
        78 &    36\\
        79 &    36\\
        80 &    10\\
        81 &    23\\
        82 &   111\\
        83 &   111\\
        84 &    10\\
        85 &    10\\
        86 &    31\\
        87 &    31\\
        88 &    18\\
        89 &    31\\
        90 &    18\\
        91 &    93\\
        92 &    18\\
        93 &    18\\
        94 &   106\\
        95 &   106\\
        96 &    13\\
        97 &   119\\
        98 &    26\\
        99 &    26\\
        100 &    26\\
    \end{tabular}
\end{minipage}\hfil

\problem{}
\begin{theorem}
    For an arbitrary $n \in Z$, such that $n \geq 0$,

    \begin{equation*}
        \sum _{i = 1}^n = 1^2 + 2^2 + \cdots + n^2 = \frac{n(n + 1)(2n + 1)}{6}
    \end{equation*}
\end{theorem}

\begin{proof}
    \begin{description}
        \item[Step \#1] Prove by induction that for $n = 0$

            \begin{equation*}
                \sum _{i = 1}^n i^2 = 1^2 + 2^2 + \cdots + n^2 = \frac{n(n + 1)(2n + 1)}{6}
            \end{equation*}

            We choose the summation form as $f(x)$ and the explicit form $g(x)$.

        \item[Step \#2] Check the base case and two other values,

           \begin{align*}
               \sum _{i = 1}^1 n^2 = 1 &= \frac{1(1 + 1)(2*1 + 1)}{6} \\
               \sum _{i = 1}^2 n^2 = 5 &= \frac{1(1 + 1)(2*1 + 1)}{6} \\
               \sum _{i = 1}^3 n^2 = 14 &= \frac{2*(2+1)(2*2 + 1)}{6}
           \end{align*}

       \item[Step \#3] $\forall n > 0$, prove that $P(n)$ is true if $P(n - 1)$ is true,

           \begin{alignat*}{2}
               f(x) &= \sum _{i = 1}^n i^2      &\text{by definition} \\
                    &= f(x - 1) + n^2           &\text{by recursion} \\
                    &= g(x - 1) + n^2           &\text{by hypothesis} \\
                    &= \frac{(n-1)(n - 1 + 1)(2(n + 1) + 1)}{6} + n^2 \\
                    &= \frac{n(n + 1)(2n + 1)}{6} &\text{algebraic simplification}
           \end{alignat*}

       \item[Step \#4]  Since the hypotheses of Simple Induction are true, the conclusion follows,namely, for all natural numbers, $f(n) = g(n)$.
    \end{description}
\end{proof}


\problem{}
\begin{theorem}
    For an arbitrary $n \in Z$, such that $n \geq 0$,

    \begin{equation*}
        11^{n + 2} + 12^{2n + 1} \mod 133 = 0
    \end{equation*}
\end{theorem}

\begin{proof}
    \begin{description}
        \item[Step \#1] Prove by induction that for $n > 0$

            \begin{equation*}
                11^{n + 2} + 12^{2n + 1} \mod 133 = 0
            \end{equation*}


        \item[Step \#2] Check the base case and two other values,

           \begin{align*}
               11^3 + 12^3 \mod 133 = 0 \\
               11^4 + 12^5 \mod 133 = 0
           \end{align*}

       \item[Step \#3] $\forall n > 0$, prove that $P(n)$ is true if $P(n - 1)$ is true,

           \begin{alignat*}{2}
               f(x) &= 11^{n + 2} + 12^{2n + 1} \\
                    &= 11^{n + 1} + 12^{2n - 1} + 11^{n + 2} + 12^{2n + 1}
           \end{alignat*}

           Because we assume $11^{n + 2} + 12^{2n + 1}$ to be divisible, we can rewrite it to be $133a_1$, signifying it is evenly divisible by $133$.

           \begin{alignat*}{2}
               f(x) &= 11 * 11^n + 12^{-1} * 144^n + 133a_1 \\
                    &= a_2 11^n + a_3 144^n + 133a_1
           \end{alignat*}

            We know the superposition of all of these numbers to be divisible by $133$.

       \item[Step \#4]  Since the hypotheses of Simple Induction are true, the conclusion follows, namely, for all natural numbers, $f(n) \mod 133 = 0$.
    \end{description}
\end{proof}

\problem{}
Let $F(0)$ denote the root of the tree $T$. If we start at $F(0)$, there must be an adjacent vertex to it, namely, $v_1$. There are infinitely many vertexes by going through the tree (this must be the case, or the tree $T$ would be finite).

We can repeat this process, for at any vertex $v_n$, there exists a vertex $v_{n + 1}$. Because the tree is infinite, there are infinitely many $v_{n + 1}$.

By induction, for any $n$, there exists a path of length $L$ in $T$.

\problem{}
\begin{lstlisting}
class Binary_Search_Tree:
    def __init__(self, data):
        if data is None:
            self.data = [0, [], []]
        else:
            self.data = data

    def add_node(self, node):
        if node.data[0] < self.data[0]:
            self.__add_node_left(node)
        else:
            self.__add_node_right(node)

    def __add_node_left(self, node):
        if self.data[1] == []:
            self.data[1] = node
        else:
            self.data[1].add_node(node)

    def __add_node_right(self, node):
        if self.data[2] == []:
            self.data[2] = node
        else:
            self.data[2].add_node(node)

    def leaf_count(self):
        if self.data[1] == []:
            if self.data[2] == []:
                return 1
            else:
                return self.data[2].leaf_count()
        elif self.data[2] == []:
            return self.data[1].leaf_count()
        else:
            return self.data[1].leaf_count() + self.data[2].leaf_count()

    def internal_count(self):
        if self.data[1] == []:
            if self.data[2] == []:
                return 0
        elif self.data[2] == []:
            return 1 + self.data[1].internal_count()
        else:
            return 1 + self.data[1].internal_count() + self.data[2].internal_count()


def main():
    T = Binary_Search_Tree([
        42,
        Binary_Search_Tree([29, [], []]),
        Binary_Search_Tree([51, [], []])
    ])

    print("Leaf: {0}, Internal: {1}".format(T.internal_count(), T.leaf_count()))

    T.add_node(Binary_Search_Tree([25, [], []]))
    T.add_node(Binary_Search_Tree([22, [], []]))
    T.add_node(Binary_Search_Tree([45, [], []]))
    T.add_node(Binary_Search_Tree([8, [], []]))
    T.add_node(Binary_Search_Tree([1000, [], []]))
    T.add_node(Binary_Search_Tree([3, [], []]))
    T.add_node(Binary_Search_Tree([1, [], []]))

    print("Leaf: {0}, Internal: {1}".format(T.internal_count(), T.leaf_count()))


if __name__ == "__main__":
    main()
\end{lstlisting}

From the results, we can very much see that $\text{Internal Nodes} = \text{Leaf Nodes} - 1$.

\subsection{Proof}
\begin{proof}
    Take $n$ to be the number of nodes, $I(T)$ the number of internal nodes and $L(T)$ to be the number of leaf nodes. We know the internal nodes to be the sum of previous internal nodes (i.e., $I(t) = I(t_1) + I(t_2) + 1$).

    From this, we get

    \begin{align}
        I(T) &= I(t_1) + I(t_2) + 1 \\
        I(t_1) &= L(t_1) - 1 \\
        I(t_2) &= L(t_2) - 1
        I(T) &= L(t_1) - 1 + L(t_2) - 1 \\
             &= L(T) - 1
    \end{align}
\end{proof}
\end{document}
