\RequirePackage[l2tabu, orthodox]{nag}
\documentclass[12pt]{scrartcl}

\usepackage{amssymb,amsmath,verbatim,graphicx,microtype,upquote,units,booktabs,siunitx,listings,fancyvrb,newverbs,tcolorbox}
\usepackage{xcolor}

\newcommand{\shellcmd}[1]{\texttt{\colorbox{gray!30}{\strut#1}}}
\newtcbox{\syntax}{nobeforeafter,colframe=blue,colback=blue!10!white,boxrule=0.5pt,arc=4pt,
  boxsep=0pt,left=4pt,right=4pt,top=3pt,bottom=3pt,tcbox raise base}

\lstdefinestyle{cC++}{%
    language=C++,
    basicstyle=\footnotesize\ttfamily,
    keywordstyle=\color{blue}\ttfamily,
    stringstyle=\color{red}\ttfamily,
    commentstyle=\color{gray}\ttfamily,
    morecomment=[l][\color{magenta}]{\#},
    showstringspaces=false,
    numbers=left,
    keepspaces=true,
    tabsize=4,
    breaklines=true,
    morekeywords={string,NULL}
}

\lstset{%
escapeinside={!*}{*!},%
}

\title{Modern C++ (The Good Parts)}
\subtitle{CS1570 --- Introduction to Programming}
\date{\today}
\author{Illya Starikov}

\begin{document}
\maketitle

Since its creation, C++ has gone through many different ``versions'' --- more formally, known as standards. With \shellcmd{fg++}, you've been using C++03\footnote{Yes, that is ``hella'' old}. Since then, there have been two major standards released: C++11 and C++14.

C++11 introduced many modern programming paradigms from other programming languages, and C++14 built on these principles. Namely, some of these features are type inference, lambda expressions, constant expressions, \shellcmd{default} and \shellcmd{deleted} keywords, and much more!

To compile C++11 or C++14 code, the \shellcmd{-std=c++11} or \shellcmd{-std=c++14} flags must be added (respectively). So, to compile with C++14, the command would be

\begin{center}
    \shellcmd{g++ -std=c++14 *.cpp}
\end{center}

\section{Type Inference (\shellcmd{auto})}
Type inference is a way for a language to systematically ``choose'' the type you are trying to use. This can intuitively be thought of as you, the programmer, guessing the type of a math equation:

\begin{equation}
    \text{answer} = 42
\end{equation}

Mathematically, we know answer to be an integer --- subsequently, the compiler (GCC) should guess its type to be an \shellcmd{int}. Wouldn't it be great if C++ could just magically do that? Well, now it can.

\begin{equation}
    \text{\shellcmd{auto} answer} = 42\shellcmd{;}
\end{equation}

The actual syntax is

\begin{lstlisting}[backgroundcolor=\color{lightgray},frame=single,basicstyle=\linespread{2}\ttfamily,emph={auto,inference},emphstyle={\syntax}]
    auto variable = inference;
\end{lstlisting}

Now, we no longer have to type out those pesky return types (very useful if it's constantly changing). Let's do an example.

\begin{lstlisting}[style=cC++]
template <typename T>
T oneUp(const T value) {
	return value + 1;
}

int main(int argc, char *argv[]) {
	auto integer = oneUp(1);
	auto character = oneUp('P');
	auto string = oneUp("vs NP");
	auto boolean = oneUp(false);
}
\end{lstlisting}

Compare \shellcmd{main} to it's pre-C++11/14 version.

\begin{lstlisting}[style=cC++]
template <typename T>
T oneUp(const T value) {
	return value + 1;
}

int main(int argc, char *argv[]) {
	int integer = oneUp(1);
	char character = oneUp('P');
	string std_string = oneUp("vs NP");
	bool boolean = oneUp(false);
}
\end{lstlisting}

So there it is efficient for the programmer. How efficient is it for the program? Let's take another example.

\begin{lstlisting}[style=cC++]
	char charArray[42] = "Hello";
	auto length = strlen(charArray);
\end{lstlisting}

Can you guess the type of \shellcmd{length}? Spoiler alert, it's not \shellcmd{int}. It's actually \shellcmd{size\_t}, a different type leftover from the C era. It's not vital knowing what \shellcmd{size\_t} is or how it works, it's just vital knowing \textit{it is not an }\shellcmd{int}. So, every time \shellcmd{strlen} it is used as an \shellcmd{int}, i.e.,

\begin{lstlisting}[style=cC++]
    for (int i = 0; i < strlen(charArray); i++) {
        /* do something here */
    }
\end{lstlisting}

this has a performance cost. Why? \shellcmd{size\_t} has to be downcast to an integer.

A note, in C++14, \shellcmd{auto} can be used as the return type of a function.

\subsection{Range Based For Loops (\shellcmd{for (auto)})}
With \shellcmd{auto}, range based for loops are possible! What are these strange loops? They are just a way to iterate through all the element of a collection type (i.e., arrays).

\begin{lstlisting}[style=cC++]
    auto  coolTeachers = { "Illya", "Andrea", "Not Price" };

	for (auto teacher : coolTeachers) {
		cout << teacher << " ";
	}
\end{lstlisting}

As you can guess, the output is ``\texttt{Illya Andrea Not Price}''. Raaad. Some reasons it might be useful.

\begin{itemize}
    \item Iterating through more complex data structures becomes much easier.
    \item It's more readable.
    \item You don't have to worry about proper indexing.
\end{itemize}

\section{Lambda Expressions}
Such a scary name for a scary topic. The simplest definition that will make thinking about them easier:

\begin{quote}
    Functions as variables and types.
\end{quote}

That's it. Now, the syntax:

\begin{lstlisting}[backgroundcolor=\color{lightgray},frame=single,basicstyle=\linespread{2}\ttfamily,emph={capture,parameters,functionBody},emphstyle={\syntax}]
    [ capture ]( parameters ) { functionBody }
\end{lstlisting}

For now, ignore the \syntax{capture} part. We will not be using it for this course. But now here is where shit gets crazy. Let's do an example.

\begin{lstlisting}[style=cC++]
    std::function<bool(int, int)> lessThan = [](int x, int y) { return x < y; };
\end{lstlisting}

First, notice the ugly return type \shellcmd{std::function<bool(int, int)>} --- how can we get rid of it? \shellcmd{auto}! Second, it's not so bad! It sort of like a function but we are assigning it to a variable called \shellcmd{lessThan}.

\begin{lstlisting}[style=cC++]
    auto lessThan = [](int x, int y) { return x < y; };

    cout << lessThan(128, 256); /* Prints 1 for true */
\end{lstlisting}

How else did we use variables? Functions!

\begin{lstlisting}[style=cC++]
auto sort(std::function<bool(int, int)> compare, int array[], const int size) {
	// This is just a bubble sort
	for (int i = 0; i < size; i++) {
		for (int j = 0; j < size - i - 1; j++) {

			// Here's where we use compare
			if (compare(array[j], array[j + 1])) {
				int temporary = array[j];
				array[j] = array[j + 1];
				array[j + 1] = temporary;
			}
		}
	}
}

int main(int argc, char *argv[]) {
	auto lessThan = [](int x, int y) { return x < y; };
	int array[5] = { 1, 2, 3, 4, 5 };
	sort(lessThan, array, 5);

	for (auto element : array) {
		cout << element << " ";
	}
}
\end{lstlisting}

We can also return lambda expressions! Don't mind the \shellcmd{[]}, again it's an advanced feature. For this, all you have to know is it signifies the variable is ``held'' in temporary memory.

\begin{lstlisting}[style=cC++]
auto counter(const int start, const int incrementor) {
	return [=]() {
		static auto x = start;
		x += incrementor;

		return x;
	};
}

int main(int argc, char *argv[]) {
	auto count = counter(42, 2);

	cout << count() << "\n" << count() << "\n" << count();
	/* prints 44, 46, 48 */
}
\end{lstlisting}

We can use them exactly as we have used all variables --- hold them in arrays

\begin{lstlisting}[style=cC++,escapechar=ä]
    auto styleChecker = []() {
        ä$\vdots$ä
		std::cout << "Style Checking...\n";
	};
	auto plagarismChecker = []() {
        ä$\vdots$ä
		std::cout << "Checking For Plagarism (Caught 2)...\n";
	};
	auto inputOutputChecker = []() {
        ä$\vdots$ä
		std::cout << "Checking Input/Output...\n";
	};

	std::function<void()> graderStack[5] = { styleChecker, plagarismChecker, inputOutputChecker };

	for (auto gradeFunction: graderStack) {
		if (gradeFunction != NULL) {
			gradeFunction();
		}
	}
\end{lstlisting}

Notice the checking of line \shellcmd{gradeFunction != NULL}. We have three functions in the array, but the size is five. If try to call the function, without a function in there, \textit{it will cause a runtime error}. So be careful.

And finally, we can add them as variables to classes. I won't give an example here, because it's pretty straightforward.

\section{Constant Expressions \shellcmd{constexpr}}
\shellcmd{constexpr} is a way to move calculations from runtime to compile time. So, if you have a function that's going to compute \textbf{known} values ahead of time. So, if you're going to be calculating $\ln 2$ ahead of time often,

\begin{lstlisting}[style=cC++,emph={constexpr},emphstyle={\color{blue}\textit}]
constexpr double ln(const int x) {
	double sum = 0;

	for (int n = 1; n <= 100; n++) {
		sum += 1.0/n/(4*n - 2);
	}

	return sum;
}
\end{lstlisting}

Why might this be useful? Performance.

\section{The \shellcmd{default} And \shellcmd{delete} Keywords}
Remember when you implemented one constructor, and suddenly you lost the copy constructor and the assignment operator? Welp, it took roughly 30 years to figure out this was a pain in the ass for everyone. Now, if you overload the copy constructor, you can get the default constructor by prototyping it, then putting \shellcmd{= default}.

\begin{lstlisting}[style=cC++]
class Zombie {
	string nameOfGradStudent;

public:
	Zombie(const string name) {
		nameOfGradStudent = name;
	}


};

int main(int argc, char *argv[]) {
	Zombie Fred; // compiler error
}
\end{lstlisting}

This won't compile, Fred doesn't have a default constructor. But, if did something like

\begin{lstlisting}[style=cC++,escapechar=ä,emphstyle={\textit}]
class Zombie {
	string nameOfGradStudent;

public:
	Zombie(const string name) {
		nameOfGradStudent = name;
	}

    ä\shellcmd{Zombie() = \textcolor{blue}{default};}ä
};

int main(int argc, char *argv[]) {
	Zombie Fred; // compiler error
}
\end{lstlisting}

\shellcmd{delete} works the same way, except the opposite. Instead of gaining the default, it simply deletes the function. So, if there is no way possible to have a default constructor for the \texttt{Zombie} class, something like this is possible:

\begin{lstlisting}[style=cC++,escapechar=ä,emphstyle={\textit}]
class Zombie {
	string nameOfGradStudent;

public:
	Zombie(const string name) {
		nameOfGradStudent = name;
	}

    ä\shellcmd{Zombie() = \textcolor{blue}{delete};}ä
};

int main(int argc, char *argv[]) {
	Zombie Fred; // compiler error
}
\end{lstlisting}

This is useful when you're creating a class for someone else to use. It makes the deletion of certain functionality more explicit.

\section{Further Topics}
If time permitted, some topics of further discussion:

\begin{itemize}
    \item Preventing Exception Propagation (\shellcmd{noexcept})
    \item Smart Pointers
    \item Move Semantics
    \item Variadic Templates
    \item There's no way we'd get here.
\end{itemize}

\end{document}
