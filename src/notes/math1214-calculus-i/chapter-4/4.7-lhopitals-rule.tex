%
%  lhopitals-rule.tex
%  chapter-4
%
%  Created by Illya Starikov on 07/02/17.
%  Copyright 2017. Illya Starikov. All rights reserved.
%

\subsection{L'H\^opital's Rule}
Suppose $f$ and $g$ are differentiable on an open interval $I$ containing $a$ with $g'(x) \neq 0$ on $I$ when $x \neq a$. If $\limit{x \rightarrow a} f(x) = \limit{x \rightarrow a} g(x) = 0$ or $\pm\infty$, then 

\begin{equation}
    \lim_{x\rightarrow a} \frac{f(x)}{g(x)} = \lim_{x \rightarrow a} \frac{f'(x)}{g'(x)}
\end{equation}

provided the limit on the right exists (or is $\pm \infty$). The rule also applies if $x \rightarrow a$ is replaced by $x \rightarrow \pm\infty$, $x \rightarrow a^+$, or $x \rightarrow a^-$.

\subsubsection{Indeterminate Forms $1^\infty$, $0^0$ and $\infty^0$}
Assume $\limit{x \rightarrow a} f(x)^{g(x)}$ has the indeterminate form $1^{\infty}$, $0^0$, or $\infty^0$.

\begin{enumerate}
    \item Evaluate $L = \limit{x \rightarrow a} g(x) \ln f(x)$. This limit can be put in the form $0 / 0$ or $\infty / \infty$, both of which are handled by l'H\^opital's rule.
    \item Then $\limit{x \rightarrow a} f(x) ^{g(x)} = e^L$
\end{enumerate}


\subsubsection{Growth Rates of Functions (as $x \rightarrow \infty$)}
Suppose $f$ and $g$ are functions with $\limit{x \rightarrow \infty} f(x) = \limit{x \rightarrow \infty} g(x) = \infty$. Then \textbf{$f$ grows faster than g} as $x \rightarrow \infty$ if

\begin{equation}
    \lim_{x \rightarrow \infty} \frac{g(x)}{f(x)} = 0 \quad \text{ or, quantitatively, if} \quad \lim _{x \rightarrow \infty} \frac{f(x)}{g(x)} = \infty
\end{equation}

The functions $f$ and $g$ have \textbf{comparable growth rates} if

\begin{equation*}
    \lim _{x \rightarrow \infty} \frac{f(x)}{g(x)} = M
\end{equation*}

where $M \in \mathbb{R}^+.$

\subsubsection{Ranking Growth Rates as $x \rightarrow \infty$}
Let $f << g$ mean that $g$ grows faster than $f$ as $f \rightarrow \infty$. With positive real numbers $p, q, r, s$ and $b > 1$,

\begin{equation}
    \ln ^q x << x^p << x^p \ln ^r x << x ^{p + s} << b^x << x^x
\end{equation}
