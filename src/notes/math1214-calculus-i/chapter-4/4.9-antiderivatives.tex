%
%  4.9-antiderivatives.tex
%  chapter-4
%
%  Created by Illya Starikov on 07/03/17.
%  Copyright 2017. Illya Starikov. All rights reserved.
%

\subsection{Antiderivatives}
A function $F$ is an \textbf{antiderivative} of $f$ on an interval $I$ provided $F'(x) = f'(x)$, for all $x$ in $I$.

\subsubsection{The Family of Antiderivatives}
Let $F$ be any antiderivative of $f$ on an interval $I$. Then \textit{all} antiderivatives of $f$ on $I$ have the form $F + C$, where $C$ is an arbitrary constant.

\subsubsection{Power Rule for Indefinite Integrals}
\begin{equation}
    \int x^p\, dx = \frac{d^{p + 1}}{p + 1} + C,
\end{equation}

where $p \neq -1$ is a real number and $C$ is an arbitrary constant.

\subsubsection{Constant Multiple and Sum Rules}
The constant multiple rule is as follows:

\begin{equation}
    \int c f(x)\, dx = c \int f(x)\, dx
\end{equation}

The sum rules as follows:

\begin{equation}
    \int (f(x) + g(x))\, dx
\end{equation}

\subsubsection{Indefinite Integrals of Trigonometric Functions}
\begin{align}
    \frac{d}{dx}(\sin ax) = a\, \cos ax \quad&\rightarrow\quad \int \cos ax\, dx = \frac{1}{a} \sin ax + C \\
    \frac{d}{dx}(\cos ax) = -a\, \sin ax \quad&\rightarrow\quad \int \sin ax\, dx = \frac{1}{a} -\cos ax + C \\
    \frac{d}{dx}(\tan ax) = a\, \sec^2 ax \quad&\rightarrow\quad \int \sec^2 ax\, dx = \frac{1}{a} \tan ax + C \\
    \frac{d}{dx}(\cot ax) = -a\, \csc^2 ax \quad&\rightarrow\quad \int \csc^2 ax\, dx = -\frac{1}{a} \cot ax + C \\
    \frac{d}{dx}(\sec ax) = a\, \sec ax \tan ax \quad&\rightarrow\quad \int \sec ax \tan ax \, dx = \frac{1}{a} \sec ax + C \\
    \frac{d}{dx}(\csc ax) = -a\, \csc ax \cot ax \quad&\rightarrow\quad \int \csc ax \cot ax\, dx = -\frac{1}{a} \csc ax + C 
\end{align}
