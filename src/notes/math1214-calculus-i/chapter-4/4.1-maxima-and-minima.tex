%
%  4.1-maxima-and-minima.tex
%  chapter-4
%
%  Created by Illya Starikov on 07/01/17.
%  Copyright 2017. Illya Starikov. All rights reserved.
%

\subsection{Maxima and Minima}
\subsubsection{Absolute Maximum and Minimum}
Let $f$ be defined on an interval $I$ containing $c$. If $f(c) \geq f(x)$ for every $x$ in $I$, then $f$ has an \textbf{absolute maximum} value of $f(c)$ on $I$ at $c$. If $f(c) \leq f(x)$ for every $x$ in $I$, then $f$ has an \textbf{absolute minimum} value of $f(c)$ on $I$ at $c$.

\subsubsection{Extreme Value Theorem}
A function that is continuous on a closed interval $[a, b]$ has an absolute maximum value and an absolute minimum value on that interval.

\subsubsection{Local Extreme Point Theorem}
If $f$ has a local maximum or minimum value at $c$ and $f'(c)$ exists, then $f'(c) = 0$.

\subsubsection{Critical Point}
An interior point $c$ of the domain of $f$ at which $f'(c) = 0$ or $f'(c)$ fails to exist is called a \textbf{critical point} of $f$.

\subsubsection{Locating Absolute Maximum and Minimum Values}
Assume the function $f$ is continuous on the closed interval $[a, b]$.

\begin{enumerate}
    \item Locate the critical points $c$ in $(a, b)$, where $f'(c) = 0$ or $f'(c)$ does not exist. These points are candidates for absolute maximum and minimum. 
    \item Evaluate $f$ at the critical points and at the endpoints of $[a, b]$.
    \item Choose the largest and smallest values of $f$ from the previous step for the absolute maximum and minimum values, respectively.
\end{enumerate}
