%
%  4.3-graphing-functions.tex
%  chapter-4
%
%  Created by Illya Starikov on 07/02/17.
%  Copyright 2017. Illya Starikov. All rights reserved.
%

\subsection{Graphing Functions}
\subsubsection{Graphing Guidelines for $y = f(x)$}
\begin{enumdescript}
    \item[Identify the domain or interval of interest.] On what interval should the function be graphed? It may be the domain of the function or some subset of the domain.

    \item[Exploit symmetry.] Take advantage of symmetry. For example, is the function \textit{even} ($f(-x) = f(x)$), \textit{odd} ($f(-x) = -f(x)$) or neither?

    \item[Find the first and second derivatives.] They are needed to determine extreme values, concavity, inflection points, and intervals of increase and decrease. Computing derivatives—particularly second derivatives may not be practical, so some functions may need to be graphed without complete derivative information.

    \item[Find critical points and possible inflection points.] Determine points at which $f'(x) = 0$ or $f'$ is undefined. Determine points at which $f''(x) = 0$ or $f''$   is undefined.

    \item[Find intervals on which the function is increasing/decreasing and concave up/down.] The first derivative determines the intervals of increase and decrease. The second derivative determines the intervals on which the function is concave up or concave down.

    \item[Identify extreme values and inflection points.] Use either the First or the Second Derivative Test to classify the critical points. Both x- and y-coordinates of maxima, minima, and inflection points are needed for graphing.

    \item[Locate vertical/horizontal asymptotes and determine end behavior.] Vertical asymptotes often occur at zeros of denominators. Horizontal asymptotes require examining limits as $x \rightarrow \pm \infty$; these limits determine end behavior.

    \item[Find the intercepts.] The y-intercept of the graph is found by setting $x = 0$. The x-intercepts are the real zeros (or roots) of a function: those values of $x$ that satisfy $f(x) = 0$.
\end{enumdescript}
