%
%  2.4-infinite-limits.tex
%  chapter-2
%
%  Created by Illya Starikov on 06/11/17.
%  Copyright 2017. Illya Starikov. All rights reserved.
%

\subsection{Infinite Limits}
Suppose $f$ is defined for all $x$ near $a$. If $f(x)$ grows arbitrarily large for all $x$ sufficiently close (but not equal) to $a$, we write

\begin{equation}
    \lim_{x \rightarrow a} f(x) = \infty
\end{equation}

We say the limit of $f(x)$ as $x$ approaches $a$ is infinity.

If $f(x)$ is negative and grows arbitrarily large in magnitude for all $x$ sufficiently close (but not equal) to $a$, we write

\begin{equation}
    \lim_{x \rightarrow a} f(x) = -\infty
\end{equation}

In this case, we say the limit of $f(x)$ as $x$ approaches $a$ is negative infinity. In both cases, \textit{the limit does not exist.}

\subsubsection{One-Sided Infinite Limits}
Suppose $f$ is defined for all $x$ near $a$ with $x > a$. If $f(x)$ becomes arbitrarily large for all $x$ sufficiently close to $a$ with $x > a$, we write $\limit{x \rightarrow a^+} f(x) = \infty$. The one-sided infinite limit $\limit{x \rightarrow a^+} = -\infty$, $\limit{x \rightarrow a^-} f(x) = \infty$, and $\limit{x \rightarrow a^-} = -\infty$ are defined analogously.

\subsubsection{Vertical Asymptote}
If $\limit{x \rightarrow a} f(x) = \pm \infty$, $\limit{x \rightarrow a^+} = \pm \infty$ or $\limit{x \rightarrow a^-} f(x) = \pm \infty$, the line $x = a$ is called a \textbf{vertical asymptote} of $f$.
