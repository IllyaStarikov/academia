%
%  2.2-limits-of-a-function -preliminary.tex
%  chapter-1
%
%  Created by Illya Starikov on 06/11/17.
%  Copyright 2017. Illya Starikov. All rights reserved.
%

\subsection{Definitions of Limits}
\subsubsection{Limits of a Function (Preliminary)}
Suppose the function $f$ is defined for all $x$ near $a$ except possibly at $a$. If $f(x)$ is arbitrarily close to $L$ (as close to $L$ as we like) for all $x$ sufficiently close (but not equal) to $a$, we write

\begin{equation*}
    \lim_{x \rightarrow a} f(x) = L
\end{equation*}

and say the limit of $f(x)$ as $x$ approaches $a$ equals $L$.

\subsubsection{One-Sided Limits}
\begin{enumdescript}
    \item[Right-sided limits] Suppose $f$ is defined for all $x$ near $a$ with $x > a$. If $f(x)$ is arbitrarily close to $L$ for all $x$ sufficiently close to $a$ with $x > a$, we write

    \begin{equation}
        \lim_{x \rightarrow a^+} f(x) = L
    \end{equation}

    and say the limit of $f(x)$ as $x$ approaches $a$ from the right equals $L$.

    \item[Left-sided limits] Suppose $f$ is defined for all $x$ near $a$ with $x < a$. If $f(x)$ is arbitrarily close to $L$ for all $x$ sufficiently close to $a$ with $x < a$, we write

    \begin{equation}
        \lim_{x \rightarrow a^-} f(x) = L
    \end{equation}

    and say the limit of $f(x)$ as $x$ approaches $a$ from the left equals $L$.
\end{enumdescript}

\subsubsection{Relationship Between One-Sided and Two-Sided Limits}
Assume $f$ is defined for all $x$ near $a$ except possibly at $a$. Then $\limit{x \rightarrow a} f(x) = L$ if and only if $\limit{x \rightarrow a^+} = \limit{x \rightarrow a^-} = L$.
