%
%  3.2-rules-of-differentiation.tex
%  chapter-3
%
%  Created by Illya Starikov on 06/18/17.
%  Copyright 2017. Illya Starikov. All rights reserved.
%

\subsubsection{Constant Rule}
If $c$ is a real number, then $\frac{d}{dx}(c) = 0$.

\subsubsection{Power Rule}
If $n$ is a positive integer, then $\frac{d}{dx}(x^n) = nx^{n - 1}$.

\subsubsection{Constant Multiple Rule}
If $f$ is differentiable at $x$ and $c$ is a constant, then 

\begin{equation}
    \frac{d}{dx}(cf(x)) = cf'(x)
\end{equation}

\subsubsection{The Number $e$}
The number $e = 2.7182 \ldots$ satisfies

\begin{equation}
    \lim_{h \rightarrow 0} \frac{e^h - 1}{h} = 1
\end{equation}

It is the base of the natural exponential function $f(x) = e^x$.

\subsubsection{The Derivative of $e^x$}
The function $f(x) = e^x$ is differentiable, for all real numbers $x$, and 

\begin{equation}
    \frac{d}{dx} (e^x) = e^x
\end{equation}

\subsubsection{Higher-Order Derivatives}
Assuming $f$ can be differentiated as often as necessary, the \textbf{second derivative} of $f$ is

\begin{equation}
    f''(x) = f^{(2)}(x) = \frac{d^2 f}{dx^2} = \frac{d}{dx}\left(f'(x)\right)
\end{equation}

For integers $n \geq 1$, the \textbf{$n$th derivate} is

\begin{equation}
    f^{(n)}(x) = \frac{{d^n}f}{dx^n} = \frac{d}{dx}\left(f^{(n-1)}(x)\right)
\end{equation}
