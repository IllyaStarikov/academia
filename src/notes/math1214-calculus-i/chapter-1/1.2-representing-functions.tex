%
%  1.2-representing-functions.tex
%  chapter-1
%
%  Created by Illya Starikov on 05/17/17.
%  Copyright 2017. Illya Starikov. All rights reserved.
%

\subsection{Representing Functions}
Some brief families of functions can include

\begin{description}
    \item[Polynomials] are functions of the form

        \begin{equation*}
            f(x) = a_n\, x^n\, + a_{n-1}\,x^{n-1} + \cdots + a_1\, x + a_0,
        \end{equation*}

        where the \textbf{coefficients} $a_0, a_1, \ldots, a_n$ are real numbers with $a_n \neq 0$ and the nonnegative integer $n$ is the \textbf{degree} of the polynomial. The domain of any polynomial is the  set of all real numbers. An $n$th-degree polynomial can have as many as $n$ real \textbf{zeros} or \textbf{roots} --- values of $x$.
    \item[Rational Functions] are ratios of the form $f(x) = p(x) / q(x)$, where $p$ and $q$ are polynomials. Because division by zero is prohibited, the domain of a rational function is the set of all real numbers except those for which the denominator is zero.
    \item[Algebriac Functions]  are constructed using the operations of algebra: addition, subtraction, multiplication, division, and roots. Examples of algebraic functions are $f(x) = \sqrt{2x^3 + 4}$ and $f(x) = x^{\nicefrac{1}{4}}(x^3 + 2)$. In general, if an even root (square root, fourth root, and so forth) appears, then the domain does not contain points at which the quantity under the root is negative (and perhaps other points).
    \item[Exponential Functions] have the form $f(x) = b^x$, where the base $b \neq 1$ is a positive real number. Closely associated with exponential functions are logarithmic functions of the form $f(x) = \log _b x$, where $b > 0$ and $b \neq q$. An exponential function has a domain consisting of all real numbers. Logarithmic functions are defined for positive, real numbers.
        The most important function is the \textbf{natural exponential function} $f(x) = e^x$, with base $b = e$, where $e \approx 2.71828\ldots$ is one of the fundamental constants of mathematics. Associated with the natural exponential function is the \textbf{natural logarithmic function} $f(x) = \ln x$, which also has the base $b = e$.
    \item[Trigonometric Functions] are $\sin x$, $\cos x$, $\tan x$, $\sec x$, and $\cos x$; they are fundamental to mathematics and many areas of application. Also important are their relatives, the \textbf{inverse trigonometric functions}.
    \item[Transcendental Functions] Trigonometric, exponential, and logarithmic functions are few examples of a large family called transcendental functions.

\end{description}

\subsubsection{Transformations}
Given the real numbers $a$, $b$, $c$, and $d$ and the function $f$, the graph of $y = cf(a(x - b)) + d$ is obtained from the graph of $y = f(x)$ in the following steps.

\begin{align*}
    y = f(x) &\xrightarrow{\text{horizontal scaling by a factor of |$a$|}} y = f(ax) \\
             &\xrightarrow{\text{horizontal shift by $b$ units}} y = f(a(x - b)) \\
             &\xrightarrow{\text{vertical scaling by a factor of |$c$|}} y = cf(a(x - b)) \\
             &\xrightarrow{\text{horizontal scaling by a factor of |a|}} y = cf(a(x - b)) + d \\
\end{align*}