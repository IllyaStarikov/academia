%
%  1.4-trigonometric-functions-and-their-inverses.tex
%  chapter-1
%
%  Created by Illya Starikov on 05/17/17.
%  Copyright 2017. Illya Starikov. All rights reserved.
%

\subsection{Trigonometric Functions and Their Inverses}
Let $P(x,\, y)$ be a point on a circle of radius $r$ associated with the angle $\theta$. Then

\begin{align}
    \sin\theta = \frac{y}{r} \qquad \cos\theta &= \frac{x}{r} \qquad \tan\theta = \frac{y}{x} \\
    \cot\theta = \frac{x}{y} \qquad \sec\theta &= \frac{r}{x} \qquad \csc\theta = \frac{r}{y} \\
\end{align}

\subsubsection{Trigonometric Identities}
\noindent\textbf{Reciprocal Identities}
\begin{align}
    \tan\theta = \frac{\sin\theta}{\cos\theta} &\qquad \cot\theta = \frac{1}{\tan\theta} = \frac{\cos\theta}{\sin\theta} \\
    \csc\theta = \frac{1}{\sin\theta} &\qquad \sec\theta = \frac{1}{\cos\theta}
\end{align}

\noindent\textbf{Pythagorean}
\begin{equation}
    \sin^2\theta + \cos^2\theta = 1 \qquad 1 + \cot^2\theta = \csc^2\theta \qquad \tan^2\theta + 1 = \sec^2\theta
\end{equation}

\noindent\textbf{Double- and Half-Angle Formulas}
\begin{align}
    \sin 2\theta = 2\sin\theta \cos\theta &\qquad \cos 2\theta = \sin^2\theta - \sin^2\theta \\
    \cos^2\theta = \frac{1 + \cos 2\theta}{2} &\qquad \sin^2\theta = \frac{1 - \cos 2\theta}{2}
\end{align}

\subsubsection{Period of Trigonometric Function}
The function $\sin\theta$, $\cos\theta$, $\sec\theta$, and $\csc\theta$ have a period of $2\pi$

\begin{align}
    \sin(\theta + 2\pi) = \sin\theta &\qquad \cos(\theta + 2\pi) = \cos\theta \\
    \sec(\theta + 2\pi) = \sec\theta &\qquad \csc(\theta + 2\pi) = \csc\theta
\end{align}

for all $\theta$ in the domain.

The functions $\tan\theta$ and $\cot \theta$ have a period of $\pi$:

\begin{equation}
    \tan(\theta + \pi) = \tan\theta \qquad \cot(\theta + \pi) = \cot\theta
\end{equation}

for all $\theta$ in the domain.

\subsubsection{Inverse Sine and Cosine}
$y = \sin^{-1} x$ is the value of $y$ such that $x = \sin y$, where $-\frac{\pi}{2} \leq y \leq \frac{\pi}{2}$. $y = \cos^{-1} x$ is the value of $y$ such that $x = \cos y$, where $0 \leq y \leq \pi$. The domain of both $\sin^{-1} x$ and $\cos^{-1} x$ is $\{ x: -1 \leq x \leq 1 \}$.

\subsubsection{Other Inverse Trigonometric Functions}
\begin{itemize}
    \item $\tan^{-1} x$ is the value of $y$ such that $x = \tan y$, where $-\frac{\pi}{2} < \frac{\pi}{2}$.
    \item $\cot^{-1} x$ is the value of $y$ such that $x = \tan y$, where $0 < y < \pi$.
\end{itemize}

The domain of both $\tan^{-1} x$ and $\cot^{-1} x$ is $\{ x: -\infty < x < \infty \}$

\begin{itemize}
    \item $\sec^{-1} x$ is the value of $y$ such that $x = \sec y$, where $0 < y < \pi$ with $y \neq \frac{\pi}{2}$
    \item $\tan^{-1} x$ is the value of $y$ such that $x = \tan y$, where $-\frac{\pi}{2} < \frac{\pi}{2}$.
\end{itemize}