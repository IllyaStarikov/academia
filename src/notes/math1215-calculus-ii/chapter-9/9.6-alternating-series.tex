\subsection{Alternating Series}
\subsubsection{The Alternating Series Test}
The alternating series $\sum (-1)^{k+1} a_k$ converges provided

\begin{enumerate}
    \item the terms of the series are non-increasing in magnitude ($0 < a_{k+1} \leq a_k$, for $k$ greater than some index $N$) and
    \item $\limit{k \rightarrow \infty} = 0$
\end{enumerate}

\subsubsection{Alternating Harmonic Series}
The alternating harmonic series $\sum _{k = 1} ^{\infty} \frac{(-1)^{k + 1}}{k} = 1 - \frac{1}{2} + \frac{1}{3} - \frac{1}{4} + \frac{1}{5}- \cdots$ converges (even though the harmonic series $\sum _{k = 1} ^{\infty} \frac{1}{k} = 1 + \frac{1}{2} + \frac{1}{3} + \frac{1}{4} + \frac{1}{5} + \cdots$ diverges).

\subsubsection{Remainder in Alternating Series}
Let $R_n = |S - S_{n}|$ be the remainder in approximating the value of a convergent alternating series $\sum _{k = 1} ^{\infty} (-1)^{k + 1} a_k$ by the sum of its first $n$ terms. Then $R_n \leq a_{n + 1}$. In other words, the remainder is less than or equal to the magnitude of the first neglected term.

\subsubsection{Absolute and Conditional Convergence}
Assume the infinite series $\sum a_k$ converges. The series $\sum a_k$ \textbf{converges absolutely} if the series $\sum |a_k|$ converges. Otherwise, the series $\sum a_k$ \textbf{converges conditionally}.

\subsubsection{Absolute Convergence Implies Convergence}
If $\sum |a_k|$ converges, then $\sum a_k$ converges (absolute convergence implies convergence). If $\sum a_k$ diverges, then $\sum |a_k|$ diverges.
