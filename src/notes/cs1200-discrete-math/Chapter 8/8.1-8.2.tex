\documentclass[12pt]{minimal}
\begin{document}
Let $x = {a,b,c}$. Define relation \textbf{J} on $P(x)$ as follows: $\forall A,B \in P(x), A \textbf{J} B \Leftrightarrow A \cap B \not = \not 0$ \tabularnewline

\noindent
a) Is $\{A\} \textbf{J} \{C\}$? No, because $\{A\} \cup \{C\} = \not 0$ \\
b) Is $\{a,b\} \textbf{J} \{b,c\}$? Yes, because $\{a,b\} \cap \{b,c\} = \{b\} \not = \not 0$\\
c) Is $\{a,b\} \textbf{J} \{a,b,c\}?$ Yes, because $\{a,b\} \cap \{a,b,c\} = \{a,b\} \not = \not 0$\\

Let $A = \{3,4,5\}, B = \{4,5,6\}$ and let \textbf{S} be the ``Divides'' relation: $\forall \(x,y\) \in A \times B, x \textbf{S} y, \Leftrightarrow x | y$\tabularnewline

\noindent
a) $S = \{\(3,6\),\(4,4\),\(5,5\)\}$ \\
b) $ S^{-1} = \{\(6,3\),\(4,4\),\(5,5\)\}$ \\


Suppose a function $F: X \rightarrow Y$ is onto but not one-to-one. Is $F^{-1}$ a function? \tabularnewline
No, because if $F: X \rightarrow Y$ is not one-to-one, then $\exists (\ x_1, x_2 )\ \in X_n \bigwedge \exists y \in Y, x_1 \not = x_2 \bigwedge (\ x_1,y )\ \in F \bigwedge (\ x_2,y )\ \in F$. But this implies $\exists x_1, x_2 \in X \bigwedge \exists \y \in Y, x_1 \not x_2 \bigwedge (\ y,x_1)\ \in F^-1 \bigwedge (\y, x_2)\ \in \F^{-1}.$ Consequently, $F^{-1}$ does not satisfy property (2) of the definition of function. \\

Define relations \textbf{R} and \textbf{S} on \mathbb{R} as follows: \\
$\textbf{R} = \{x,y\} \in \mathbb{R} x \mathbb{R} | x^2 + y^2 = 4 $ Square! \\
$\textbf{S} = \{x,y\} \in \mathbb{R} x \mathbb{R} | x = y $ Line! \\
Graph $R, S, R \cup S, R \cap S$ in the Cartesian plane. \textit{R, Circle. S, Line. R and S, both of them. R or S, just the intersection points.} \\

Let $A = \{2, 3, 4, 6, 7, 9\} $ and define a relation $\mathbb{R}$ on A as follows: $\forall x,y \in A, x \textbf{R} y \Leftrightarrow 3 | (x - y)$.

\textit{
\noindent
1. Each element has a loop to itself. \\
2. If element a is related to element b, then b is also related to a.\\
3. In each case where there is an arrow from one point to another point, and that to a third, there is an arrow going from the first to the third.\\
}


\textbf{
1. Reflixive? Yes. \\
2. Symmetric? HELL YES. \\
3. Transitive? HELL TO THE HELL TO THE YES \\
}

R is reflexive iff $\forall x \in A, x \textbf{R} x$
R is symmetric iff $\forall x, y \in A, x \textbf{R} y \rightarrow x \textbf{R} y$
R is transitive iff $\forall x, y, z \in A, x \textbf{R} y \bigwedge y \textbf{R} z \rightarrow x \textbf{R} z $

Let $A = \{0,1,2,3\}$ and define relations \textbf{R, S, T} as follows:
R: $\{ (\ 0,0 )\ , (\ 0,1 )\ , (\ 0,3 )\ , (\ 1,0 )\ , (\ 1,1 )\ , (\ 2,2 )\ , (\ 3,0 )\ , (\ 3,3 )\ \}$
S: $\{ (\ 0,0 )\ , (\ 0,2 )\ , (\ 0,3 )\  (\ 2,3 )\ \}$
T: $\{ (\ 0,1 )\ , (\ 2,3 )\ \}$


\noindent
a) Is \textbf{R} reflexive, symmetric, transitive? Yes, Yes, Nah.\\
b) Is \textbf{S} reflexive, symmetric, transitive? No, no, yasss.\\
c) Is \textbf{T} reflexive, symmetric, transitive? No, no, yasss.\\

\end{document}