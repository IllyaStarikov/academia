\documentclass{article}
\usepackage{amsmath}
\begin{document}

\setcounter{section}{10}
\setcounter{subsection}{2}

\subsection{Matrix Representations of Graphs}
An \textbf{m $\times$ n} (read “\textit{m} by \textit{n}”) \textbf{matrix A over a set S} is a rectangular array of elements of \textit{S} arranged into \textit{m} rows and \textit{n} columns.

We write \textbf{A} = $(a_{ij})$

\subsubsection{Adjacency Matrix}
Let \textit{G} be a directed graph with ordered vertices $v_1, v_2, \ldots , v_n$ . The adjacency matrix of \textit{G} is the $n \times n$ matrix $\textbf{A} = ( a_{ij} )$ over the set of nonnegative integers such that

\begin{equation}
a_{ij} = \text{the number of arrows from} \ v_i \ \text{to} \ v_j \ \text{for all} \ i, j = 1, 2, \ldots, n.
\end{equation}

\subsubsection{Adjacency Matrix}
Let G be an undirected graph with ordered vertices $v_1, v_2, \ldots,v_n$. The adjacency matrix of G is the $n \times n$ matrix $A = ( a_{ij} )$ over the set of nonnegative integers such that

\begin{equation}
a_{ij} = \text{the number of edges connecting} \ v_i \ \text{and} \ v_j
\end{equation}

for all $i, j = 1, 2, \ldots, n$.

\subsubsection{Symmetric}
An $n \times n$ square matrix $A = ( a_{ij} )$ is called \textbf{symmetric} if, and only if, for all $i, j =
1, 2, \ldots, n,$

\begin{equation}
a_{ij} = a_{ji}
\end{equation}

\subsubsection{Identity Matrix}
For each positive integer \textit{n}, the $n \times n$ \textbf{identity matrix}, denoted \textbf{$I_n = (\delta {ij})$} or just \textbf{I} (if the size of the matrix is obvious from context), is the $n \times n$ matrix in which all the entries in the main diagonal are 1’s and all other entries are 0’s. In other words,

\[
\delta _{ij} =
\begin{cases}
1 \ \text{if} \ i = j \\
0 \ \text{if} \ i \neq j
\end{cases}
\text{for all} \ i, j = 1, 2, \ldots, n
\],

\subsubsection{The Number of Walks}
If \textit{G} is a graph with vertices $v_1,v_2, \ldots, v_m$ and \textit{A} is the adjacency matrix of \textit{G}, then for each positive integer \textit{n} and for all integers $i, j = 1, 2, ldots , m,$

\begin{center}
the \textit{ij}th entry of $A_n$ = the number of walks of length \textit{n} from $v_i$ to $v_j$.
\end{center}

\setcounter{subsection}{4}
\subsection{Trees}
\subsubsection{Terminology}
A graph is said to be \textbf{circuit-free} if, and only if, it has no circuits. A graph is called a \textbf{tree} if, and only if, it is circuit-free and connected. A \textbf{trivial tree} is a graph that consists of a single vertex. A graph is called a \textbf{forest} if, and only if, it is circuit-free and not connected.

\subsubsection{Vertex-Degree Relation}
Any tree that has more than one vertex has at least one vertex of degree 1.

\subsubsection{Terminal Vertex}
Let \textit{T} be a tree. If \textit{T} has only one or two vertices, then each is called a \textbf{terminal vertex}. If \textit{T} has at least three vertices, then a vertex of degree 1 in \textit{T} is called a \textbf{terminal vertex} (or a \textbf{leaf}), and a vertex of degree greater than 1 in \textit{T} is called an \textbf{internal vertex} (or a \textbf{branch vertex}).

\subsubsection{Edges And Vertices}
For any positive integer \textit{n}, any tree with \textit{n} vertices has \textit{n - 1} edges.

\subsection{Connected, Circuit}
If \textit{G} is any connected graph, \textit{C} is any circuit in \textit{G}, and any one of the edges of \textit{C} is removed from \textit{G}, then the graph that remains is connected.



\end{document}