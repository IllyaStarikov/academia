\documentclass{article}
\usepackage{amsmath}
\begin{document}

\setcounter{section}{9}
\setcounter{subsection}{4}

\subsection{R-Combination}
Let \textit{n} and \textit{r} be nonnegative integers with $r \leq n$. An \textbf{r-combination} of a set of n elements is a subset of \textit{r} of the n elements. As indicated in Section 5.1, the symbol

\[
{{n}\choose{r}}
\]

which is read \underline{n choose r}, denotes the number of subsets of size \textit{r} (\textit{r}-combinations) that can be chosen from a set of \textit{n} elements.

\subsubsection{Definition}
The number of subsets of size \textit{r} (or \textit{r}-combinations) that can be chosen from a set of \textit{n} elements,  \textit{n}, is given by the formula

\[
{{n}\choose{r}}
 = \frac{P(n, r)}{r!}\]

or, equivalently,

\[
{{n}\choose{r}}
 = \frac{n!}{r!(n - r)!}\]

where \textit{n} and \textit{r} are nonnegative integers with $r \leq n$.

\subsubsection{Permutations with sets of Indistinguishable Objects}
Suppose a collection consists of \textit{n} objects of which

\begin{center}
$n_1$ are of type 1 and are indistinguishable from each other.
$n_2$ are of type 2 and are indistinguishable from each other.
$\vdots$
$n_k$ are of type k and are indistinguishable from each other.
\end{center}

and suppose that $n_1 + n_2 + \cdots + n_k = n$. Then the number of distinguishable.

\[
{{n}\choose{n_1}}{{n - n_1}\choose{n_2}}{{n - n_1 - n_2}\choose{n_3}}\cdots{{n - n_1 - n_2 - \cdots - n_k{k - 1}}\choose{n_k}}
\]

\begin{equation}
\frac{n!}{n_1!n_2!n_3! \cdots n_k!}
\end{equation}

\subsection{r-Combinations with Repetition Allowed}
An \textbf{r-combination with repetition allowed}, or \textbf{multiset of size \textit{r}}, chosen from a set \textit{X} of \textit{n} elements is an unordered selection of elements taken from \textit{X} with repetition allowed. If $X = \{ x_1, x_2, \ldots, x_n \},$ we write an \textit{r}-combination with repetition allowed,or multiset of size \textit{r}, as \textbf{ $[ x_{i1}, x_{i2}, \ldots, x_i]$ }, where each $x_ij$ is in \textit{X} and some of the $x_ij$ may equal.

\subsubsection{Theorem}
The number of r-combinations with repetition allowed (multisets of size r) that can be selected from a set of n elements is

\[
{{r+ n - 1}\choose{r}}
\]

This equals the number of ways \textit{r} objects can be selected from \textit{n} categories of objects with repetition allowed.

\end{document}