\documentclass{article}
\usepackage[utf8]{inputenc}
\usepackage[english]{babel}
\usepackage{amsmath}
\begin{document}
\setcounter{section}{8}
\section{Counting And Probability}
\subsection{Introduction}

\subsubsection{Sample Space}
A \textbf{sample space} is the set of all possible outcomes of a random process or experiment. An \textbf{event} is a subset of a sample space.

\subsubsection{Equally Likely Probability Formula}
If \textit{S} is a finite sample space in which all outcomes are equally likely and \textit{E} is an event in \textit{S}, then the \textbf{probability of E}, denoted \textbf{P(E)}, is

\begin{equation}
P(E) = \frac{\text{the number of outcomes in \textit{E}}}{\text{the total number of outcomes in \textit{S}}}
\end{equation}

The notation is as follows: For any finite set \textit{A, N(A)} denotes the number of elements in \textit{A}.

\subsubsection{The Number of Elements in a List}
If \textit{m} and \textit{n} are integers and $m \leq n$, then there are \textit{n - m + 1} integers from \textit{m} to \textit{n} inclusive.

\subsection{Possibility Trees and the Multiplication Rule}
\subsubsection{The Multiplication Rule}
If an operation consists of \textit{k} steps and

\begin{center}
the first step can be performed $n_1$ ways,
the second step can be performed in $n_2$ ways [\textit{regardless of how the first step was performed}]
$\vdots$
the \textit{k}th step can be performed in $n_k$ ways [\textit{regardless of how the preceding steps were performed}],
\end{center}

then the entire operation can be performed in $n_1 n_2 \cdots n_k$ ways.

\subsubsection{Number of Permutations}
For any integer \textit{n} with $n \geq 1$, the number of permutations of a set with \textit{n} elements is \textit{n!}.

\subsubsection{R-Permutation}
An \textbf{r-permutation} of a set of \textit{n} elements is an ordered selection of \textit{r} elements taken from the set of \textit{n} elements. The number of r-permutations of a set of \textit{n} elements is denoted \textbf{P(n, r)}.

If \textit{n} and \textit{r} are integers and $1 \leq r \leq n$, then the number of \textit{r}-permutations of a set of \textit{n} elements is given by the formula

\begin{equation*}
P(n, r) = n(n - 1)(n - 2) \cdots (n - r + 1)
\end{equation*}

or, equivalently,

\begin{equation*}
P(n, r) = \frac{n!}{(n - 2)!}
\end{equation*}

\end{document}