\documentclass{article}
\usepackage{amsmath}
\begin{document}

\setcounter{section}{10}
\setcounter{subsection}{2}
\subsection{Matrix Representations of Graphs}
\subsubsection{Example 10.3.19}
\[\text{Let} \ A =
\begin{bmatrix}
1 & 1 & 2 \\
1 & 0 & 1 \\
2 & 1 & 0
\end{bmatrix}
\]


\[
A^2 =
\begin{bmatrix}
1 & 1 & 2 \\
1 & 0 & 1 \\
2 & 1 & 0
\end{bmatrix}
\times
\begin{bmatrix}
1 & 1 & 2 \\
1 & 0 & 1 \\
2 & 1 & 0
\end{bmatrix}
=
\begin{bmatrix}
6 & 3 & 3 \\
3 & 2 & 2 \\
3 & 2 & 5
\end{bmatrix}
\]

\[
A^3 =
\begin{bmatrix}
1 & 1 & 2 \\
1 & 0 & 1 \\
2 & 1 & 0
\end{bmatrix}
\times
\begin{bmatrix}
6 & 3 & 3 \\
3 & 2 & 2 \\
3 & 2 & 5
\end{bmatrix}
=
\begin{bmatrix}
15 & 9 & 15 \\
9 & 5 & 8 \\
15 & 8 & 8
\end{bmatrix}
\]


Assume \textit{A} is an adjacney matrix. How many walks of length 2 from $v_1$ to $v_3$? 3. \\

Assume \textit{A} is an adjacney matrix. How many walks of length 3 from $v_1$ to $v_3$? 15. \\

Examine the calculations you performed to find five walks of length 2 form $v_3$ to $v_3$. Then draw \textit{G} and find the walks by visual inspection.

\begin{equation}
2 \cdot 2 + 1 \cdot 1 + 0 \cdot 0
\end{equation}

Left edges from $v_3$ to $v_1$, right is edges from $v_1$ to $v_3$. $2 \times 2$ = walk of length 2 from $v_3$ to $v_3$ via $v_1$.

\setcounter{subsection}{1}
\subsection{Matrix Representations of Graphs}
\textit{Graph in book}.
\begin{itemize}
\item How many from \textit{a} to \textit{c}? \textbf{4.}
\item How many trails from \textit{a} to \textit{c}? \textbf{4 + 3 + 2 + 1}
\item How many walks form \textit{a} to \textit{c}? $\infty$
\end{itemize}

\subsubsection{Example 10.3.6b}
An edge whose removal disconnects graph it is part of is called an \underline{bridge}. Find the bridge in:

\textit{The graph is in the book}

\begin{itemize}
\item $<v_7, v_8>$
\item $<v_3, v_4>$
\item $<v_1, v_2>$
\end{itemize}

\subsubsection{Example 10.2.8d}
Find the number of connected components in: \textit{Image in book.} \textbf{2.}

\subsubsection{Euler Circuit}
Let \textit{G} be a graph. An Euler circuit for \textit{G} is a circuit that contains every vertex and every every edge of \textit{G}. That is, an Euler Circuit for \textit{G} is a sequence of adjacent vertices and edges in \textit{G} that has at least one edge, starts and ends in the same vertex, uses every vertex of \textit{G} at least once, and uses every edge exactly edge exactly one.

\subsubsection{Theorem 10.2.2}
If a graph has an Euler circuit, then every vertex has even degree.

\subsubsection{Theorem 10.2.3}
if a graph \textit{G} is connected and the degree of every vertex is even, then \textit{G} has an Euler circuit.

\subsubsection{Example 10.2.6}
\textit{Insert graph here.}


\end{document}