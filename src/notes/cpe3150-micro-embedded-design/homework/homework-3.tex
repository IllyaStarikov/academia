\RequirePackage[l2tabu, orthodox]{nag}
\documentclass[12pt]{article}

\usepackage{amssymb,amsmath,verbatim,graphicx,microtype,upquote,units,booktabs,enumitem,akkwidepage,multicol}
\newcounter{map}
\newcommand{\address}{
    \textbf{000\themap}
    \addtocounter{map}{1}
}
\newcommand{\contents}[1]{
    \texttt{#1}
}

\title{Homework \#3}
\date{Due Date: September 22, 2016}
\author{Illya Starikov}

\begin{document}
\maketitle

\section*{Problem \#1}
\begin{enumerate}[label=(\alph*)]
    \item Illegal (\textit{\texttt{ADD} only applies to A.})
    \item Illegal (\textit{\texttt{255H} is too large})
    \item Legal
    \item Illegal (\textit{Same reasoning as (a)})
    \item Legal
    \item Illegal (\textit{\texttt{F5H} will be treated as a label instead of a number.})
\end{enumerate}

\section*{Problem \#2}
\begin{verbatim}
    MOV R4, #25H      ; Moves 37 into register 4.
    MOV A, #1FH       ; Moves 31 into temporary register A
    ADD A, R4         ; Adds register 4's value to register A's value
\end{verbatim}

It's a simple add operation, resulting in the value of $37_{10} + 31_{10} = 68_{10} = 44_{16}$; it's kept in ROM.

\section*{Problem \#3}
\begin{multicols}{2}
\begin{verbatim}
    ORG 0;
    MOV R4, #0FH    ; Move 0F to register 4
    MOV A, #0A2H    ; Move A2 to register A
    ADD A, R4       ; Add R4 To A
    MOV R7, A       ; Move the result to R7
    END;
\end{verbatim}

\begin{center}
    \begin{tabular}{|c|c|}
        \hline
        \textbf{Address} & \textbf{Contents} \\ \hline
        \address & \contents{0F} \\ \hline
        \address & \contents{A2} \\ \hline
        \address & \contents{B1} \\ \hline
        \address & \contents{B1} \\ \hline
    \end{tabular}
\end{center}

\end{multicols}

\section*{Problem \#4}
\begin{verbatim}
        ORG 0

        MOV R0, #30H        ; Move the value 30H to R0
LOOP:   MOV @R0, #0FFH      ; Move FFH at R0
        INC R0              ; Increment R0
        CJNE R0, #40H, LOOP

        END
\end{verbatim}

\section*{Problem \#5}
\begin{verbatim}
    ORG 0;

    SETB PSW.3      ; Switch to register bank 3
    MOV R0, #0A5H   ; Load A5 into register 0
    PUSH 0          ; Push R0 onto stack

    SETB PSW.0      ; Switch to register bank 4
    POP 5           ; Pop onto register 5

    END;
\end{verbatim}

\section*{Problem \#6}
Assuming A is the register to store results,

\begin{verbatim}
    ORG 0;

    MOV R6, #5FH        ; Load 5F into R6
    MOV R5, #01001001B  ; Load 01001001 into R5

    /* Swap */
    MOV A, R6;          ; temp (A) = R6
    MOV 6, 5;           ; R6 = R5
    MOV R5, A           ; R5 = Temp (A)

    /* Add */
    ADD A, R6;          ; A already had R6's values, just add original R5
                        ; Which got switched.. so actually add R6
    ADD A, #20;

    END;
\end{verbatim}
\end{document}