\RequirePackage[l2tabu, orthodox]{nag}
\documentclass[12pt]{article}

\usepackage{amssymb,amsmath,verbatim,graphicx,microtype,upquote,units,booktabs,siunitx,fullpage,mathtools,listings}
\usepackage[dvipsnames]{xcolor}
\setcounter{secnumdepth}{0}

\title{Homework \#8}
\date{Due Date: December 1\textsuperscript{st}, 2016}
\author{Illya Starikov}

\lstdefinelanguage
   [x64]{Assembler}     % add a "x64" dialect of Assembler
   [x86masm]{Assembler} % based on the "x86masm" dialect
   % with these extra keywords:
   {morekeywords={ACALL, CPL}} % etc.

\lstdefinestyle{cC}{
    language=C,
    basicstyle=\footnotesize\ttfamily,
    keywordstyle=\color{blue}\ttfamily,
    stringstyle=\color{red}\ttfamily,
    commentstyle=\color{gray}\ttfamily,
    morecomment=[l][\color{magenta}]{\#},
    numbers=left,
    keepspaces=true,
    tabsize=4,
    breaklines=true,
    morekeywords={true, false, bool}
}

\lstdefinestyle{cASM}{
    language=[x64]Assembler,
    basicstyle=\footnotesize\ttfamily,
    keywordstyle=\color{blue}\ttfamily,
    stringstyle=\color{red}\ttfamily,
    commentstyle=\color{gray}\ttfamily,
    morecomment=[s][\color{magenta}]{\#}{\ },
    morecomment=[s][\color{gray}]{/*}{*/},
    numbers=left,
    keepspaces=true,
    tabsize=2,
    breaklines=true,
    emph={R0, R1, R2, R3, R4, R5, R6, A, B},
    emphstyle={\color{ForestGreen}},
    keywords={},
    morekeywords={LJMP, RETI, DJNZ, CPL, MOV, INC, ACALL, SJMP, SETB, RET, CLR, JC, ADD,
                  CJNE},
}
\begin{document}
\maketitle

\noindent This homework will be \textbf{extra credit.}

\section{Problem \#1}
\begin{lstlisting}[style=cC]
#include <reg51.h>

typedef enum { false, true } bool;
unsigned char R0;

void main() {
	TMOD = 0x06; // Counter 2, Mode 2
	TH0 = -0x60; // Initial value given

	R0 = 0; // Just in case
	TR0 = true; // Start the count

	do {
		if (TF0) {
			R0++;
			TF0 = false; // continue the count
		}
	} while (R0 != 60);

	TR0 = false; // Stop the count
}
\end{lstlisting}

\section{Problem \#2}
\begin{lstlisting}[style=cASM]
        ORG 0H

        LJMP MAIN

        /* Timer 1 Interrupt */
        ORG 001BH
        CPL P1.1
        RETI

        /* Main */
        ORG 0030H
MAIN:   MOV P2, #00H        ; P2 = output port
        MOV TMOD, #20H      ; Timer 1, mode 2
        MOV TH1, #-37D      ; 5 kHz => 25000 Hz => 4^-5 s
                            ; => 40 us => 40/1.085 ticks => 37 ticks
        MOV IE, #88H
        SETB TR1

LOOP:   INC R0
        MOV P2, R0          ; P2 = R0
        ACALL DELAY
        SJMP LOOP

; so we need to generate a 10.85us delay
; 10.85/1.085 = 10MC. Easier to just do a brute force
DELAY:  MOV R1, #4D         ; 2 MC
HERE:   DJNZ R1, HERE       ; 4 * 2 = 8 MC
        RET                 ; 8 + 2 = 10 MC


        END
\end{lstlisting}
\begin{figure}[!ht]
    \centering
    \includegraphics{meme.jpg}
\end{figure}

\section{Problem \#3}
\begin{lstlisting}[style=cASM]
        ORG 0
        LJMP MAIN

        /* Timer 0 Interrupt */
        ORG 000B
        CPL P1.1
        RETI

        /* Timer 1 Interrupt */
        ORG 001B
        CPL P1.2
        RETI

        /* Main */
        ORG 30H
MAIN:   MOV TMOD, #12H          ; T0M2 & T1M1
        MOV P1, #0              ; P1 = output port
        ACALL WAVE0
        ACALL WAVE1

        MOV IE, #8AH
REPEAT: SJMP REPEAT

; 5kHz => 1/5000 s = .0002 s => 200 microseconds
; 200 us => 184 ticks PER PERIOD. Assuming a 50%
; duty cycle. So actual value is 184/2 = 92 ticks
WAVE0:  MOV TH0, #-92
        SETB TR0
        RET

; 500 Hz => .002 s => 2000 us
; (FFFF - x + 1)(1.085) = 2000
; x = FFFF - 1842 => E7 BD
WAVE1:  MOV TH1, #0E7H
        MOV TL1, #0BDH
        SETB TR1
        RET


        END
\end{lstlisting}
\begin{figure}[!ht]
    \centering
    \includegraphics{meme2.jpg}
\end{figure}

\section{Problem \#4}
\begin{lstlisting}[style=cASM]
        ORG 0H
        LJMP MAIN

        /* Interrupts */
        ORG 001BH
        LJMP SWAVE
        RETI

        /* Main */
        ORG 30H
MAIN:   MOV R7, #1D
        MOV P1, #0              ; Make P1 an output port
        MOV IE, #88H            ; Enable interrupt on T1

LOOP:   MOV P2, R7              ; R7 holds the value of the div series
        ACALL INFSER
        SJMP LOOP

; code for the square wave interrupt
; determines via R6 what the last wave was
SWAVE:  MOV TMOD, #20H  ; Timer 2, mode 2

        CJNE R6, #70, SKIP2LOW
        ACALL HWAVE
        SJMP NEXT

SKIP2LOW:
        ACALL LWAVE

NEXT:   SETB TR1
        RET

; the wave has a duty cycle of 70%
; so the high wave needs a high wave of 7kHz
; & kHz => 7000 Hz => 1.42^-4 s => 142 us => 130 ticks
; 130.8 *.7 = 92 ticks for 70% duty cycle of a 7kHz wave
HWAVE:
        SETB P1.1
        MOV TH1, #-92
        MOV R6, #70             ; Just to keep track of what our last value was
        RET

; so, exact same reasoning as before, we need 130 ticks for
; the period of the wave, but now we just subtract 92 from 130
; so, 130 - 92 = 38
LWAVE:
        CLR P1.1
        MOV TH1, #-38
        MOV R6, #30
        RET

; computes infinite divergent series incrementally
; uses R7
INFSER:
        MOV A, R7
        ACALL TWOCOMP
        ACALL ISNEGATIVE

        JC RETURN
        INC A

RETURN: MOV R7, A
        RET


; returns two's compliment of A
TWOCOMP:
        CPL A
        ADD A, #1D
        RET


; returns 1 in C if B is negative
ISNEGATIVE:
        CJNE A, #10000000B, NEXT2
NEXT2:  CPL C
        RET


        END
\end{lstlisting}
\begin{figure}[!ht]
    \centering
    \includegraphics{meme3.jpg}
\end{figure}

\end{document}
