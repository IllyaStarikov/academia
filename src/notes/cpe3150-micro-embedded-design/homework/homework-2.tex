\RequirePackage[l2tabu, orthodox]{nag}
\documentclass{article}

\usepackage{amssymb,amsmath,verbatim,graphicx,microtype,upquote,units,booktabs,tabularx,multicol}
\usepackage[margin=10pt, font=small, labelfont=bf, labelsep=endash]{caption}
\usepackage[colorlinks=true, pdfborder={0 0 0}]{hyperref}
\usepackage[utf8]{inputenc}

\newcounter{map}
\newcommand{\address}{
    \textbf{000\themap}
    \addtocounter{map}{1}
}
\newcommand{\contents}[1]{
    \texttt{#1}
}

\title{Homework \#2}
\date{Due Date: September 15, 2016}
\author{Illya Starikov}

\begin{document}
\maketitle

\begin{enumerate}
    \item Printer, Disk, Keyboard, Monitor.

    \item Because Windows and Android both support the x86 architecture, there is a monopolization of the x86 architecture support --- meaning millions of computers could run x86 compatible software. Other benefits could include more functionality (i.e. BIOS Bootloading) and computing power.

    \item The ROM is where the program is stores; too large of a program could mean not being able to run the executable.

    \item Power consumption.

    \item Addressing modes are as follows:
    \begin{enumerate}
        \item Directed, Indexed
        \item Indexed, Immediate
        \item Absolute
        \item Direct
        \item Relative
        \item Register, Relative
        \item Register
        \item Long
    \end{enumerate}

    \item Addressing maps is as follows:
\end{enumerate}

\begin{center}
    \begin{tabular}{|c|c|}
        \hline
        \textbf{Address} & \textbf{Contents} \\ \hline
        \address & \contents{79} \\ \hline
        \address & \contents{A3} \\ \hline
        \address & \contents{74} \\ \hline
        \address & \contents{2D} \\ \hline
        \address & \contents{27} \\ \hline
        \address & \contents{04} \\ \hline
        \address & \contents{C4} \\ \hline
        \address & \contents{82} \\ \hline
    \end{tabular}
\end{center}

\end{document}