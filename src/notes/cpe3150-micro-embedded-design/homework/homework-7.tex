\RequirePackage[l2tabu, orthodox]{nag}
\documentclass[12pt]{article}

\usepackage{amssymb,amsmath,verbatim,graphicx,microtype,upquote,units,booktabs,siunitx,fullpage,mathtools}
\setcounter{secnumdepth}{0}

\title{Homework \#7}
\date{Due Date: November 10\textsuperscript{th}, 2016}
\author{Illya Starikov}

\DeclarePairedDelimiter\ceil{\lceil}{\rceil}
\DeclarePairedDelimiter\floor{\lfloor}{\rfloor}

\begin{document}
\maketitle

\section{Problem \#1}
\begin{align}
    (65536 - x) \times \SI{1.085}{\micro\second} = \SI{5}{\milli\second} &= \SI{5000}{\micro\second} \\
    &= 60928_{10} \\
    &= EE00_{16}
\end{align}

Therefore, \texttt{TH1} = $EE_{16}$ and \texttt{TL1} = $00_{16}$

\section{Problem \#2}
Assuming Problem \#1 is correct,

\begin{verbatim}
        ORG 0H

        MOV R0, #100D
LOOP:   MOV TMOD, #01D
        MOV TH0, #EEH
        MOV TL0, #00H   ; (FFFF - EE00 + 1)*1.085 = 5ms
        ACALL DELAY
        DJNZ R0, LOOP   ; 100*5ms = 500 ms

        AJMP DONE

DELAY:  SETB TR0
LOOP2:  JNB TF0, LOOP2
        CLR TR0
        CLR TF0
        RET

DONE:   NOP
        END
\end{verbatim}

\section{Problem \#3}
\SI{5}{\kilo\hertz} = \SI{500}{\micro\second}. So,

\begin{equation}
    \mathtt{TH1} = EE_{16} \quad\text{and}\quad \mathtt{TL1} = 00_{16}
\end{equation}

\noindent The code is as follows,

\begin{verbatim}
        ORG 0H

        MOV R0, #100D
        CLR P0.0

LOOP:   MOV TMOD, #01D
        MOV TH0, #EEH
        MOV TL0, #00H   ; (FFFF - EE00 + 1)*1.085 = 5ms
        ACALL DELAY

        CJNE R0, #75D, SKIP
        SETB P0.0

SKIP:   DJNZ R0, LOOP   ; 100*5ms = 500 ms

        AJMP DONE

DELAY:  SETB TR0
LOOP2:  JNB TF0, LOOP2
        CLR TR0
        CLR TF0
        RET

DONE:   NOP
        END
\end{verbatim}

\section{Problem \#4}
Because we know frequency ($f$) = $\frac{1}{\text{period ($T$)}}, \floor*{f} \implies \ceil*{T}, \{ \forall f,\, T \in \mathbb{R}^+ | \ f \neq 0, T \neq 0 \}$. Using this, we find the smallest period to be,

\begin{equation}
    (FFFF_{16} - FFFF_{16} + 1) \cdot \SI{1.085}{\micro\second} = \SI{1.085}{\micro\second}
\end{equation}

\noindent From this, we use $f = \frac{1}{T} = \frac{1}{1.085} = \SI{0.92}{\micro\hertz}$. We do a similar process for lowest frequency.


\begin{verbatim}
        ORG 0H
        CLR P1.3

LOOP:   MOV TMOD, #01D
        MOV TH0, #0D    ; lowest frequency => highest period
        MOV TL0, #0D    ; highest period => Smallest TH, TL
        SETB P1.3
        ACALL DELAY

        CLR P1.3
        AJMP DONE

DELAY:  SETB TR0
LOOP2:  JNB TF0, LOOP2
        CLR TR0
        CLR TF0
        RET

DONE:   NOP
        END
\end{verbatim}


\section{Problem \#5}
\subsection{Part A}
\begin{align}
    2C_{16} + 11_{16} &\implies 2C11_{16} \\
    FFFF_{16} - 2C11_{16} + 1_{16} &= D3EF_{16} \\
    &= 54254_{10} \\
    &\implies \SI{58865.59}{\micro\second} \\
    &= \SI{58.87}{\milli\second}
\end{align}

\subsection{Part B}
\begin{align}
    1001\, 0011_2 \implies 93_{16} &\implies 147_{10} \\
    (65536 - 147) \times 1.085 &= \SI{70947.07}{\micro\second} \\
    &= \SI{70.95}{\milli\second}
\end{align}

\end{document}
