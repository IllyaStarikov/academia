\RequirePackage[l2tabu, orthodox]{nag}
\documentclass[12pt]{article}

\usetheme[logo=template-presentation/logo,faculty=ped]{fibeamer}

\usepackage{amssymb,amsmath,verbatim,graphicx,pdfpages,microtype,units,booktabs,upquote,xcolor,siunitx,csquotes,fancyvrb,newverbs,wrapfig,multicol,tikz,textcomp,wrapfig,cutwin}
\usepackage{fontawesome,setspace,rotchiffre,lipsum,listings,animate,listings}
\usepackage[xspace]{ellipsis}

\hypersetup{%
            colorlinks = true,
            linkcolor = orange,
            urlcolor  = orange,
            citecolor = orange,
            anchorcolor = orange}

\newcommand{\hugeslide}[1]{%
\begin{frame}[plain,c]
    \centering {\usebeamerfont*{frametitle} \usebeamercolor[fg]{frametitle}{\fontsize{40}{50}\selectfont\textit{#1}}}
\end{frame}
}

\newcommand{\presentaddcount}[1]{\addtocounter{#1}{1}\Roman{#1}}
\newcommand{\presentcount}[1]{\Roman{#1}}
\newcommand{\shellcmd}[1]{\texttt{\colorbox{gray!30}{#1}}}

\lstdefinelanguage{swift}
{%
  morekeywords={%
    func,if,then,else,for,in,while,do,switch,case,default,where,break,continue,fallthrough,return,
    typealias,struct,class,enum,protocol,var,func,let,get,set,willSet,didSet,inout,init,deinit,extension,
    subscript,prefix,operator,infix,postfix,precedence,associativity,left,right,none,convenience,dynamic,
    final,lazy,mutating,nonmutating,optional,override,required,static,unowned,safe,weak,internal,
    private,public,is,as,self,unsafe,dynamicType,true,false,nil,Type,Protocol,print
  },
  morecomment=[l]{//}, % l is for line comment
  morecomment=[s]{/*}{*/}, % s is for start and end delimiter
  morestring=[b]" % defines that strings are enclosed in double quotes
}

\definecolor{keyword}{HTML}{BA2CA3}
\definecolor{string}{HTML}{D12F1B}
\definecolor{comment}{HTML}{008400}
\definecolor{type}{HTML}{66B9AA}

\lstdefinestyle{Swift}{%
  language=swift,
  basicstyle=\ttfamily,
  showstringspaces=false, % lets spaces in strings appear as real spaces
  columns=fixed,
  keepspaces=true,
  keywordstyle=\color{keyword},
  stringstyle=\color{string},
  commentstyle=\color{comment},
  emph={Int,Character,Double,Float,Unsigned},
  emphstyle={\color{type}},
  morestring=[b]",
  escapeinside={(*}{*)}
}

\newcommand\syntaxbox[2][fill=orange!80]{%
    \tikz[baseline]\node[%
        inner ysep=0pt,
        inner xsep=2pt,
        anchor=text,
        rectangle,
        rounded corners=1mm,
        #1] {\strut#2};%
}


\chapterNumber{6}

\begin{document}
\section{Arithmetic and Logic Instructions}
\begin{itemize}
    \item Unpacked = 4 0s out front
    \item Packed = no 0s out front, more effecient.
    \begin{itemize}
        \item However, there are reasons for having 0000s out front
        \item Could process nibble, instead of byte by byte
        \item Makes processing easier
    \end{itemize}

    \item BCD addition \textbf{does not} work.. kind of.
    \item \texttt{DA} takes into account the \texttt{AC} as well.
    \begin{itemize}
        \item If $\texttt{AC} = 1$ or $> 9$, add $06H$
        \item If $\texttt{CY} = 1$ or $> 9$, add $60H$.
    \end{itemize}

    \item The extra \texttt{B} in \texttt{SUBB} mean subtracted with borrow.
    \begin{itemize}
        \item Make sure to set clear \texttt{CY} before using \texttt{SUBB}
    \end{itemize}

    \item One special case, divide by zero: $\texttt{OV} = 1$, values remain the same.
    \begin{itemize}
        \item Division example: $\texttt{A} = 9, \texttt{B} = 5$
    \end{itemize}

    \item \texttt{XRL} only works for 8 bits.
    \begin{itemize}
        \item Same addressing modes as for \texttt{ANL}
    \end{itemize}

    \item Compliment works for \texttt{A}, \texttt{C} or \textit{anything} that is bit addressable.

    \item \texttt{CJNE} \textbf{changes the CY flag}
    \item Serial Communication example (we use RLC because we want to use the Carry flag for transmitting data)
\end{itemize}

\begin{verbatim}
    ORG 0

    MOV A, #35H
    MOV P2, #0
    MOV R0, #8

    SETB P2.1
    SETB P2.1

TX: RLC A
    MOV P2.1, C
    DJNZ R0, TX

    SETB P2.1
    SETB P2.1

    END
\end{verbatim}

\begin{itemize}
    \item Same example, backwards
\end{itemize}

\begin{verbatim}
    ORG 0

    MOV R0, #8

RX: MOV C, P2.5
    RRC A
    DJNZ R0, RX

    MOV R2, A

    END
\end{verbatim}

 \begin{itemize}
    Number of `$1$''s example
 \end{itemize}

\begin{verbatim}
    ORG 0
    MOV R0, #0  ; Counter for 1s
    MOV R1, #8  ; Counter for loop

    MOV A, P2

LP: RRC A

    DJNE
\end{verbatim}


\begin{equation}
    n \text{bit} \cdot n \text{bit} = 2n \text{bit}
\end{equation}

\begin{verbatim}
    ORG 0
    MOV R0, #30H
    MOV @R0, #0

    XCHD A, @R0     ; M[30] = 07, A = 30H
    SWAP A          ; A = 30H
    ORL A, #30H
\end{verbatim}



\end{document}
