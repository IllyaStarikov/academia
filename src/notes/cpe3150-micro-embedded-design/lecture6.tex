\RequirePackage[l2tabu, orthodox]{nag}
\documentclass[12pt]{article}

\usepackage{amssymb,amsmath,verbatim,graphicx,microtype,upquote,units,booktabs,akkwidepage}

\newcommand{\chapterNumber}[1]{
    \setcounter{section}{#1}
    \addtocounter{section}{-1}
}
\chapterNumber{6}

\begin{document}
\section{Arithmetic and Logic Instructions}
\begin{itemize}
    \item Unpacked = 4 0s out front
    \item Packed = no 0s out front, more effecient.
    \begin{itemize}
        \item However, there are reasons for having 0000s out front
        \item Could process nibble, instead of byte by byte
        \item Makes processing easier
    \end{itemize}

    \item BCD addition \textbf{does not} work.. kind of.
    \item \texttt{DA} takes into account the \texttt{AC} as well.
    \begin{itemize}
        \item If $\texttt{AC} = 1$ or $> 9$, add $06H$
        \item If $\texttt{CY} = 1$ or $> 9$, add $60H$.
    \end{itemize}

    \item The extra \texttt{B} in \texttt{SUBB} mean subtracted with borrow.
    \begin{itemize}
        \item Make sure to set clear \texttt{CY} before using \texttt{SUBB}
    \end{itemize}

    \item One special case, divide by zero: $\texttt{OV} = 1$, values remain the same.
    \begin{itemize}
        \item Division example: $\texttt{A} = 9, \texttt{B} = 5$
    \end{itemize}

    \item \texttt{XRL} only works for 8 bits.
    \begin{itemize}
        \item Same addressing modes as for \texttt{ANL}
    \end{itemize}

    \item Compliment works for \texttt{A}, \texttt{C} or \textit{anything} that is bit addressable.

    \item \texttt{CJNE} \textbf{changes the CY flag}
    \item Serial Communication example (we use RLC because we want to use the Carry flag for transmitting data)
\end{itemize}

\begin{verbatim}
    ORG 0

    MOV A, #35H
    MOV P2, #0
    MOV R0, #8

    SETB P2.1
    SETB P2.1

TX: RLC A
    MOV P2.1, C
    DJNZ R0, TX

    SETB P2.1
    SETB P2.1

    END
\end{verbatim}

\begin{itemize}
    \item Same example, backwards
\end{itemize}

\begin{verbatim}
    ORG 0

    MOV R0, #8

RX: MOV C, P2.5
    RRC A
    DJNZ R0, RX

    MOV R2, A

    END
\end{verbatim}

 \begin{itemize}
    Number of `$1$''s example
 \end{itemize}

\begin{verbatim}
    ORG 0
    MOV R0, #0  ; Counter for 1s
    MOV R1, #8  ; Counter for loop

    MOV A, P2

LP: RRC A

    DJNE
\end{verbatim}


\begin{equation}
    n \text{bit} \cdot n \text{bit} = 2n \text{bit}
\end{equation}

\begin{verbatim}
    ORG 0
    MOV R0, #30H
    MOV @R0, #0

    XCHD A, @R0     ; M[30] = 07, A = 30H
    SWAP A          ; A = 30H
    ORL A, #30H
\end{verbatim}



\end{document}
