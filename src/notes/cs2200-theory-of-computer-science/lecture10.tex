\RequirePackage[l2tabu, orthodox]{nag}
\documentclass[12pt]{article}

\usepackage{amssymb,amsmath,verbatim,graphicx,microtype,upquote,units,booktabs,akkwidepage}

\newcommand{\chapterNumber}[1]{
    \setcounter{section}{#1}
    \addtocounter{section}{-1}
}
\chapterNumber{10}

\begin{document}
\section{Turing Machines}

\begin{itemize}
    \item We still have..
    \begin{itemize}
        \item a set of states
        \item some final states
    \end{itemize}

    \item The input alphabet is a subset of the input alphabet
    \item We have a simbol for the space: $\square$.
    \begin{itemize}
        \item Not really different from $\lambda$
    \end{itemize}
    \item This is deterministic
    \item \textbf{You don't have to process all the input}
    \begin{itemize}
        \item All you have to do is end up in the final state
    \end{itemize}

    \item Imagine we have a tape, and some symbol $a$. The read-write head starts as some state $q_0$.
    \begin{itemize}
        \item Now we write over $a$ to a value $d$, we now are at $q_1$
    \end{itemize}

    \item Now suppose the input $abb$.
    \begin{enumerate}
        \item Start at $a$ ($q_0$).
        \item Move to $b$ ($q_1$)
        \item Move to $b$ ($q_1$)
    \end{enumerate}

    \item Example $abcd \implies ab q_1 c \vdash abe q_2 d$
    \begin{itemize}
        \item A way of describing what we go from to.
    \end{itemize}




\end{itemize}


\end{document}
