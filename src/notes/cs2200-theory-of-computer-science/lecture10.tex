\RequirePackage[l2tabu, orthodox]{nag}
\documentclass[12pt]{article}

\usetheme[logo=template-presentation/logo,faculty=ped]{fibeamer}

\usepackage{amssymb,amsmath,verbatim,graphicx,pdfpages,microtype,units,booktabs,upquote,xcolor,siunitx,csquotes,fancyvrb,newverbs,wrapfig,multicol,tikz,textcomp,wrapfig,cutwin}
\usepackage{fontawesome,setspace,rotchiffre,lipsum,listings,animate,listings}
\usepackage[xspace]{ellipsis}

\hypersetup{%
            colorlinks = true,
            linkcolor = orange,
            urlcolor  = orange,
            citecolor = orange,
            anchorcolor = orange}

\newcommand{\hugeslide}[1]{%
\begin{frame}[plain,c]
    \centering {\usebeamerfont*{frametitle} \usebeamercolor[fg]{frametitle}{\fontsize{40}{50}\selectfont\textit{#1}}}
\end{frame}
}

\newcommand{\presentaddcount}[1]{\addtocounter{#1}{1}\Roman{#1}}
\newcommand{\presentcount}[1]{\Roman{#1}}
\newcommand{\shellcmd}[1]{\texttt{\colorbox{gray!30}{#1}}}

\lstdefinelanguage{swift}
{%
  morekeywords={%
    func,if,then,else,for,in,while,do,switch,case,default,where,break,continue,fallthrough,return,
    typealias,struct,class,enum,protocol,var,func,let,get,set,willSet,didSet,inout,init,deinit,extension,
    subscript,prefix,operator,infix,postfix,precedence,associativity,left,right,none,convenience,dynamic,
    final,lazy,mutating,nonmutating,optional,override,required,static,unowned,safe,weak,internal,
    private,public,is,as,self,unsafe,dynamicType,true,false,nil,Type,Protocol,print
  },
  morecomment=[l]{//}, % l is for line comment
  morecomment=[s]{/*}{*/}, % s is for start and end delimiter
  morestring=[b]" % defines that strings are enclosed in double quotes
}

\definecolor{keyword}{HTML}{BA2CA3}
\definecolor{string}{HTML}{D12F1B}
\definecolor{comment}{HTML}{008400}
\definecolor{type}{HTML}{66B9AA}

\lstdefinestyle{Swift}{%
  language=swift,
  basicstyle=\ttfamily,
  showstringspaces=false, % lets spaces in strings appear as real spaces
  columns=fixed,
  keepspaces=true,
  keywordstyle=\color{keyword},
  stringstyle=\color{string},
  commentstyle=\color{comment},
  emph={Int,Character,Double,Float,Unsigned},
  emphstyle={\color{type}},
  morestring=[b]",
  escapeinside={(*}{*)}
}

\newcommand\syntaxbox[2][fill=orange!80]{%
    \tikz[baseline]\node[%
        inner ysep=0pt,
        inner xsep=2pt,
        anchor=text,
        rectangle,
        rounded corners=1mm,
        #1] {\strut#2};%
}


\chapterNumber{10}

\begin{document}
\section{Turing Machines}

\begin{itemize}
    \item We still have..
    \begin{itemize}
        \item a set of states
        \item some final states
    \end{itemize}

    \item The input alphabet is a subset of the input alphabet
    \item We have a simbol for the space: $\square$.
    \begin{itemize}
        \item Not really different from $\lambda$
    \end{itemize}
    \item This is deterministic
    \item \textbf{You don't have to process all the input}
    \begin{itemize}
        \item All you have to do is end up in the final state
    \end{itemize}

    \item Imagine we have a tape, and some symbol $a$. The read-write head starts as some state $q_0$.
    \begin{itemize}
        \item Now we write over $a$ to a value $d$, we now are at $q_1$
    \end{itemize}

    \item Now suppose the input $abb$.
    \begin{enumerate}
        \item Start at $a$ ($q_0$).
        \item Move to $b$ ($q_1$)
        \item Move to $b$ ($q_1$)
    \end{enumerate}

    \item Example $abcd \implies ab q_1 c \vdash abe q_2 d$
    \begin{itemize}
        \item A way of describing what we go from to.
    \end{itemize}




\end{itemize}


\end{document}
