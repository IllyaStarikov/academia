\RequirePackage[l2tabu, orthodox]{nag}
\documentclass[12pt]{article}

\usepackage{amssymb,amsmath,verbatim,graphicx,microtype,upquote,units,booktabs,listings,siunitx,fancyvrb,newverbs}
\usepackage[margin=10pt, font=small, labelfont=bf, labelsep=endash]{caption}
\usepackage[colorlinks=true, pdfborder={0 0 0}]{hyperref}
\usepackage[utf8]{inputenc}

\usepackage{xcolor}
\newcommand{\shellcmd}[1]{\texttt{\colorbox{gray!30}{#1}}}

\makeatletter
\renewcommand{\boxed}[1]{\text{\fboxsep=.2em\fbox{\m@th$\displaystyle#1$}}}
\makeatother

\lstdefinestyle{cC++}{
    language=C++,
    basicstyle=\footnotesize\ttfamily,
    keywordstyle=\color{blue}\ttfamily,
    stringstyle=\color{red}\ttfamily,
    commentstyle=\color{gray}\ttfamily,
    morecomment=[l][\color{magenta}]{\#},
    showstringspaces=false,
    numbers=left,
    keepspaces=true,
    tabsize=4,
    breaklines=true,
    morekeywords={}
}

\definecolor{cverbbg}{gray}{.7}
\newenvironment{lcverbatim}
 {\SaveVerbatim{cverb}}
 {\endSaveVerbatim
  \flushleft\fboxrule=0pt\fboxsep=.5em
  \colorbox{cverbbg}{%
    \makebox[\dimexpr\linewidth-2\fboxsep][l]{\BUseVerbatim{cverb}}%
  }
  \endflushleft
}

\title{Quiz \#11}
\date{\today}
\author{Ilya Starikov}

\begin{document}
\maketitle

\section{Source Code}
\begin{lstlisting}[style=cC++]
    /*    example.l

     	 Illya Starikov
    */

    %{
    /*
    Definitions of constants, variables, & function prototypes go here
    */

    #define T_IDENT     1
    #define T_INTCONST  2
    #define T_UNKNOWN   3
    #define T_FOO       4
    #define T_KEYWORD   5
    #define T_OPERATOR  6
    #define T_NUMBER    7
    #define T_ALPHA     8

    int numLines = 1;

    void printTokenInfo(const char* tokenType, const char* lexeme);

    %}

    /* Defintions of regular expressions go here */

    WSPACE		[ \t\v\r]+
    NEWLINE     \n

    KEYWORD     (var|if|then|else|while|true|false)
    OPERATOR    (;|\{|\}|;|\[|\]|=|\(|\)|\+|>|<|!=|>=|<=|==)
    NUMBER      -?{INTCONST}
    ALPHA       {LETTER}

    DIGIT	        [0-9]
    LETTER         	[a-zA-Z]

    IDENT			{LETTER}({LETTER}|{DIGIT})*
    INTCONST      	{DIGIT}+

    %%

    "foo"           {
    				printTokenInfo("FOO", yytext);
    				return T_FOO;
    			}
    {KEYWORD}  	{
    				printTokenInfo("KEYWORD", yytext);
    				return T_KEYWORD;
    			}
    {OPERATOR}	{
    				printTokenInfo("OPERATOR", yytext);
    				return T_OPERATOR;
    			}
    {ALPHA}	{
    				printTokenInfo("ALPHA", yytext);
    				return T_ALPHA;
    			}
    {INTCONST}	{
    				printTokenInfo("INTCONST", yytext);
    				return T_INTCONST;
    			}
    {INTCONST}	{
    				printTokenInfo("INTCONST", yytext);
    				return T_INTCONST;
    			}
    {IDENT}		{
    				printTokenInfo("IDENT", yytext);
    				return T_IDENT;
    			}
    {NEWLINE}	{
                          numLines++;
               	}
    {WSPACE}    { }
    .			{
    				printTokenInfo("UNKNOWN", yytext);
    				return T_UNKNOWN;
    			}
    %%

    // User-written code goes here

    void printTokenInfo(const char* tokenType, const char* lexeme)
    {
      printf("TOKEN: %s  LEXEME: %s\n", tokenType, lexeme);
    }

    // You should supply a yywrap function.
    // Having it return 1 means only 1 input file will be scanned.
    int yywrap() { return(1); }

    int main()
    {
      while ( yylex() ) ;  // Process tokens until 0 returned

      printf("Processed %d lines\n", numLines);
      return(0);
    }

\end{lstlisting}

\section{Input}
\begin{lcverbatim}
    var
    x; y; z;
    A[4];
    B[2][3];
    \$\% \#
    {
    x = 12;
    B[1][2] = x;
    A[x] = 5;
    while (x > 100) x = x + 1;
    if (A[2] <= A[B[x][y]]) then x = y + 1 else z = 10;
    \}
\end{lcverbatim}

\section{Output}
\begin{lcverbatim}
    TOKEN: KEYWORD  LEXEME: var
    TOKEN: ALPHA  LEXEME: x
    TOKEN: OPERATOR  LEXEME: ;
    TOKEN: ALPHA  LEXEME: y
    TOKEN: OPERATOR  LEXEME: ;
    TOKEN: ALPHA  LEXEME: z
    TOKEN: OPERATOR  LEXEME: ;
    TOKEN: ALPHA  LEXEME: A
    TOKEN: OPERATOR  LEXEME: [
    TOKEN: INTCONST  LEXEME: 4
    TOKEN: OPERATOR  LEXEME: ]
    TOKEN: OPERATOR  LEXEME: ;
    TOKEN: ALPHA  LEXEME: B
    TOKEN: OPERATOR  LEXEME: [
    TOKEN: INTCONST  LEXEME: 2
    TOKEN: OPERATOR  LEXEME: ]
    TOKEN: OPERATOR  LEXEME: [
    TOKEN: INTCONST  LEXEME: 3
    TOKEN: OPERATOR  LEXEME: ]
    TOKEN: OPERATOR  LEXEME: ;
    TOKEN: UNKNOWN  LEXEME: \$
    TOKEN: UNKNOWN  LEXEME: \%
    TOKEN: UNKNOWN  LEXEME: \#
    TOKEN: OPERATOR  LEXEME: {
    TOKEN: ALPHA  LEXEME: x
    TOKEN: OPERATOR  LEXEME: =
    TOKEN: INTCONST  LEXEME: 12
    TOKEN: OPERATOR  LEXEME: ;
    TOKEN: ALPHA  LEXEME: B
\end{lcverbatim}

\begin{lcverbatim}
    TOKEN: OPERATOR  LEXEME: [
    TOKEN: INTCONST  LEXEME: 1
    TOKEN: OPERATOR  LEXEME: ]
    TOKEN: OPERATOR  LEXEME: [
    TOKEN: INTCONST  LEXEME: 2
    TOKEN: OPERATOR  LEXEME: ]
    TOKEN: OPERATOR  LEXEME: =
    TOKEN: ALPHA  LEXEME: x
    TOKEN: OPERATOR  LEXEME: ;
    TOKEN: ALPHA  LEXEME: A
    TOKEN: OPERATOR  LEXEME: [
    TOKEN: ALPHA  LEXEME: x
    TOKEN: OPERATOR  LEXEME: ]
    TOKEN: OPERATOR  LEXEME: =
    TOKEN: INTCONST  LEXEME: 5
    TOKEN: OPERATOR  LEXEME: ;
    TOKEN: KEYWORD  LEXEME: while
    TOKEN: OPERATOR  LEXEME: (
    TOKEN: ALPHA  LEXEME: x
    TOKEN: OPERATOR  LEXEME: >
    TOKEN: INTCONST  LEXEME: 100
    TOKEN: OPERATOR  LEXEME: )
    TOKEN: ALPHA  LEXEME: x
    TOKEN: OPERATOR  LEXEME: =
    TOKEN: ALPHA  LEXEME: x
    TOKEN: OPERATOR  LEXEME: +
    TOKEN: INTCONST  LEXEME: 1
    TOKEN: OPERATOR  LEXEME: ;
\end{lcverbatim}

\begin{lcverbatim}
    TOKEN: KEYWORD  LEXEME: if
    TOKEN: OPERATOR  LEXEME: (
    TOKEN: ALPHA  LEXEME: A
    TOKEN: OPERATOR  LEXEME: [
    TOKEN: INTCONST  LEXEME: 2
    TOKEN: OPERATOR  LEXEME: ]
    TOKEN: OPERATOR  LEXEME: <=
    TOKEN: ALPHA  LEXEME: A
    TOKEN: OPERATOR  LEXEME: [
    TOKEN: ALPHA  LEXEME: B
    TOKEN: OPERATOR  LEXEME: [
    TOKEN: ALPHA  LEXEME: x
    TOKEN: OPERATOR  LEXEME: ]
    TOKEN: OPERATOR  LEXEME: [
    TOKEN: ALPHA  LEXEME: y
    TOKEN: OPERATOR  LEXEME: ]
    TOKEN: OPERATOR  LEXEME: ]
    TOKEN: OPERATOR  LEXEME: )
    TOKEN: KEYWORD  LEXEME: then
    TOKEN: ALPHA  LEXEME: x
    TOKEN: OPERATOR  LEXEME: =
    TOKEN: ALPHA  LEXEME: y
    TOKEN: OPERATOR  LEXEME: +
    TOKEN: INTCONST  LEXEME: 1
    TOKEN: KEYWORD  LEXEME: else
    TOKEN: ALPHA  LEXEME: z
    TOKEN: OPERATOR  LEXEME: =
    TOKEN: INTCONST  LEXEME: 10
    TOKEN: OPERATOR  LEXEME: ;
    TOKEN: OPERATOR  LEXEME: }
\end{lcverbatim}

\end{document}
