\RequirePackage[l2tabu, orthodox]{nag}
\documentclass[12pt]{article}

\usepackage{amssymb,amsmath,verbatim,graphicx,microtype,upquote,units,booktabs,siunitx,xcolor,textcomp,hyperref,csquotes,endnotes}

\title{Essay \#3}
\date{Due Date: March 20\textsuperscript{th}, 2017}
\author{Illya Starikov}

\newcommand{\source}[1]{\footnote{\url{#1}}}
\let\footnote=\endnote

\begin{document}
\maketitle

Very often when making a decision, one might hear advice such as the following:

\begin{itemize}
    \item ``Go with your heart''
    \item ``Go with your gut''
    \item ``When you need to make a hard decision, flip a coin. Why? Because
      when that coin is in the air you suddenly know what your hoping for.''
\end{itemize}

This advice is awful; simply because it forces you to rely one thing:
emotion. Emotions should never be considered in decision making. As result,
they should not be used in ethical decision making; simply because emotions are
subjective, emotions are inconsistent, and there are much better systems
and frameworks.

The first problem about including emotions in ethics is
incredible subjectivity that goes into it. People just simply feel a
certain way, yet they can never explain it. This is a biproduct of so
much influence from surroundings (whether it be from people, media, or
simply culture). Whether one believes in nature vs.~nurture, one cannot
deny that certain influences effect the way a person feels.

A famous example of this is the Trolley problem; or, rather, a modified
version of it.

\begin{quote}
    Suppose there is a runaway train heading towards an unenviable
    collisions with five people (who are tied to the tracks). However, there
    is a switch you, the observed, can pull to save the five people at the
    cost of one person.
\end{quote}

When asked this question, roughly 90\% of participants
{\source{http://healthland.time.com/2011/12/05/would-you-kill-one-person-to-save\\-five-new-research-on-a-classic-debate/}} stated
they would pull the lever, holding a utilitarianism viewpoint. However,
when posed with a twist:

\begin{quote}
    The one person you must kill to save five is your child, parent, or
    sibling.
\end{quote}

The number shot down to 33\%, nearly two-thirds less. The best guess is
humans are wired to be emotionally selfish, from an evolutionary
standpoint. It made sense in more primitive, hunter-gathered times; but
this has an impact on us today.

Applying this to business ethics, humans will worry about self-interest
first, and ethical implications will be an afterthought. Take the
case of the
\source{https://philosophia.uncg.edu/phi361-metivier/module-2-why-does-business-need-ethics/case-the-ford-pinto/}{Ford
Pinto},
\source{http://www.investopedia.com/updates/enron-scandal-summary/}{Enron},
or, a more modern scandal,
\source{http://fortune.com/2016/12/28/biggest-corporate-scandals-2016/}{Samsung's
battery recall}. Emotion is rarely based on logic or reasoning, which
ties to the second reason emotions should never be included in ethical
decisions: systematic reasoning.

Emotions are rarely consistent; one might feel a particular way about a
subject today, but an event might change their view on the subject
later. Some great examples of this are religion, patriotism, and
political beliefs. It is quite common to hear of ``Born again
Christians'' or disavowed Christians, those who would die for their
country become less steadfast, or people transitions from conservatism
to a more liberal viewpoint. Those long-held beliefs usually take years
to change; the trouble lies when it is a shorter timespan.

The real danger is emotional inconsistency is not always on a yearly
timescale, but often daily or hourly. A simple ``bad day'' is enough to
make one feel less empathy. One friend's remark might sway your feelings
one way or another. Simple entanglement of other feelings would be
enough to change one's mind about something; all of this is dangerous.

The solutions to all these problems is simple, do not allow emotions into
ethical decisions. There are much better systems to use for ethical
decision making that are well structured, unambitious, and not
subjective.

Emotions are should never be allowed to influence a
decisions in regards to ethics. Emotions are incredibly subjective. They do not always
reflect what one is thinking at that point in time. The
inconsistency of emotions make it difficult to systematize them. In
short: they are \emph{much} better alternatives to using emotions in
decisions making.

\theendnotes
\end{document}
