\RequirePackage[l2tabu, orthodox]{nag}
\documentclass[12pt]{article}

\usepackage{amssymb,amsmath,verbatim,graphicx,microtype,upquote,units,booktabs,siunitx,xcolor,textcomp,hyperref,csquotes,endnotes}

\title{Essay \#2}
\date{Due Date: February 23\textsuperscript{rd}, 2017}
\author{Illya Starikov}

\newcommand{\source}[1]{\footnote{\url{#1}}}
\let\footnote=\endnote

\begin{document}
\maketitle

Utilitarianism is the doctrine that actions are right if they benefit the majority. Examples of this are prevalent everyday, from voting systems to workplace ethics to criminal justice; it would intuitively makes sense that utilitarianism ethics is a great model for ethics. This is not the case. Utilitarianism is a terrible model for ethics because it can give way to arbitrariness, produce poor results, and is simply not pragmatic.

The first problem of utilitarianism is the definition of "benefit". It's easy to give a textbook definition, such as "an advantage or profit gained from something"\source{http://www.dictionary.com/browse/benefit}, coming to the realization that what some people believe to be a benefit can take to be harm. If one is to believe that everybody could benefit from religion, one would be met by wide criticism. If one was to believe that a non-GMO diet would be beneficial, one would be wrong. One could make the case that having cellular devices to keep in contact with loved ones all the time is beneficial, but one could make the case that addiction to constant connectivity is not a good habit to have. These examples are quite simple examples, but they are \textit{prevelant} in our everday life.

Supposing this problem is side-stepped, there is a subsequent problem: although we have established something is "truly" beneficial, the outcome may be less than desirable. Some examples of this might include:

\begin{itemize}
    \item Agreeing on never lying being beneficial , yet telling the truth when it is harmful (i.e. imagine the "Does this dress make me look fat" example scaled to the masses).
    \item The trolly problem. Supposing "living" is purely beneficial, one would be inclined to agree that pulling the lever or pushing the larger person in front of the track is more beneficial than not.
    \item If overall well being is considered most beneficial, euthanasia might have to have serious consideration.
\end{itemize}

These problems are real, and do become a concern when seriously considering utilitarianism.

Last, and although utilitarianism may work well for smaller groups of people, the model completely falls apart for an fractions of the population. The problem of this model is when bringing humanity as a whole into the mix; which, unfortunately, nulls most of the cases. Large groups of people are, in a large part, selfish, unsystematic, arbitrary and short-sighted. Not only is decisions making often misguided and irrational (i.e. voting to leave the European Union before knowing what it is\source{http://fortune.com/2016/06/24/brexit-google-trends/}), it is not always based on logic or reasoning. The idealized model may work fine, but would never get mass approval.

As stated in the thesis, utilitarianism makes a good heuristic for decision making, it does not live past that (that it, being a heuristic). A working definition of being truly "beneficial" is impossible to put into words, being purely "beneficial" might not produce fruitful results, and masses of people would never abide utilitarianism. It makes for a great thought experiment, but the pragmatism simply isn't there.

\theendnotes
\end{document}
