\RequirePackage[l2tabu, orthodox]{nag}
\documentclass[12pt]{article}

\usepackage{amssymb,amsmath,verbatim,graphicx,microtype,upquote,units,booktabs,siunitx,xcolor,textcomp,hyperref,csquotes,endnotes}

\title{Essay \#4}
\date{Due Date: April 21\textsuperscript{st}, 2017}
\author{Illya Starikov}

\newcommand{\source}[1]{\footnote{\url{#1}}}
\let\footnote=\endnote

\begin{document}
\maketitle

In summary, the Ultimatum game is thus:

\begin{quote}
    Suppose you are approached by host of a game, name him Joel, and asks if you would like to play. Reluctantly, you agree. Joel hands you \$100 in \$1 bills and says that you are to make an offer to another player, who has just joined both of you. You keep what you don't offer if and only if the person accepts your offer; otherwise, you both go home empty-handed.
\end{quote}

The question is:

\begin{itemize}
    \item What would you offer, from a \textit{rational} point of view? Why?
    \item What would you offer, from a \textit{moral} point of view? Why?
    \item If different, why different?
\end{itemize}

From a \textit{rational} point of view, the obvious answers would typically be in fractions/wholes of the money: give away 50\$, 25\$, or \$0. However, from a personal standpoint, I would take a much more radical standpoint: give away \$100 with the condition that the other person knows \textbf{that is all the money to give}.

By giving the full \$100, it may appear I have side-stepped the argument. No negotiation is made, the game is effectively irrelevant to me --- I have instantaneously gained and lost \$100, as if I never had it from the start. However, this is not taking into consideration the other participant.

To the other participant, I appear giving and caring (whether those are self-reflective traits is up to you), even though \textbf{I have done nothing}. By acting as a middle man in this transaction, I have swooned the third party. Now, this is not a random act of kindness or ``returning the karma''. I could pull \$100 out of my bank and hand it to the person next to me as I am writing this paper, and it would never have the same effect as it would in this situation.

In regards to what the offer would be from a \textit{moral} standpoint, I would offer a fair \$50 split. Surely there a more implications from taking advantage of such a way, like:

\begin{itemize}
    \item Utilitarianism would say maximizing the benefit between the people would be the two participants gaining an equal split (unless we were in radically different social classes).
    \item Justice and fairness ethics would say split the money evenly.
    \item The general deceitfulness of the situation (on my part) might raise some ethical concerns.
\end{itemize}

There are only two options for why the numbers are radically different: \textit{I'm immoral} or \textit{pure morality is not a great framework for this situation} --- personally, I believe it a mixture of both. Selfishly, I try to maximize my own benefit for the situation (as most people would). Also, this is a test designed to show the distinct lines of morality and ethics (such as a trolley vs transplant problem).

% \theendnotes
\end{document}
