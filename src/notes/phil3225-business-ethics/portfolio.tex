\RequirePackage[l2tabu, orthodox]{nag}
\documentclass[12pt]{article}

\usepackage{amssymb,amsmath,verbatim,graphicx,microtype,upquote,units,booktabs,siunitx,xcolor,hyperref}

\title{Business Portfolio}
\date{Due Date: May 10\textsuperscript{th}, 2017}
\author{Illya Starikov}

\begin{document}
\maketitle


Word count: \textbf{1,339 words}.

\section{Nature of Your Company}\label{component-1-nature-of-your-company}

Currently, in the United States, one of the most controversial topics is self-driving cars. Google has notoriously been famous in the field for producing one of the earliest --- and most famous ---~self-driving cars. This paved the way for Tesla to take said cars into the mainstream, with their \href{https://www.tesla.com/autopilot}{Autopilot} functionality. This made other companies take notice.

Since Tesla has shipped Autopilot in October of 2015, BMW, Mercedes Benz, Ford, and others have announced plans to create a fully-autonomous self-driving car. And why would a company not join the initiative? At the time of this writing, there are there are approximately \href{https://www-fars.nhtsa.dot.gov/Main/index.aspx}{30,000} deaths in the United States \emph{every year} caused by drivers. And these are not accidents where neither driver is to blame --- most of these accidents \textbf{are human error}. By joining the fight to create a self-driving car, I would actually be fighting to save human lives --- not just a few lives, but \emph{saving millions of lives}.

Sidestepping the fight for humanity, my self-driving car business \emph{will} become indispensable to humanity --- as someone who has rode in a self-driving car, I can vouch for the utility. Getting over the initial hump that one is no longer in control (usually taking about 5 minutes), sitting back and \emph{actually enjoying} driving is incomparable. Aside from the relaxation, my self-driving car company would provide much more utility, such as:

\begin{itemize}
    \item Reduce traffic by preventing accidents and being able to know when to \href{https://www.youtube.com/watch?v=iHzzSao6ypE}{accelerate and decelerate} in congested traffic
    \item Do mundane errands, like pick up groceries, people, etc.
    \item Virtually have no need for car insurance, saving the consumer hundreds-thousands of dollars a year
\end{itemize}

For reasons listed above, marketing the product is incredibly easily --~so much so, advertising will not be necessary. Taking the Tesla approach (\href{http://insideevs.com/teslas-advertisement-cost-just-6-per-car/}{not spending at all on advertising}) would actually be a better strategy for my company. At this point in the self-driving car life cycle, advertising would only cause a loss in focus and add additional expense to software. The same philosophy would initially apply to stakeholders. The self-driving car business would have too much politics for outside influence, so it would be ideal to be a private company with just initial funding from angel investors.

However, in a juxtaposition with that, everyone would \emph{technically} have a stake in the company. People will come to rely on this technology; however, it \emph{is dangerous}. A 2-ton machine made of aluminum going over 80 kph is nothing to scoff at, \textbf{it will kill you}. Because so many lives are at stake, it's imperative to make sure the algorithm works perfectly.

When considering my personal business, I would solely work on the software of the self driving car. Disconnecting from the hardware (except for the camera sensors) allows for actual focus on the brain (i.e., the algorithm). By abstracting the actual mechanics of the car, it would allow the company to focus on the major factors that matter. This would include:

\begin{itemize}
    \item The specialized artificial intelligence. This would actually determine when to break, where to turn, how far ahead to stop, what to actually do.
    \item The visualization and camera sensors. This would mean taking in live video and able to analyze what is happening.
    \item The quality assurance. Again, there will be people at risk. It \emph{has to be perfect}, so a dedicated team to make sure the code is flawless would be necessary.
\end{itemize}

Taking these into consideration, the team size would have to be \emph{large} -- a guess would be at least 250-500. It's better to think the size of the teams as percentages. Roughly 70\% of the team would be solely dedicated to engineering the algorithm, 10\% for the cameras and sensors, and the 20\% leftover would work on quality assurance. The majority 80\% would simply be programmers, each with dedicated skills in

\begin{itemize}
    \item Artificial Intelligence and Machine Learning
    \item Camera Input/Outputs
    \item Code Architecture
\end{itemize}

\section{Moral Dilemmas}\label{component-2-moral-dilemmas}

When considering the company, three major moral dilemmas arise:

\begin{enumerate}
    \item The Trolly-esque problem
    \item The ``I am taking millions of jobs'' problem
    \item The luddite problem
\end{enumerate}

\subsection{The Trolly-esque}\label{the-trolly-esque}

Consider the trolly problem

\begin{quote}
    Suppose you are standing on train tracks. You notice on one side of the tracks, there is an oncoming trolley. On the other side, there is a fork in the tracks. On one side, there is a person tied to the tracks; on the other, five people are tied. The tracks are oriented towards the five people; however, there is a lever to divert the tracks. Do you save five at the expense of one?
\end{quote}

This exactly applies to self driving cars. If one is driving, there might come a situation where there is a brake failure, and the artificial intelligence in the car might have to either kill the one passenger or the five pedestrians. This will be at the hands of the programmer to decide.

Is it the responsibility of the car to protect the driver, or take a utilitarian point and kill the driver for the five pedestrians? These are decisions I, the CEO, will have to make on behalf of the self driving cars and my company.

\subsection{\texorpdfstring{The ``I Am Taking Millions of Jobs'' Problem}{The I Am Taking Millions of Jobs Problem}}\label{the-i-am-taking-millions-of-jobs-problem}

It is no secret that self driving cars will take over \emph{millions} of jobs; this involves many chauffeurs, taxi drivers, truckers, and more.

At the hand of my company, people who have to support their families will be out of a job because of my company.

\subsection{The Luddite Problem}\label{the-luddite-problem}

As they did with the cotton mill, \href{https://en.wikipedia.org/wiki/Luddite}{luddites} will definitely try to prevent self-driving car wave. Unfortunately, unlike the era of the cotton mill, many luddites will also take the form of congressmen; furthermore, not only will there be a lot of opposition from the house and senate, but other countries as well. Politically, there will be a huge barrier for me to overcome.

\section{Moral Solutions}\label{component-2-moral-solutions}

Below I will list the solutions to the above problem

\subsection{The Trolly-esque Solution}\label{the-trolly-esque-solution}

I always hold the philosophy that the simplest solution is \emph{generally} the best one. And that is a philosophy I will apply here. In reality, this situation will arise so little of the time (in a electric car, with about 10 moving parts, break failure is very uncommon), that I would not even program the car to do anything explicitly. Programming the car to protect the driver adds unnecessary code. The car will do a calculation of all methods of course correction, and whatever is the most optimal, I will force it to chose.

However, \textbf{I would never make this known}. When asked how my car handles this problem, I will never comment. If I say the car chooses the driver, then the argument could be made that I choose one life over five. If I say it chooses the five people, people would be scared to use the car. It is a lose-lose situation.

\subsection{\texorpdfstring{The ``I Am Taking Millions of Jobs'' Solution}{The I Am Taking Millions of Jobs Solution}}\label{the-i-am-taking-millions-of-jobs-solution}

There is no \emph{real} good solution to this. This is an intrinsic problem of the self driving car business. Although I would creating a couple hundred jobs, it will take millions from hard working people. I would have to hope the government could create a transition system, and possibly lobby for it. But as just what my company could do, there is next to nothing.

\subsection{The Luddite Problem}\label{the-luddite-problem-1}

This is also an intrinsic problem of the company, however more easily fixed. The choice is obvious: jobs are replaceable, human life is not. So self driving cars \emph{should} be accepted by the government.

The best solution would be to create a coalition with other companies to fight the government's stance on self driving cars. I feel it is a moral duty, not just in a Kantian and a Utilitarian context, but in a \emph{these save lives} context. As someone who has almost lost his life and his mother's life in a car wreck, I feel it is my duty to create market adoption.

\end{document}
