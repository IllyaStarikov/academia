\RequirePackage[l2tabu, orthodox]{nag}
\documentclass[12pt]{article}

\usepackage{amssymb,amsmath,verbatim,graphicx,microtype,upquote,units,booktabs,siunitx,xcolor,textcomp,hyperref,csquotes,endnotes}

\title{Essay \#2}
\date{Due Date: May 10\textsuperscript{th}, 2017}
\author{Illya Starikov}

\newcommand{\source}[1]{\footnote{\url{#1}}}
\let\footnote=\endnote

\begin{document}
\maketitle

Utilitarianism is the doctrine that actions are right if they benefit the majority. Examples of this are prevalent everyday, from voting systems to workplace ethics to criminal justice; it would intuitively makes sense that utilitarianism ethics is a great model for ethics. This is not always the case. Utilitarianism is a terrible model for ethics because it can give way to arbitrariness (even from the definition), can produce poor results, and is simply not pragmatic.

The first problem of utilitarianism is the definition of "benefit". It's easy to give a textbook definition, such as "an advantage or profit gained from something"\source{http://www.dictionary.com/browse/benefit}, coming to the realization that what some people believe to be a benefit can take to be harm. If one is to believe that everybody could benefit from religion, one would be met by wide criticism. If one was to believe that a non-GMO diet would be beneficial, one would be wrong. One could make the case that having cellular devices to keep in contact with loved ones all the time is beneficial, but one could make the case that addiction to constant connectivity is not a good habit to have. These examples are quite simple, but they are \textit{prevelant} in our everyday life. The task of defining what is beneficial is borderline impossible.

The next impossible task is coming to a universal definition of how to distribute the benefits. An oligarchy approach would be to give the people with the most benefit more benefit. Without getting too political, this has some problems in itself. This system is proven to produce incredible inequality. There is a hypothetical trickle down effect, where the top would redistribute the benefit to people with less --- we know this to not always be the case.

A different approach could be maximizing equality --- everyone gets to be on a more level playing field. This would typically mean providing benefit to the lower class before to the higher class. This would attempt to put everyone at a more level playing field.

The last option is to simply distribute equally. If there are $p$ people and $b$ benefit, everyone gets $\frac{b}{p}$ however it shares some common qualities to communism, a not-particularly favored form of government. This appears to be the most fair system, but its relation to a particular form of government will always make someone not like it.

An alternative solution would a democratic vote to find the superior system. The voting system holds its own problems, but its the most reasonable solution.

Supposing these impossible problems are side-stepped, there is a subsequent problem: although we have established something is "truly" beneficial, the outcome may be less than desirable. Some examples of this might include:

\begin{itemize}
    \item Agreeing on never lying being beneficial, yet telling the truth when it is harmful (i.e. imagine the "Does this dress make me look fat" example scaled to the masses).
    \item The trolly problem. Supposing "living" is purely beneficial, one would be inclined to agree that pulling the lever or pushing the larger person in front of the track is more beneficial than not.
    \item If overall well being is considered most beneficial, euthanasia might have to have serious consideration.
\end{itemize}

These problems are real, and do become a concern when seriously considering utilitarianism.

Last, and although utilitarianism may work well for smaller groups of people, the model completely falls apart for an fractions of the population. The problem of this model is when bringing humanity as a whole into the mix; which, unfortunately, nulls most of the cases. Large groups of people are, in a large part, selfish, unsystematic, arbitrary and short-sighted. Not only is decisions making often misguided and irrational (i.e. voting to leave the European Union before knowing what it is\source{http://fortune.com/2016/06/24/brexit-google-trends/}), it is not always based on logic or reasoning. The idealized model may work fine, but would never get mass approval.

There are several solutions to this, one of which is to have a representative for your views. This has intrinsic problem of your representative not always having your interest at heart, but its a more feasible solution. If the representative always has the \textit{majority} in mind, then it is already a better system than having a democracy of the people.

Another alternative would be to have an idealistic ``perfect'' person make a decision for the majority. Assuming this person always has the majority's interest at heart, then this system could hold. However, it closely resembles a dictatorship, a seemingly ostracized form of government.

As stated in the thesis, utilitarianism makes a good heuristic for decision making, it does not live past that (that it, being a heuristic). The definition of utilitarianism is much too vague, and gives way to too much ambiguity when actually using it. Two of the ambiguities is defining what is benefit, and how to distribute it. If this problem was side-stepped, it is not always clear whether one would like the benefit. Lastly, the problem does not scale properly. Larger than small groups of people create massive problems. There are two solutions to this: a republic and a dictatorship.

\theendnotes
\end{document}
