\documentclass[12pt,letterpaper]{article}
\usepackage{ifpdf}
\usepackage{mla}
\begin{document}
\begin{mla}{Illya}{Starikov}{Professor Amanda Burch}{General Psychology}{\today}{Sleep States and Memory Processes In Humans: Procedural Versus Declarative Memory Systems}

``Sleep on it''. This single clich\'e has permeated society for quite a long time. Although when applied strictly to the context it is usually said (typically when faced with a difficult decision), it does not have merit. Sleep does not help one make a decision no more than a physics textbook helps make you a theoretical physicist; however, just as the book can be a \textit{tool} to make one a physicist, sleep can be a \textit{tool} used in the decision making process. We will try to unravel this in this paper.

The relationship for finding correlation between sleep and learning/memory had been, for a long time, something like this:

\begin{quotation}
\noindent ``There was substantial evidence from animal studies to support the idea that post-training REM sleep was important for memory consolidation. The human studies, however, provided mixed results. The human studies cited as showing no relationship between sleep and memory utilized simple memorization tasks.''
\end{quotation}

\noindent These findings would later come to be proven false. The main reason for the discrepancies came from a naive understanding of how memory worked:

\begin{quotation}
\noindent ``These included a very short registration phase taking less than 1 s to occur followed by a period (minutes to hours) after training, during which learned material existed as relatively labile short-term memory that was vulnerable to various agents such as strong drugs or electroconvulsive therapy (ECT). As the minutes turned to hours and memory processing continued, the learned material was considered less vulnerable to disruption. After 24 h (or even less), memory formation was considered complete and, thus, no longer vulnerable to disruption. At this point a stable, permanent long-term memory was considered to have been formed. This process has been called consolidation and most workers still use the concept of a permanent “memory trace” existing in the brain''
\end{quotation}

This poses a problem when testing two different types of learning: ``declarative'' and ``procedural'' knowledge. Declarative knowledge ``refers to memories accessible to conscious recollection'', while procedural ``are memories of how to do some skill or how to solve a problem''. This is the discrepancy: psychological studies have always assumed that memory was a single, colossal process; however, we know this not to be the case.

There have been four types of studies performed to study the relationship between sleep and memory: ``One approach has been to train subjects on some task and then to record the changes in sleep parameters'', ``deprive the subject of total sleep or of some specific phase or aspect of sleep following task acquisition in order to induce possible memory impairments'',, and finally ``A third approach has been to have the subject learn the task either just before bed or during the night at some particular point''. These four types have been carried out in regards to declarative material, and the results were, as aforementioned, strongly positive. It had been clearly shown that ``after learning a PA [paired associates] task showed superior memory to participants that stayed awake for an equal amount of time''. The research did not stop there --- further research showed that total and even \textit{selective} sleep deprivation was harmful; so much so that you could not tell the selectively sleep deprived and totally sleep deprived apart.

Switching to procedural, patients that underwent study for procedural knowledge were ``to engage in cognitive activity that was more complex than simple recall or recognition of previously memorized material. These studies demonstrated a relationship between REM sleep and memory.'' Again, the studies were overwhelming positive:

\begin{quotation}
\noindent ``Mandai$\ldots$ found an increase in REM sleep duration and number of REM episodes in subjects who underwent 90min training in translation of Morse code. They also reported a high correlation between retention, number of REMs and REM density.''
\end{quotation}

\noindent And this was not limited to just partial sleep deprivation, but also to total and selective sleep deprivation as well. Those who experienced any sleep loss with 48-72 hours also were susceptible to memory loss related to procedural tasks. At this point, there is no denying the relationship between sleep and memory recall; as stated, ``The level of recall was positively correlated with the average duration of the NREM-REM sleep cycles''.

This all comes back to the ``Sleep on it'' statement made previously --- although you cannot use sleep a sole deducer for your choices; sleep can make you more effectively a decision maker by allowing you to more quickly recall declarative and procedural knowledge that might be relevant to the question.

I do not think this article is that controversial; humans needs sleep to recharge, reboot, and restore our energy. The article also did not fall into the trap of a ``catchall'' for sleep, stating that a person needs x-hours of sleep to full function. I also agree with the statements being made; it is quite intuitive that a person will need sleep to recall declarative and procedural knowledge.


\end{mla}
\end{document}