\documentclass[12pt,letterpaper]{article}
\usepackage{ifpdf}
\usepackage{mla}
\begin{document}
\begin{mla}{Illya}{Starikov}{Professor Amanda Burch}{General Psychology}{\today}{Quantum Cognition: A New Theoretical Approach to Psychology}

For decades, judgments and human cognition has been precedented on two different concepts --- heuristic and rational. The heuristic approach ``posits that, to make judgments and decisions, people tend to employ simple heuristics (e.g., representativeness, anchoring-and-adjustment, take-the-best)'', while the rational approach ``posits that people can derive inferences from the Bayes rule and decisions from the expected utility rule in a rational manner''. These two methods have been proven an intuitive model; however, a third approach has emerged: quantum cognition. This framework for cognition assumes that humans abide basic axioms from probability theory and quantum physics. However, these axioms can be proven to be less intuitive than their former counterparts (heuristic and rational).

The theory is quite younger than quantum physics (quantum cognition being 20 years while quantum physics being over 100 years old), but that time discrepancy has given researches the ability to draw ``inspiration from both the mathematical structure of quantum theory and its associated dynamic principles.''

As for how quantum physics relates to cognition, less us first look at an example of the disparity between the logic of quantum physics and the logic of boolean algebra. The statement $A$ and $B$ (more formally, $A \cap B$ denotes $A$ and $B$, and $P(A \cap B)$ denotes \textit{the probability} of $A$ and $B$) in classical boolean algebra has the commutative property, meaning one can write $P(A \cap B)$ or $P(B \cap A)$ and there would be no difference. This is not the case in quantum physics, nor is it in cognition. In quantum physics, the ordering does matter; as it does for human cognition. Occasionally, the $\cup$ can be replaced with \textit{before} (i.e. $A$ \textit{before} $B$).

To further clarify this, let us take two examples. First, suppose a student (name him Illya) turns in two papers, paper $A$ and paper $B$. In paper $A$, he does an excellent job, citing the paper, pulling examples from the paper, etc. The second paper, Illya does well, but not nearly to the level of excellence as the first. In this case, the order matters: if his first paper (paper $A$) was extraordinary, there will be that bias for the second paper (paper $B$) to be done. In this instance, the probability of --- $P(\cdots)$ --- doing well is not commutative; $P(A \cup B) \neq P(B \cup A)$, the probability of doing well on paper $A$ and paper $B$ are not quite the same as doing well on paper $B$ and $A$. However, this is not always the case; for example we can contemplate whether to fail Illya on the second paper while we contemplate what is for dinner, as we often might do. This provides a simple introduction to the principle of complementary, which ``leads directly to the best-known principle of quantum theory --- the uncertainty principle.''

``The [uncertainty] principle holds that when we are certain about a quantum particle’s position, we are necessarily uncertain about its momentum, and vice versa.'' This is important in the context of incompatible events (events that can't be considered at the same time, such as Illya's papers). When one has two events dependent on the order, they can't think of them, therefore they can't think of them at the same time; just like the uncertainty of a particle's position and momentum.

Now that we have this background knowledge, we shall consider another quantum physics approach to understanding the brain itself: vector spaces. A vector space, in mathematics, is a space of vectors; and a vector is an object with direction and magnitude. This might seem abstract, but ``it is analogous to the distributed input across nodes in a connectionist neural network''. With this vector space, and a projector (something that maps a space to a subspace), this can ``provides algorithm-level predictions for this more complex neural implementation.''

However, these are all just new applications of quantum theory to cognition --- we can ``highlight the expressive power of quantum models by pointing out that they have already been applied to a broad range of cognitive phenomena, including perception, memory, conceptual combinations, attitudes, probability judgments, causal reasoning, decision making, and strategic games''. This is still a maturing field, as we still do not quite fully understand all of quantum physics, but applications are being found and used everyday.

In my personal opinion, I believe the article is controversial. If psychology does a paradigm shift where quantum physics is the framework to understand human cognition, this particular field might not be quite as fruitful. Can you imagine a third-year psychology major walking into class and seeing $-\frac{\hbar^2}{2m} \frac{d^2 \psi}{dx^2} + V\psi = E\psi$ (time--independent Schr\"odinger's equation) for the first time? Quantum cognition is based on quantum mechanics, and quantum mechanics is based on heavy math (differential equations, complex linear algebra, multi-dimensional calculus) and every kind of physics imaginable (Newtonian mechanics, electromagnetism, theoretical, etc.). Even taking the most primitive example of physics in this article, the uncertainty principle is understood through analogy, not physics. The moving parts of this problem are: the principle itself ($\sigma _x \sigma _p \geq \frac{\hbar}{2}$), momentum ($\vec{p} = m \vec{v}$), and position (usually denoted by the vector $\vec{r}$). These are hard, mathematical and physics problems to understand. So either this framework of cognition will be left to the few that are also studying higher-level physics or be left out of the curriculum outright.

I do agree with the article. It shows the uncertainty and complexity that the human mind posses, and how it can be described through such complex means. Although I am skeptical if quantum physics is truly the best framework for understanding the human mind, I think it holds as the biggest contendor at this point in time.

\end{mla}
\end{document}