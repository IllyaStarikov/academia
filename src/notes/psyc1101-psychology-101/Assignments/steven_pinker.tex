\documentclass[12pt,letterpaper]{article}
\usepackage{ifpdf}
\usepackage{mla}

\usepackage{xcolor}
\newcommand{\todo}[1]{\textbf{\colorbox{red!50}{cite}}{\footnote{#1}}}

\newenvironment{mlatitlepage}[5]{\linespread{1}\small\normalsize\vspace*{3in} \centering #5 \\ \vspace{4in} \centering {#1} \\ #2 \\ #3 \\ #4}{\newpage}

\begin{document}

\begin{mlatitlepage}{Illya Starikov}{Professor Amanda Burch}{General Psychology}{\today}{Steven Pinker: The Language of Cognition}
\end{mlatitlepage}

\begin{mla}{Illya}{Starikov}{Professor Amanda Burch}{General Psychology}{\today}{Steven Pinker: The Language of Cognition}
Steven Pinker is one of the most influential psychologists of our time. Aside from being named Time's 100 most influential people in 2004 and winning Foreign Policy's 100 top public intellectuals in 2004 and 2008, he has won awards from the National Academy of Sciences (which he was elected to in 2016), the Royal Institution, American Psychological Association, the American Humanist Association, and the Cognitive Neuroscience Society. Although his work is primarily in linguistics and its relation to psychology, he has also does work work in cognitive science and visual cognition. From his research he has published ten books; some which include \textit{The Language Instinct}, \textit{How the Mind Works}, \textit{The Blank Slate}, \textit{The Stuff of Thought} and, more recently, \textit{The Better Angels of Our Nature}. Before diving into his formal research, it might be more helpful to look at his earlier years and their influence on his later life.

Steven Pinker was born in Montreal, Quebec, in 1954 to a Jewish family. Although his parents were devoted Jewish, he proudly renounced the faith, a sentiment that is found to this day:

\begin{quotation}
``“The Bible is a manual for rape, genocide, and the destruction of families...Religion has given us stonings, witch burnings, crusades, Inquisitions, jihads, fatwas, suicide bombers...and mothers who drown their children in the river,” he said.''
\end{quotation}

\noindent This rejection of religion ultimately lead him down a life of science; specifically, psychology.

After graduating with a bachelor's degree from McGill University in 1976, he later finished his doctorate only three years later from Harvard. Before finally settling as the Johnstone Family Professor in the Department of Psychology at Harvard, where he currently works, he bounced around between Stanford, MIT, University of California, Santa Barbara and Harvard. The attendance of these prestigious colleges was not without merit, of course; Steven Pinker had accomplished quite a bit at this time.

Pinker's most notable work is not just the work of the mind, but of his enthusiasm for language. This fascination came to him in graduate school, which he decided to pursue in his research. Roughly during this time, he had published two books. His first book, \textit{Language Learnability and Language Development}, was regarded as ``A fiercely reasoned, bently written landmark of psychological science.''. As hinted by the name, the book sheds light on the problem of language learning; in particular, language acquisition in children. In the book, he gives a brief summary on the matter:

\begin{quotation}
``The core assumption of the theory is that children are innately equipped with algorithms designed to acquire the grammatical rules and lexical entries of a human language. The algorithms are triggered at first by the meanings of the words in the input sen- tences and knowledge of what their referents are doing, gleaned from the context. Their outputs, the first grammatical rules, are used to help analyze subsequent inputs and to trigger other learning algorithms, which come in sets tailored to the major components of language. Empirically, children do seem to show the rapid but piecemeal acquisition of adultlike rules predicted by the theory, with occasional systematic errors that betray partly acquired rules.'' (Pinker, 2)
\end{quotation}

\noindent As such, he has spent many years of his later years still studying the lexical structure of language. However, he had another passion, which he touched on in his second book, \textit{Visual Cognition: Computational Models of Cognition and Perception}.

\textit{Visual Cognition} addresses some of the more difficult question of human visualization, such as:

\begin{quotation}
``How do we recognize objects? How do we reason about objects when they are absent and only in memory? How do we conceptualize the three dimensions of space? Do different people do these things in different ways? And where are these abilities located in the brain?''
\end{quotation}

\noindent Throughout the remainder of the book, Pinker goes about answer these questions through past studies, modeling of cognitive processes, and new experimental techniques.

Although his main claim to fame was progressive studies in these two fields, visual cognition and language, he has done additional studies regarding violence, written additional books (regarding writing, violence, humanitarian efforts, and the human mind), advocated for science based thinking, participated in public debates, and occasionally done interviews with notable online celebrities.

In general, I do agree with Steven Pinker. His work for language, in my limited exposure to it, has always been intuitive and downright spot on. In one of his Ted Talks, \textit{What our language habits reveal}, a particular statement stands out:

\begin{quote}
I think the key idea is that language is a way of negotiating relationships, and human relationships fall into a number of types. There's an influential taxonomy by the anthropologist Alan Fiske, in which relationships can be categorized, more or less, into communality, which works on the principle ``what's mine is thine, what's thine is mine,'' the kind of mindset that operates within a family, for example; dominance, whose principle is ``don't mess with me;'' reciprocity, ``you scratch my back, I'll scratch yours;'' and sexuality, in the immortal words of Cole Porter, ``Let's do it.''\
\end{quote}

\noindent This gives a more modest expression of language --- instead of an overly-romanticized art, language can be a negotiation, among many things. And Pinker continues through the talk discussing these strictly-scientific and psychological definitions of language and their effects on our daily lives.

Although I do agree with his work, I do not particularly see his work around me; even if it is ever-present. Language does give a direct method of communication that I may not particularly give attention to, but it does affect my daily life.

\begin{workscited}
\bibent

Pinker, Steven A. ``About.'' Steven Pinker. Harvard, n.d. Web. 28 June 2016.

Pinker, Steven A. ``Language Learnability and Language Development.'' Steven Pinker. Steven Pinker, 17 Aug. 2015. Web. 28 June 2016.

Pinker, Steven A. ``Visual Cognition: Computational Models of Cognition and Perception.'' Steven Pinker. Harvard, 08 Aug. 2015. Web. 28 June 2016.

Pinker, Steven. Language Learnability and Language Development. Cambridge, MA: Harvard UP, 1984. Print.

Pinker, Steven. ``Transcript of ``What Our Language Habits Reveal'''' Steven Pinker: What Our Language Habits Reveal. TED, Sept. 2007. Web. 28 June 2016.
\end{workscited}

\end{mla}
\end{document}