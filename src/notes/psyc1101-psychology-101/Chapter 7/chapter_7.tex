\RequirePackage[l2tabu, orthodox]{nag}
\documentclass{article}

\newcounter{chapter}
\setcounter{chapter}{7} % Modify Counter To Chapter

\setcounter{section}{\value{chapter}}
\addtocounter{section}{-1}

\usepackage{amssymb,amsmath,verbatim,graphicx,microtype,units,booktabs}
\usepackage[margin=10pt, font=small, labelfont=bf, labelsep=endash]{caption}
\usepackage[colorlinks=true, pdfborder={0 0 0}]{hyperref}
\usepackage[utf8]{inputenc}
\usepackage{pdfpages}

\usepackage[left=0.75in, right=0.75in]{geometry}
\usepackage{titleps}
\newpagestyle{main}{
    \setheadrule{.4pt}
    \sethead{Chapter \thechapter: \sectiontitle}
            {}
            {Illya Starikov}
}
\pagestyle{main}

\begin{document}
\section{Cognition And Intelligence}

\subsection{Class Notes}
\begin{itemize}
    \item
\end{itemize}

\subsection{Study Guide Material}
\begin{itemize}
    \item Elaboration is a process by which a stimulus is linked to other information at the time of encoding.
    \item Visual imagery involves the creation of visual images to represent the words to be remembered, and concrete words are much easier to create images of
    \item Dual-coding theory holds that memory is enhanced by forming semantic or visual codes, since either can lead to recall.
    \item Self-referent encoding involves deciding how or whether information is personally relevant, that is, information that is personally meaningful is more memorable.
    \item Structural encoding is shallow processing that emphasizes the physical structure of the stimulus, as in encoding the shapes that form letters in the alphabet.
    \item Sensory memory is one of two temporary storage buffers that information must pass through before reaching long-term storage. As the name implies, sensory memory preserves information through the senses, in its original form
    \item flashbulb memories, which are unusually vivid and detailed recollections of momentous events
    \item Schemas are organized clusters of knowledge about an object or event abstracted from previous experience
    \item Our poor abilities to retrieve information accurately has been extensively studied and is now known as the misinformation effect,
    \item Another explanation for why we sometimes fail to retrieve memories accurately is due to source monitoring
    \item Interference theory, which proposes that people forget information because of competition from other material, is a well-documented process and can account at least for some of our forgetting.
    \item We repress memories or we are motivated to forget
    \item Retrograde amnesia results in loss of memories for events that occurred prior to the injury whereas anterograde amnesia results in loss of memories for events that occur after the injury.
    \item Consolidation refers to a hypothetical process involving gradual conversion of information into durable memory codes for storage in long-term memory. These areas are those around the hippocampus, which comprise the medial temporal lobe.
    \item Declarative memory handles factual information, whereas nondeclarative memory houses memory for actions, skills, conditioned responses, and emotional memories.
    \item Episodic memory is an autobiography whereas semantic memory is an encyclopedia.
    \item Prospective memory involves remembering to perform actions in the future.
    \item Retrospective memories involve remembering events from the past or previously learned information
\end{itemize}

\subsection{Book Notes}

\end{document}