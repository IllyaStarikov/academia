\RequirePackage[l2tabu, orthodox]{nag}
\documentclass[12pt]{article}

\usepackage{amssymb,amsmath,verbatim,graphicx,microtype,units,booktabs}
\usepackage[margin=10pt, font=small, labelfont=bf, labelsep=endash]{caption}
\usepackage[colorlinks=true, pdfborder={0 0 0}]{hyperref}
\usepackage[utf8]{inputenc}

\title{A Beautiful Mind}
\date{\today}
\author{Illya Alexsandrovich Starikov}

\begin{document}
\maketitle

\section{The Role of Psychology}
The move is centered around John Nash, who, as find out later in the story, has schizophrenia. In the remainder of this movie we are taken through Nash's life as he copes with his hallucinations and illusions.

\section{Main Character}
John Nash experiences an aggressive case of schizophrenia. His extraordinary intelligence (aside from his contribution to economics, he also worked on complex manifolds and the Riemann hypothesis, two very difficult areas of mathematics) causes him to experience very vivid episodes of hallucinations. A possible reasoning for the different characters is a coping mechanism --- the roommate for his loneliness, the child for his fear of being a new father, the cryptography for having to do something great.

\section{Psychological Treatment}
John Nash first received regular treatments of \textit{Insulin shock therapy}, in which he was repeatedly injected with insulin to induce comas. Because this is not a sustainable form of treatment outside of medical facilities, he was then given a regular prescription of an anti-psychotic. However, the side effects were too detrimental to Nash's mathematical work and his love life, he decides to mentally confront it. Instead of blocking it out, he decides to accept his hallucinations.

\end{document}