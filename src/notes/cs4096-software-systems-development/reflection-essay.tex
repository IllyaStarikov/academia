%
%  reflection-essay.tex
%  CS4096
%
%  Created by Illya Starikov on 12/06/17.
%  Copyright 2017. Illya Starikov. All rights reserved.
%

\RequirePackage[l2tabu, orthodox]{nag}
\documentclass[12pt]{scrartcl}

\usepackage{amssymb,amsmath,verbatim,graphicx,microtype,upquote,units,booktabs,siunitx,xcolor}

\title{Reflection Essay}
\subtitle{Senior Design (Comp Sci 4096)}
\date{Due Date: Wednesday, December 13\textsuperscript{th}}
\author{Illya Starikov}

\begin{document}
\maketitle

Throughout the course of Senior Design, I have learned a lot of useful information that will translate not only to the remainder of school, but to the actual development in the real world. Along the way of creating War Paint, I have learned several lessons that I hope to take with me upon graduating college.

For backstory, War Paint is a real-time, augmented reality game who's primary objective is to traverse as much of a terrain as possible. Specifically,

\begin{enumerate}
    \item The player is boxed in a particular perimeter. Traversing outside said parameter is strictly forbidden (and will not count towards team score).
    \item Players are split into even teams. As a player moves around the boxed perimeter, ``paint'' is placed down (colored with the team's primary color, red or blue).
    \item Paining over another team's paint voids the other team's paint, and contributes only to who's team has the last layer of paint on the field.
    \item At the end of a designated period of time, the team with the most paint on the board wins.
\end{enumerate}

During the semester, I got to work with a team that had to deal with a whole stack. Our particular team had

\begin{itemize}
    \item An Android Team
    \item An iOS Team
    \item A R\&D Team (to focus on potential of hardware)
    \item A Server Team
\end{itemize}

Because my specialty was iOS Development, I took the role of working on the iOS application. Most of the software development process was simple: design the Model View Controllers (MVCs). The MVCs already had a hierarchy to them, so barely any architecture work had to be done.

The takeaway I gained from this class wasn't a particular tool or methodology; it was a particular mindset. Most of my group projects up to this point have placed me in a leadership role of some sorts. Because of my busy schedule this semester, I was not comfortable in that position. Because of this, I had to learn how to coordinate with others, be a team player, and put my faith into a different team lead.

My knowledge of the Software Development Process has not particularly changed up to this point. I have held several internships and have worked on massive projects before; most of my knowledge had been learned outside of the classroom.

The most important lesson I learned was how to not bite off more than I can chew. I had to usually take the minimal amount of work per week, for I had priorities in other classes and my job. This will greatly impact the way I handle work, where a work-life balance will be hard to maintain. I hope to keep this lesson in the back of my mind for future reference.

\end{document}

