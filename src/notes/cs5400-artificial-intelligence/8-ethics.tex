%
%  8-ethics.tex
%  notes
%
%  Created by Illya Starikov on 04/25/18.
%  Copyright 2018. Illya Starikov. All rights reserved.
%

\section{AI Ethics}
For background, some key historical events in AI were as follows:

\begin{description}
    \item[4\textsuperscript{th} Century BC] Aristotle propositional logic.
    \item[1600s] Descartes mind-body connection.
    \item[1805] First programmable machine.
    \item[Mid 1800s] Charles Babbage's ``difference engine'' and ``analytical engine''.
    \item[Sometime] Lady Lovelace's Objection
    \item[1847] George Boole propositional logic
    \item[1879] Gottlob Frege predicate logic
    \item[1931] Kurt Godel: Incompleteness Theorem
        \begin{quote}
            In any language expressive enough to describe natural number properties, there are undecidable (incomputable) true statements
        \end{quote}

    \item[1943] McCulloch \& Pitts: Neural Computation
    \item[1956] Term ``AI'' coined
    \item[1976] Newell \& Simon's ``Physical Symbol System Hypothesis'' A physical symbol system has the necessary and sufficient means for general intelligent action
    \item[1974--80, 1987--93] AI Winters
    \item[1980\textsuperscript{+}] Commercialization of AI
    \item[1986\textsuperscript{+}]  Rebirth of Artificial Neural Networks
    \item[1990s] Unification of Evolutionary Computation
    \item[200\textsuperscript{+}] Rise of Deep Learning
\end{description}

\subsection{Weak Vs. Strong AI Hypothesis}
The Week AI vs. Strong AI Hypothesis is thus:

\begin{description}
    \item[Weak AI Hypothesis] It's possible to create machines that can act as if they're intelligent.
    \item[Strong AI Hypothesis] It's possible to create machines that actually think.
\end{description}

This closely relates to the following problems:

\begin{itemize}
    \item Rene Descartes (1596--1650)
    \item Rationalism, Dualism, Materialism, Brain in a Vat
    \item Star Trek \& Souls
    \item Chinese Room\footnote{The Chinese room argument holds that a program cannot give a computer a ``mind'', ``understanding'' or ``consciousness'', regardless of how intelligently or human-like the program may make the computer behave.}
\end{itemize}

\subsection{Ethics}
Some ethical concerns with regards to AIs are as follows:
\begin{itemize}
    \item Autonomous AIs and the \href{http://www.trolleydilemma.com/}{Trolley Dilemma}
    \item Unemployment and Inequality
    \item Human dependency and obsoleteness
    \item Bias transfer
    \item Security
    \item Human-robot relationships
    \item Rights of sentient beings.
\end{itemize}
