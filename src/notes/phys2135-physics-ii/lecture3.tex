\RequirePackage[l2tabu, orthodox]{nag}
\documentclass[11pt]{article}

\newcommand{\chapterTitle}{Electric Field Lines, Electric Dipoles, Electric Flux, Gauss’ Law}
\newcommand{\lectureNumber}{3}

\usepackage{amssymb,amsmath,verbatim,graphicx,microtype,upquote,units,booktabs,akkwidepage}

\newcommand{\chapterNumber}[1]{
    \setcounter{section}{#1}
    \addtocounter{section}{-1}
}

\begin{document}
\section{\chapterTitle}

\subsection{Book Notes}
\begin{itemize}
    \item The direction of $\mathbf{\vec{p}}$ is from the negative to the positive charge.
\end{itemize}

\subsection{Lecture Notes}
\begin{itemize}
    \item The electric field always depends on $qd$.
    \begin{itemize}
        \item dipole moment vector $\vec{p} = q\vec{d}$
        \item Torque on the dipole is exactly the same as classical mechanics.
    \end{itemize}
    \item Remember, zero potential energy does not mean minimum potential energy!
    \item The \textbf{electric flux} passing through a surface is the number of electric field lines that pass through it.
    \item For a closed surface, dA is normal to the surface and always points away from the inside.
    \item The electric field is a vector field, so a constant electric field is one that does not change with position or time.
    \item If a conductor is in electrostatic equilibrium, any excess charge must lie on its surface, so for the charge to be uniformly distributed throughout the volume, the object must be an insulator.
\end{itemize}

\subsection{Recitation}
\begin{itemize}
    \item Electric field lines just give an easy way to imagine what the force would be.
    \item The distance $\vec{\mathbf{d}}$ points from the negative to the positive.
    \begin{itemize}
        \item That where the dipole moment comes from.
    \end{itemize}

    \item $U = - \vec{p} \vec{E}$
    \item $\phi _E = \oint \vec{E} \cdot \vec{dA}$
    \begin{itemize}
        \item Area vector points outward.
    \end{itemize}
\end{itemize}

\begin{align}
    \phi _E &= \oint \vec{E} \cdot \vec{dA}\\
    &= \oint E \ da \cos \theta && \text{Cross product expansions} \\
    &= E \oint dA && \text{If E is constant, pull it out} \\
    &= E \times \text{Surface Area} \\
\end{align}

\end{document}