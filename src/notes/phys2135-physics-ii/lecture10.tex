\RequirePackage[l2tabu, orthodox]{nag}
\documentclass[11pt]{article}

\newcommand{\chapterTitle}{Electric Current; Current Density; Resistance}
\newcommand{\lectureNumber}{10}

\usepackage{amssymb,amsmath,verbatim,graphicx,microtype,upquote,units,booktabs,akkwidepage}

\newcommand{\chapterNumber}[1]{
    \setcounter{section}{#1}
    \addtocounter{section}{-1}
}

\begin{document}
\section{\chapterTitle}

\subsection{Book Notes}
\begin{itemize}
    \item An electric current consists of charges in motion from one region to another. If the charges follow a conducting path that forms a closed loop, the path is called an electric circuit.
    \item In electrostatic situations (discussed in Chapters 21 through 24) the electric field is zero everywhere within the conductor, and there is no current. However, this does not mean that all charges within the conductor are at rest.
    \item Although we refer to the direction of a current, current as defined by Eq. (25.1) is not a vector quantity. In a current-carrying wire, the current is always along the length of the wire, regardless of whether the wire is straight or curved. No single vector could describe motion along a curved path. We'll usually describe the direction of current either in words (as in “the current flows clockwise around the circuit”) or by choosing a current to be positive if it flows in one direction along a conductor and negative if it flows in the other direction.
    \item Current density $\vec{J}$ is a vector, but current $I$ i snot --- the difference is that the current density $\vec{J}$ describes how charges flow at a certain point, and the vector's direction tells you about the direction of the flow at that point.
    \item The greater the resistivity ($\rho$), the greater the field needed to cause a given current density, or the smaller the current density caused by a given field.

\end{itemize}

\subsection{Lecture Notes}
\begin{itemize}
    \item Current goes down to milli-Amps --- \textbf{Remember for test}.
    \item Current is a scalar --- but can be negative.
\end{itemize}

\subsection{Recitation}
\begin{itemize}
    \item Ohm's Law, in general form, $\vec{J} + \sigma \vec{E}$
    \item In this class, Ohm's law will just be current though a wire.
    \item We can assume the current density is proportional $\frac{I}{A}$
    \item The electric field should be uniform!
    \item $V = IR$ servers as our definition of resistance
    \item Some materials actually have a $-\alpha$
\end{itemize}

\end{document}