% Polarization of dielectric
% Why voltage difference

\RequirePackage[l2tabu, orthodox]{nag}
\documentclass[11pt]{article}

\newcommand{\chapterTitle}{Energy stored in Capacitors and Electric Fields; Dielectrics}
\newcommand{\lectureNumber}{9}

\usepackage{amssymb,amsmath,verbatim,graphicx,microtype,upquote,units,booktabs,akkwidepage}

\newcommand{\chapterNumber}[1]{
    \setcounter{section}{#1}
    \addtocounter{section}{-1}
}

\begin{document}
\section{\chapterTitle}

% \subsection{Book Notes}
% \begin{itemize}
%     \item
% \end{itemize}

\subsection{Lecture Notes}
\begin{itemize}
    \item Charge goes up (conceptual example).
    \item Lets you apply higher voltages (so more charge).
    \item Lets you place the plates closer together (make d smaller).
    \item Increases the value of $C$ because $\kappa > 1$.
\end{itemize}

\subsection{Recitation}
\begin{itemize}
    \item Before the exam, we had a capacitance equation with a $\kappa$.
    \begin{itemize}
        \item Suppose we disconnect from battery, so $Q$ is constant>
        \item Now put an insulator inside capacitor.
        \begin{itemize}
             \item Tada, there's you $\kappa$
             \item Insulator means electrons are stuck where they're at.
             \item $\kappa > 1$, usually.
         \end{itemize}
         \item This causes an electric field inside, causes a polarization.
    \end{itemize}

    \item We can express the energy in three different forms, depending on what we know about the system.
    \item Energy conservation is inapplicable to these problems.
\end{itemize}

\end{document}