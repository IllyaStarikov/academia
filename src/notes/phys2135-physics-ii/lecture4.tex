\RequirePackage[l2tabu, orthodox]{nag}
\documentclass[11pt]{article}

\newcommand{\chapterTitle}{Gauss’ Law Calculations, Conductors and Electric Fields}
\newcommand{\lectureNumber}{4}

\usetheme[logo=template-presentation/logo,faculty=ped]{fibeamer}

\usepackage{amssymb,amsmath,verbatim,graphicx,pdfpages,microtype,units,booktabs,upquote,xcolor,siunitx,csquotes,fancyvrb,newverbs,wrapfig,multicol,tikz,textcomp,wrapfig,cutwin}
\usepackage{fontawesome,setspace,rotchiffre,lipsum,listings,animate,listings}
\usepackage[xspace]{ellipsis}

\hypersetup{%
            colorlinks = true,
            linkcolor = orange,
            urlcolor  = orange,
            citecolor = orange,
            anchorcolor = orange}

\newcommand{\hugeslide}[1]{%
\begin{frame}[plain,c]
    \centering {\usebeamerfont*{frametitle} \usebeamercolor[fg]{frametitle}{\fontsize{40}{50}\selectfont\textit{#1}}}
\end{frame}
}

\newcommand{\presentaddcount}[1]{\addtocounter{#1}{1}\Roman{#1}}
\newcommand{\presentcount}[1]{\Roman{#1}}
\newcommand{\shellcmd}[1]{\texttt{\colorbox{gray!30}{#1}}}

\lstdefinelanguage{swift}
{%
  morekeywords={%
    func,if,then,else,for,in,while,do,switch,case,default,where,break,continue,fallthrough,return,
    typealias,struct,class,enum,protocol,var,func,let,get,set,willSet,didSet,inout,init,deinit,extension,
    subscript,prefix,operator,infix,postfix,precedence,associativity,left,right,none,convenience,dynamic,
    final,lazy,mutating,nonmutating,optional,override,required,static,unowned,safe,weak,internal,
    private,public,is,as,self,unsafe,dynamicType,true,false,nil,Type,Protocol,print
  },
  morecomment=[l]{//}, % l is for line comment
  morecomment=[s]{/*}{*/}, % s is for start and end delimiter
  morestring=[b]" % defines that strings are enclosed in double quotes
}

\definecolor{keyword}{HTML}{BA2CA3}
\definecolor{string}{HTML}{D12F1B}
\definecolor{comment}{HTML}{008400}
\definecolor{type}{HTML}{66B9AA}

\lstdefinestyle{Swift}{%
  language=swift,
  basicstyle=\ttfamily,
  showstringspaces=false, % lets spaces in strings appear as real spaces
  columns=fixed,
  keepspaces=true,
  keywordstyle=\color{keyword},
  stringstyle=\color{string},
  commentstyle=\color{comment},
  emph={Int,Character,Double,Float,Unsigned},
  emphstyle={\color{type}},
  morestring=[b]",
  escapeinside={(*}{*)}
}

\newcommand\syntaxbox[2][fill=orange!80]{%
    \tikz[baseline]\node[%
        inner ysep=0pt,
        inner xsep=2pt,
        anchor=text,
        rectangle,
        rounded corners=1mm,
        #1] {\strut#2};%
}



\begin{document}
\section{\chapterTitle}

\subsection{Book Notes}
\begin{itemize}
    \item Whether there is a net outward or inward electric flux through a closed surface depends on the sign of the enclosed charge.
    \item Charges outside the surface do not give a net electric flux through the surface.
    \item The net electric flux is directly proportional to the net amount of charge enclosed within the surface but is otherwise independent of the size of the closed surface.
    \item Gauss’s law states that the total electric flux through any closed surface (a surface enclosing a definite volume) is proportional to the total (net) electric charge inside the surface.
    \item When excess charge is placed on a solid conductor and is at rest, it resides entirely on the surface, not in the interior of the material.
    \item Electrostatic equilibrium means there is no net motion of tne charges inside the conductor.
    \begin{itemize}
        \item The electric field inside the conductor must be zero.
        \item If this were not the case, charges would accelerate.
    \end{itemize}

    \item Any excess charge must reside on the outside surface of the conductor.
    \item The electric field just outside a charged conductor must be perpendicular to the conductor’s surface.
    \item The magnitude of the electric field just outside a charged conductor is equal to $\frac{|\sigma|}{\epsilon_0}$, where $|\sigma|$ is the magnitude of the local surface charge density.
\end{itemize}

\subsection{Lecture Notes}
\begin{itemize}
    \item The electric field inside the conductor must be zero.
    \begin{itemize}
        \item If not, the system would accelerate.
    \end{itemize}
\end{itemize}

\subsection{Recitation}
\begin{itemize}
    \item We're doing some vector review.
    \begin{itemize}
        \item $\vec{A} \cdot \vec{B} = AB \cos \theta$
        \begin{itemize}
             \item max at $\theta = 0$
             \item min at $\theta = \pi$
         \end{itemize}
         \item $|\vec{A} \times \vec{B} = |AB \sin \theta|$
         \begin{itemize}
             \item max at $\theta = \frac{\pi}{2}$
             \item min at $\theta = 0$
         \end{itemize}
    \end{itemize}

    \item Back to physics. $\vec{F} = q \vec{E}$
    \item Inside conductor, $E = 0$. Otherwise electrons would still be moving.
    \begin{itemize}
        \item This brings us to Gauss's law.
    \end{itemize}

    \item A good way to calculate charge would be to start at the inner most surface and work on the way out.
\end{itemize}

\end{document}