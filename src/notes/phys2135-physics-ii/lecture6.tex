\RequirePackage[l2tabu, orthodox]{nag}
\documentclass[11pt]{article}

\newcommand{\chapterTitle}{Electric Potentials of Charge Distributions, Equipotentials, Potential Gradient}
\newcommand{\lectureNumber}{6}

\usetheme[logo=template-presentation/logo,faculty=ped]{fibeamer}

\usepackage{amssymb,amsmath,verbatim,graphicx,pdfpages,microtype,units,booktabs,upquote,xcolor,siunitx,csquotes,fancyvrb,newverbs,wrapfig,multicol,tikz,textcomp,wrapfig,cutwin}
\usepackage{fontawesome,setspace,rotchiffre,lipsum,listings,animate,listings}
\usepackage[xspace]{ellipsis}

\hypersetup{%
            colorlinks = true,
            linkcolor = orange,
            urlcolor  = orange,
            citecolor = orange,
            anchorcolor = orange}

\newcommand{\hugeslide}[1]{%
\begin{frame}[plain,c]
    \centering {\usebeamerfont*{frametitle} \usebeamercolor[fg]{frametitle}{\fontsize{40}{50}\selectfont\textit{#1}}}
\end{frame}
}

\newcommand{\presentaddcount}[1]{\addtocounter{#1}{1}\Roman{#1}}
\newcommand{\presentcount}[1]{\Roman{#1}}
\newcommand{\shellcmd}[1]{\texttt{\colorbox{gray!30}{#1}}}

\lstdefinelanguage{swift}
{%
  morekeywords={%
    func,if,then,else,for,in,while,do,switch,case,default,where,break,continue,fallthrough,return,
    typealias,struct,class,enum,protocol,var,func,let,get,set,willSet,didSet,inout,init,deinit,extension,
    subscript,prefix,operator,infix,postfix,precedence,associativity,left,right,none,convenience,dynamic,
    final,lazy,mutating,nonmutating,optional,override,required,static,unowned,safe,weak,internal,
    private,public,is,as,self,unsafe,dynamicType,true,false,nil,Type,Protocol,print
  },
  morecomment=[l]{//}, % l is for line comment
  morecomment=[s]{/*}{*/}, % s is for start and end delimiter
  morestring=[b]" % defines that strings are enclosed in double quotes
}

\definecolor{keyword}{HTML}{BA2CA3}
\definecolor{string}{HTML}{D12F1B}
\definecolor{comment}{HTML}{008400}
\definecolor{type}{HTML}{66B9AA}

\lstdefinestyle{Swift}{%
  language=swift,
  basicstyle=\ttfamily,
  showstringspaces=false, % lets spaces in strings appear as real spaces
  columns=fixed,
  keepspaces=true,
  keywordstyle=\color{keyword},
  stringstyle=\color{string},
  commentstyle=\color{comment},
  emph={Int,Character,Double,Float,Unsigned},
  emphstyle={\color{type}},
  morestring=[b]",
  escapeinside={(*}{*)}
}

\newcommand\syntaxbox[2][fill=orange!80]{%
    \tikz[baseline]\node[%
        inner ysep=0pt,
        inner xsep=2pt,
        anchor=text,
        rectangle,
        rounded corners=1mm,
        #1] {\strut#2};%
}



\begin{document}
\section{\chapterTitle}

\subsection{Book Notes}
\begin{itemize}
    \item An equipotential  surface is a three-dimensional surface on which the electric potential $V$ is the same at every point.
    \item On a given equipotential surface, the potential $V$ has the same value at every point. In general, however, the electric-field magnitude $E$ is not the same at all points on an equipotential surface.
    \item When all charges are at rest, the surface of a conductor is always an equipotential surface
    \item In an electrostatic situation, if a conductor contains a cavity and if no charge is present inside the cavity, then there can be no net charge anywhere on the surface of the cavity.
    \item surface charge density on the wall of the cavity is zero at every point.
    \item Don’t confuse equipotential surfaces with the Gaussian surfaces we encountered in Chapter 22. Gaussian surfaces have relevance only when we are using Gauss’s law, and we can choose any Gaussian surface that’s convenient. We cannot choose equipotential surfaces; the shape is determined by the charge distribution.
\end{itemize}

\subsection{Lecture Notes}
\begin{itemize}
    \item The disk of charge will be on exam.
    \item The ``Ed'' equation does not require rectangular plates, or any plates at all. It works as long as E is uniform and parallel or antiparallel to d.
\end{itemize}

\subsection{Recitation}
\begin{itemize}
    \item Potential is a scalar, which awesommmmee.
    \begin{itemize}
        \item But we can't exploit symmetry because of no directions.
    \end{itemize}
\end{itemize}

\end{document}