\RequirePackage[l2tabu, orthodox]{nag}
\documentclass[11pt]{article}

\newcommand{\chapterTitle}{Electric Potential, Electric Potential Energy}
\newcommand{\lectureNumber}{5}

\usetheme[logo=template-presentation/logo,faculty=ped]{fibeamer}

\usepackage{amssymb,amsmath,verbatim,graphicx,pdfpages,microtype,units,booktabs,upquote,xcolor,siunitx,csquotes,fancyvrb,newverbs,wrapfig,multicol,tikz,textcomp,wrapfig,cutwin}
\usepackage{fontawesome,setspace,rotchiffre,lipsum,listings,animate,listings}
\usepackage[xspace]{ellipsis}

\hypersetup{%
            colorlinks = true,
            linkcolor = orange,
            urlcolor  = orange,
            citecolor = orange,
            anchorcolor = orange}

\newcommand{\hugeslide}[1]{%
\begin{frame}[plain,c]
    \centering {\usebeamerfont*{frametitle} \usebeamercolor[fg]{frametitle}{\fontsize{40}{50}\selectfont\textit{#1}}}
\end{frame}
}

\newcommand{\presentaddcount}[1]{\addtocounter{#1}{1}\Roman{#1}}
\newcommand{\presentcount}[1]{\Roman{#1}}
\newcommand{\shellcmd}[1]{\texttt{\colorbox{gray!30}{#1}}}

\lstdefinelanguage{swift}
{%
  morekeywords={%
    func,if,then,else,for,in,while,do,switch,case,default,where,break,continue,fallthrough,return,
    typealias,struct,class,enum,protocol,var,func,let,get,set,willSet,didSet,inout,init,deinit,extension,
    subscript,prefix,operator,infix,postfix,precedence,associativity,left,right,none,convenience,dynamic,
    final,lazy,mutating,nonmutating,optional,override,required,static,unowned,safe,weak,internal,
    private,public,is,as,self,unsafe,dynamicType,true,false,nil,Type,Protocol,print
  },
  morecomment=[l]{//}, % l is for line comment
  morecomment=[s]{/*}{*/}, % s is for start and end delimiter
  morestring=[b]" % defines that strings are enclosed in double quotes
}

\definecolor{keyword}{HTML}{BA2CA3}
\definecolor{string}{HTML}{D12F1B}
\definecolor{comment}{HTML}{008400}
\definecolor{type}{HTML}{66B9AA}

\lstdefinestyle{Swift}{%
  language=swift,
  basicstyle=\ttfamily,
  showstringspaces=false, % lets spaces in strings appear as real spaces
  columns=fixed,
  keepspaces=true,
  keywordstyle=\color{keyword},
  stringstyle=\color{string},
  commentstyle=\color{comment},
  emph={Int,Character,Double,Float,Unsigned},
  emphstyle={\color{type}},
  morestring=[b]",
  escapeinside={(*}{*)}
}

\newcommand\syntaxbox[2][fill=orange!80]{%
    \tikz[baseline]\node[%
        inner ysep=0pt,
        inner xsep=2pt,
        anchor=text,
        rectangle,
        rounded corners=1mm,
        #1] {\strut#2};%
}



\begin{document}
\section{\chapterTitle}

\subsection{Book Notes}
\begin{itemize}
    \item The potential-energy difference $U_a - U_b$ equals the work that is done by the electric force when the particle moves from $a$ to $b$. When $U_a$ is greater than $U_b$, the field does positive work on the particle as it “falls” from a point of higher potential energy ($a$) to a point of lower potential energy ($b$).
    \item The potential-energy difference $U_a - U_b$ is then defined as the work that must be done by an external force to move the particle slowly from $b$ to $a$ against the electric force.
    \item \textbf{Potential} is potential energy per unit charge.
    \item SI unit of potential is called one volt ($1 \mathbf{V}$).
    \item The potential difference between two points is often called \textbf{voltage}.
    \item $V_{ab}$, the potential of $a$ with respect to $b$, equals the work that must be done to move a UNIT charge slowly from $b$ to $a$ against the electric force.
    \item The electric potential at a certain point is the potential energy that would be associated with a unit charge placed at that point. That’s why potential is measured in joules per coulomb, or volts. Keep in mind, too, that there doesn’t have to be a charge at a given point for a potential $V$ to exist at that point. (In the same way, an electric field can exist at a given point even if there’s no charge there to respond to it.)
    \item Moving with the direction of $\mathbf{\vec{E}}$ means moving in the direction of \textit{decreasing} $V$, and moving against the direction of $\mathbf{\vec{E}}$ means moving in the direction of \textit{increasing} $V$.
\end{itemize}

\subsection{Lecture Notes}
\begin{itemize}
    \item $\Delta U = - [W_\text{conservative}] _{i \rightarrow f}$
    \begin{itemize}
        \item Always ask yourself which work you are calculating.
    \end{itemize}

    \item Don't fall into the trap of making $r_{1 2}$ a square.
    \item Potential energies are defined relative to some configuration of objects that you are free to choose.
    \begin{itemize}
        \item For example, it often makes sense to define the gravitational potential energy of a ball to be zero when it is resting on the surface of the earth, but you don’t have to make that choice.
    \end{itemize}

    \item Our equation for the electric potential energy of two charged particles uses the convention that the potential energy is zero when the particles are infinitely far apart.
    \item Protons fall down, electrons fall up.
    \item An electron volt (eV) is the energy acquired by a particle of charge e when it moves through a potential difference of 1 volt.
\end{itemize}

\end{document}