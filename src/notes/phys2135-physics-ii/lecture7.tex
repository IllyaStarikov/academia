% move negative into ln


\RequirePackage[l2tabu, orthodox]{nag}
\documentclass[11pt]{article}

\newcommand{\chapterTitle}{Capacitance, Capacitors in Series and Parallel}
\newcommand{\lectureNumber}{7}

\usetheme[logo=template-presentation/logo,faculty=ped]{fibeamer}

\usepackage{amssymb,amsmath,verbatim,graphicx,pdfpages,microtype,units,booktabs,upquote,xcolor,siunitx,csquotes,fancyvrb,newverbs,wrapfig,multicol,tikz,textcomp,wrapfig,cutwin}
\usepackage{fontawesome,setspace,rotchiffre,lipsum,listings,animate,listings}
\usepackage[xspace]{ellipsis}

\hypersetup{%
            colorlinks = true,
            linkcolor = orange,
            urlcolor  = orange,
            citecolor = orange,
            anchorcolor = orange}

\newcommand{\hugeslide}[1]{%
\begin{frame}[plain,c]
    \centering {\usebeamerfont*{frametitle} \usebeamercolor[fg]{frametitle}{\fontsize{40}{50}\selectfont\textit{#1}}}
\end{frame}
}

\newcommand{\presentaddcount}[1]{\addtocounter{#1}{1}\Roman{#1}}
\newcommand{\presentcount}[1]{\Roman{#1}}
\newcommand{\shellcmd}[1]{\texttt{\colorbox{gray!30}{#1}}}

\lstdefinelanguage{swift}
{%
  morekeywords={%
    func,if,then,else,for,in,while,do,switch,case,default,where,break,continue,fallthrough,return,
    typealias,struct,class,enum,protocol,var,func,let,get,set,willSet,didSet,inout,init,deinit,extension,
    subscript,prefix,operator,infix,postfix,precedence,associativity,left,right,none,convenience,dynamic,
    final,lazy,mutating,nonmutating,optional,override,required,static,unowned,safe,weak,internal,
    private,public,is,as,self,unsafe,dynamicType,true,false,nil,Type,Protocol,print
  },
  morecomment=[l]{//}, % l is for line comment
  morecomment=[s]{/*}{*/}, % s is for start and end delimiter
  morestring=[b]" % defines that strings are enclosed in double quotes
}

\definecolor{keyword}{HTML}{BA2CA3}
\definecolor{string}{HTML}{D12F1B}
\definecolor{comment}{HTML}{008400}
\definecolor{type}{HTML}{66B9AA}

\lstdefinestyle{Swift}{%
  language=swift,
  basicstyle=\ttfamily,
  showstringspaces=false, % lets spaces in strings appear as real spaces
  columns=fixed,
  keepspaces=true,
  keywordstyle=\color{keyword},
  stringstyle=\color{string},
  commentstyle=\color{comment},
  emph={Int,Character,Double,Float,Unsigned},
  emphstyle={\color{type}},
  morestring=[b]",
  escapeinside={(*}{*)}
}

\newcommand\syntaxbox[2][fill=orange!80]{%
    \tikz[baseline]\node[%
        inner ysep=0pt,
        inner xsep=2pt,
        anchor=text,
        rectangle,
        rounded corners=1mm,
        #1] {\strut#2};%
}



\begin{document}
\section{\chapterTitle}

\subsection{Book Notes}
\begin{itemize}
    \item Any two conductors seperated by an insulator (or a vacuum) form a capacitor.
    \item Don’t confuse the symbol C for capacitance (which is always in italics) with the abbreviation C for coulombs (which is never italicized).
    \item Thus capacitance is a measure of the ability of a capacitor to store energy.
    \item The reciprocal of the equivalent capacitance of a series combination equals the sum of the reciprocals of the individual capacitances.
    \item The magnitude of charge is the same on all plates of all the capacitors in a series combination; however, the potential differences of the individual capacitors are not the same unless their individual capacitances are the same. The potential differences of the individual capacitors add to give the total potential difference across the series combination: $V_{\text{total}} = V_1 + V_2 + V_3 + \cdots$
    \item The equivalent capacitance of a parallel combination equals the sum of the individual capacitances.
    \item The potential differences are the same for all capaci- tors in a parallel combination; however, the charges on individual capacitors are not the same unless their individual capacitances are the same. The charges on the individ- ual capacitors add to give the total charge on the parallel combination: $Qtotal = Q1 + Q2 + Q3 + \cdot$
\end{itemize}

\subsection{Recitation}
\begin{itemize}
    \item The simplest capacitor is just two plates, both with a charge $\pm Q$ (they different).
    \begin{itemize}
        \item Capacitance, by definition, has to be positive.
        \item $C = \frac{\sigma A \epsilon _0}{\sigma d} = \frac{\kappa \epsilon _0}{d}$
        \begin{itemize}
             \item Only depends on the geometry.
         \end{itemize}
    \end{itemize}

    \item \textit{Does example from lecture, except with sphere.}.
    \begin{itemize}
         \item Except specifies direction with $\hat{r}$
     \end{itemize}

     \item With series, $Q_1 = Q_2$.
    \begin{itemize}
        \item $\frac{1}{C_{eq}} = \frac{1}{C_1} + \frac{1}{C_2} + \cdots + \frac{1}{C_n}$
    \end{itemize}

    \item Opposite is true
    \begin{itemize}
        \item  $V_1 = V_2$
         \item $C_{eq} = C_1 + C_2 + \cdot + C_n$
     \end{itemize}
\end{itemize}

\end{document}