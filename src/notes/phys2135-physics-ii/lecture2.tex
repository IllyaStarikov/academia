\RequirePackage[l2tabu, orthodox]{nag}
\documentclass[11pt]{article}

\newcommand{\chapterTitle}{Electric Field of a Charge Distribution}
\newcommand{\lectureNumber}{2}

\usepackage{amssymb,amsmath,verbatim,graphicx,microtype,upquote,units,booktabs,akkwidepage}

\newcommand{\chapterNumber}[1]{
    \setcounter{section}{#1}
    \addtocounter{section}{-1}
}

\begin{document}
\section{\chapterTitle}

\subsection{Lecture Notes}
\begin{itemize}
    \item $\lambda \neq$ to wavelength, but linear charge density (charge per length).
    \item Arc length is angle (radians) times radius ($S = r\theta$).
    \item No two field lines can cross.
\end{itemize}

\subsection{Recitation}
\begin{itemize}
    \item Because in nature, point charges are not common, we do a distributed charges.
    \begin{itemize}
        \item We do this via superposition (i.e. add up vectorily to get the total charge).
    \end{itemize}
    \item Suppose you have a rod, and an origin at the center, and a point h distance at the origin.
    \begin{itemize}
        \item $dE_x = dE \sin \theta$
        \item Likewise, $dE_y = dE \cos \theta$
    \end{itemize}
\end{itemize}

\begin{align}
    dE_y & = dE \cos \theta \\
    & = \frac{k \lambda dx}{r^2} \frac{h}{r}
\end{align}

\begin{itemize}
    \item This is one of two charge distribution we often encounter.
    \begin{itemize}
        \item Still the same process.
    \end{itemize}
\end{itemize}

\end{document}
