\RequirePackage[l2tabu, orthodox]{nag}
\documentclass[11pt]{article}

\newcommand{\chapterTitle}{Resistors in Series and Parallel, Kirchoff's Rules}
\newcommand{\lectureNumber}{12}

\usepackage{amssymb,amsmath,verbatim,graphicx,microtype,upquote,units,booktabs,akkwidepage}

\newcommand{\chapterNumber}[1]{
    \setcounter{section}{#1}
    \addtocounter{section}{-1}
}

\begin{document}
\section{\chapterTitle}

\subsection{Book Notes}
For capacitors in series,

\begin{align}
    V_{ax} = IR_1 \quad\quad V_{xy} &= IR_2 \quad\quad V_{yb} = IR_3 \\
    V_{ab} = V_{ax} + V_{xy} + V_{yb} &= I(R_1 + R_2 + R_3) \\
    \frac{V_{ab}}{I} &= R_1 + R_2 + R_3 \\
    R_{eq} = R_1 + R_2 + R_3
\end{align}

A similar argument can be made for resistors in parallel, except $I$ is unknown, which add.

\begin{itemize}
    \item A junction in a circuit is a point where three or more conductors meet. A loop is any closed conducting path.
\end{itemize}

\end{document}