\RequirePackage[l2tabu, orthodox]{nag}
\documentclass[12pt]{article}

\usetheme[logo=template-presentation/logo,faculty=ped]{fibeamer}

\usepackage{amssymb,amsmath,verbatim,graphicx,pdfpages,microtype,units,booktabs,upquote,xcolor,siunitx,csquotes,fancyvrb,newverbs,wrapfig,multicol,tikz,textcomp,wrapfig,cutwin}
\usepackage{fontawesome,setspace,rotchiffre,lipsum,listings,animate,listings}
\usepackage[xspace]{ellipsis}

\hypersetup{%
            colorlinks = true,
            linkcolor = orange,
            urlcolor  = orange,
            citecolor = orange,
            anchorcolor = orange}

\newcommand{\hugeslide}[1]{%
\begin{frame}[plain,c]
    \centering {\usebeamerfont*{frametitle} \usebeamercolor[fg]{frametitle}{\fontsize{40}{50}\selectfont\textit{#1}}}
\end{frame}
}

\newcommand{\presentaddcount}[1]{\addtocounter{#1}{1}\Roman{#1}}
\newcommand{\presentcount}[1]{\Roman{#1}}
\newcommand{\shellcmd}[1]{\texttt{\colorbox{gray!30}{#1}}}

\lstdefinelanguage{swift}
{%
  morekeywords={%
    func,if,then,else,for,in,while,do,switch,case,default,where,break,continue,fallthrough,return,
    typealias,struct,class,enum,protocol,var,func,let,get,set,willSet,didSet,inout,init,deinit,extension,
    subscript,prefix,operator,infix,postfix,precedence,associativity,left,right,none,convenience,dynamic,
    final,lazy,mutating,nonmutating,optional,override,required,static,unowned,safe,weak,internal,
    private,public,is,as,self,unsafe,dynamicType,true,false,nil,Type,Protocol,print
  },
  morecomment=[l]{//}, % l is for line comment
  morecomment=[s]{/*}{*/}, % s is for start and end delimiter
  morestring=[b]" % defines that strings are enclosed in double quotes
}

\definecolor{keyword}{HTML}{BA2CA3}
\definecolor{string}{HTML}{D12F1B}
\definecolor{comment}{HTML}{008400}
\definecolor{type}{HTML}{66B9AA}

\lstdefinestyle{Swift}{%
  language=swift,
  basicstyle=\ttfamily,
  showstringspaces=false, % lets spaces in strings appear as real spaces
  columns=fixed,
  keepspaces=true,
  keywordstyle=\color{keyword},
  stringstyle=\color{string},
  commentstyle=\color{comment},
  emph={Int,Character,Double,Float,Unsigned},
  emphstyle={\color{type}},
  morestring=[b]",
  escapeinside={(*}{*)}
}

\newcommand\syntaxbox[2][fill=orange!80]{%
    \tikz[baseline]\node[%
        inner ysep=0pt,
        inner xsep=2pt,
        anchor=text,
        rectangle,
        rounded corners=1mm,
        #1] {\strut#2};%
}


\chapterNumber{2}

\begin{document}
\section{Error Tolerance}

\begin{itemize}
    \item There will be $15\%$ extra credit.
    \item Two important terms, accuracy \& precision.
    \begin{itemize}
        \item Accuracy refers to how close the computed value is to the true value.
        \item Precision means how close the values are together.
    \end{itemize}

    \item Because we want everything relative, we calculate with two different errors.
    \begin{itemize}
        \item Absolute error = $e_a = |\text{True value} - \text{Approximation}|$
        \item Relative error = $\epsilon_t = \frac{\text{True value} - \text{Approximation}}{\text{true}} \times 100$
    \end{itemize}

    \item These equations can be used for precision as well.
    \begin{itemize}
        \item $e_a = |\text{previous} - \text{current}|$
        \item $\epsilon_t = \frac{\text{previous} - \text{current}}{\text{current}} \times 100$
    \end{itemize}

    \item Stopping point is when it's below a certain threshold.
    \item If true value is 0, shift over by 1.
    \item For a result to be accurate or precise, should be $< 0.5 \times 10^{2-n}$, where $n$ is the number of significant digits.

    \item the function of $e^x$ can be represented as $e^x = 1 + x + \frac{x^2}{2} + \frac{x^3}{3!} + \cdots + \frac{x^n}{n!}$
    \begin{itemize}
        \item for $x = 1$, $e^1 = 2.71828$
        \item for $x = -1$, $a = 1 - 1 = 0$
    \end{itemize}
\end{itemize}

\end{document}