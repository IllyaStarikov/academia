\RequirePackage[l2tabu, orthodox]{nag}
\documentclass[12pt]{article}

\usepackage{amssymb,amsmath,verbatim,graphicx,microtype,upquote,units,booktabs,akkwidepage}

\newcommand{\chapterNumber}[1]{
    \setcounter{section}{#1}
    \addtocounter{section}{-1}
}
\chapterNumber{2}

\begin{document}
\section{Error Tolerance}

\begin{itemize}
    \item There will be $15\%$ extra credit.
    \item Two important terms, accuracy \& precision.
    \begin{itemize}
        \item Accuracy refers to how close the computed value is to the true value.
        \item Precision means how close the values are together.
    \end{itemize}

    \item Because we want everything relative, we calculate with two different errors.
    \begin{itemize}
        \item Absolute error = $e_a = |\text{True value} - \text{Approximation}|$
        \item Relative error = $\epsilon_t = \frac{\text{True value} - \text{Approximation}}{\text{true}} \times 100$
    \end{itemize}

    \item These equations can be used for precision as well.
    \begin{itemize}
        \item $e_a = |\text{previous} - \text{current}|$
        \item $\epsilon_t = \frac{\text{previous} - \text{current}}{\text{current}} \times 100$
    \end{itemize}

    \item Stopping point is when it's below a certain threshold.
    \item If true value is 0, shift over by 1.
    \item For a result to be accurate or precise, should be $< 0.5 \times 10^{2-n}$, where $n$ is the number of significant digits.

    \item the function of $e^x$ can be represented as $e^x = 1 + x + \frac{x^2}{2} + \frac{x^3}{3!} + \cdots + \frac{x^n}{n!}$
    \begin{itemize}
        \item for $x = 1$, $e^1 = 2.71828$
        \item for $x = -1$, $a = 1 - 1 = 0$
    \end{itemize}
\end{itemize}

\end{document}