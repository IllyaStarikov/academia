\documentclass{article}
\usepackage{amsmath}
\usepackage{fancyvrb}
\usepackage{mathtools}
\usepackage{tabularx}
\DeclarePairedDelimiter\ceil{\lceil}{\rceil}
\DeclarePairedDelimiter\floor{\lfloor}{\rfloor}
\begin{document}

\title{Test I Study Guide}
\author{Illya Starikov}
\date{February 15, 2016}
\maketitle

\fvset{%
numbers=left
}

\section{Sorting Algorithms}
\subsection{Insertion-Sort}
\begin{Verbatim}[commandchars=\\\{\},codes={\catcode`$=3\catcode`_=8}]
Insertion-Sort(A)
    for j = 2 to A.length
        key = A[j]
        // Insert A[j] into the sorted sequence A[1..j - 1]
        i = j - 1
        while i > 0 and A[j] > key
            A[i + 1] =  A[i]
            i = i - 1
        A[i + 1] = key
\end{Verbatim}

\subsection{Merge Sort}
\begin{Verbatim}[commandchars=\\\{\},codes={\catcode`$=3\catcode`_=8}]
Merge-Sort(A, p, r)
    if p < r
        q = $ \floor*{(q + r) /2}$
        Merge-Sort(A, p, q)
        Merge-Sort(A, q + 1, r)
        Merge(A, p, q, r)

Merge(A, p, q, r)
    $n_1$ = q - p + 1
    $n_2$ = r - q
    let L[1...$n_1$ + 1] and R[1..$n_2$] be new arrays
    for i = 1 to $n_1$
        L[i] = A[p + i - 1]
    for j = 1 to $n_2$
        R[j] = A[q + j]
    R[$n_1$ + 1] = $\infty$
    R[$n_2$ + 1] = $\infty$
    i = 1
    j = 1
    for k = p to r
        if L[i] $\leq$ R[j]
            A[k] = L[i]
            i = i + 1
        else A[k] = R[j]
            j = j + 1
\end{Verbatim}

\subsection{Heap Sort}
\begin{Verbatim}[commandchars=\\\{\},codes={\catcode`$=3\catcode`_=8}]
Max-Heapify(A, i)
    l = Left(i)
    r = Right(i)
    if l $\leq$ A.heap-size and A[l] > A[i]
        largest = l
    else largest = i
    if r $\leq$ A.heap-size and A[r] > A[largest]
        largest = r
    if largest $\neq$ i
        exchange A[i] with A[largest]
        Max-Heaphify(A-largest)

Heapsort(A)
    Build-Max-Heap(A)
    for i = A.length downto 2
        exchange A[1] with A[i]
        A.heap-size = A.heap-size - 1
        Max-Heapify(A, 1)
\end{Verbatim}

\subsection{Quicksort}
\begin{Verbatim}[commandchars=\\\{\},codes={\catcode`$=3\catcode`_=8}]
Quicksort(A, p, r)
    if p < r
        q = Partition(A, p, r)
        Quicksort(A, p, q - 1)
        Quicksort(A, q + 1, r)

Partition(A, p, r)
    x = A[r]
    i = p - 1
    for j = p to r - 1
        if A[j] $\leq$ x
            i = i + 1
            exchange A[i] with A[j]
    exchange A[i + 1] with A[r]
    return i + 1
\end{Verbatim}

\subsection{Counting Sort}
Let $A$ = input array, $B$ = output array, $n$ = size of array, $k$ = maximum number.
\begin{Verbatim}
countingSort(A, B, n, k)
    let C[0...k] be the counting array
    for i = 0 to k
        C[i] = 0
    for j = 1 to n
        C[A[j]] = C[A[j]] + 1
    for i = 1 to k
        C[i] = C[i] + C[i - 1]
    for j = n down to 1
        B[C[A[j]]] = A[j]
        C[A[j]] = C[A[j]] - 1
\end{Verbatim}

\subsection{Rate of Growth}
\begin{description}
    \item [Insertion Sort] Worst: $\Theta(n^2)$, Average: $
    \Theta(n^2)$
    \item [Merge Sort] Worst: $\Theta(n \lg n)$, Average: $
        \Theta(n \lg n)$
    \item [Quick Sort] Worst: $\Theta(n^2)$, Average: $
        \Theta(n \lg n)$
    \item [Heap Sort] Worst: $\Theta(n \lg n)$, Average: $
        \Theta(n \lg n)$
    \end{description}

\section{Growth Classes}
\subsection{$O$-notation}
\begin{equation*}
O(g(n)) = \{ f(n) : \ \text{there exists positive constant $c$ and $n_0$ such that} \ 0 \leq f(n) \leq c \ g(n) \ \text{for all} \ n \geq n_0 \}
\end{equation*}

\subsection{$\Omega$-notation}
\begin{equation*}
\Omega(g(n)) = \{ f(n) : \ \text{there exists positive constants $c$ and $n_0$ such that} \ 0 \leq c \ g(n) \leq f(n) \ \text{for all} \ n \geq n_0 \}
\end{equation*}

\subsection{$\Theta$-notation}
\begin{align*}
\Theta(g(n)) = \{ f(n) : \ \text{there exists positive constants $c_1, c_2$ and $n_0$ such that} \\ \ 0 \leq c_1 \ g(n) \leq f(n) \leq c_2\ g(n) \ \text{for all} \ n \geq n_0 \}
\end{align*}

\subsection{$o$-notation (Little-o)}
$f(n) \in o(g(n)) $ iff for every $c > 0$ there is an $n_0 > 0$ such that

\begin{equation*}
0 \leq f(n) < c \ g(n)
\end{equation*}

for all $n \geq n_0$

\subsection{$\omega$-notation (Little-omega)}
\begin{equation*}
f(n) \in \omega(g(n)) \ \text{iff} \ g(n) \in o(f(n)).
\end{equation*}

or $f(n) \in \omega(g(n))$ iff for every $c > 0$ there is an $n_0 > 0$ such that

\begin{equation*}
0 \leq c \ g(n) < f(n)
\end{equation*}

for all $n \geq n_0$

\section{Master Theorem}
The master theorem can only be applied to recurrence equations of the form:

\begin{equation*}
T(n) = aT(n/b) + f(n)
\end{equation*}

\subsection{Constants}
\begin{description}
    \item [n] The size of the problem
    \item [a] The number of subproblems
    \item [n/b] The size of each subproblem
    \item [f(n)] cost outside of recursive calls (divide, combine)
\end{description}

\subsection{Cases}
\begin{tabularx}{\textwidth}{X|X}
 \hline \\
$f(n) \in O(n^{\log _b a - \epsilon})$ & $T(n) \in \Theta(n^{\log _b a})$ \\
$f(n) \in \Theta(n^{\log _b a})$ & $T(n) \in \Theta(n ^{\log _b a} \lg n)$ \\
$f(n) \in \Omega(n^{\log _b a + \epsilon})$ & $T(n) \in \Theta (f(n))$ \\
\hline
\end{tabularx}

\section{Definitions}
\begin{description}
    \item [Algorithm] Any well-defined computational procedure that takes a set of values as input and produces a set of values as output in a finite number of steps
    \item [Correct Algorithm]  One returns the correct solution for every valid instance of a problem
    \item [Loop Invariance] Define a key property about the relationship among variables of the algorithms.
    \begin{itemize}
        \item Holds in the \emph{initial} case.
        \begin{description}
            \item [Initialization] The loop invariance must be true prior to the first iteration.
        \end{description}
        \item Is \emph{maintained} each iteration.
        \begin{description}
            \item [Maintenance] If the property holds prior to an iteration, it must still hold after the iteration is complete.
        \end{description}
        \item Yields correctness when the loop \emph{terminates}.
        \begin{description}
            \item [Termination] The invariant provides a useful property that helps demonstrate the algorithm is correct.
        \end{description}
    \end{itemize}
    \item [Heapify] Go all the way down to the heap and fix the violations of the max-heap property by sifting-up
\end{description}
\end{document}