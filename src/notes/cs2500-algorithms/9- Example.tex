\documentclass{article}
\begin{document}

\setcounter{section}{8}
\section{Examples}

\subsection{Counting Sort Algorithm}
\begin{verbatim}
4 6 4 1 3 4 1 4

Assume [0...k]. Max integer: 6, size of array: 8.

Original Array: [ 4 | 6 | 4 | 1 | 3 | 4 | 1 | 4 ]
Counting Array: [ 0 | 2 | 0 | 2 | 3 | 0 | 1 ] (size is max + 1)
Update Array:   [ 0 | 2 | 2 | 4 | 7 | 7 | 8 ]
Output Array:   [ 1 | 1 | 3 | 4 | 4 | 4 | 6 ]
(reverse order of original array via indexes, the output is not index based)

\end{verbatim}

\subsection{Growth Classes Example}
Show $20n^2 + 2n + 5  = O(n^2)$

\begin{eqnarray}
O(n^2) & = & 20n^2 + 2n + 5 \\
& = & 20n^2 + 2n^2 + 5n^2 \\
& = & 27n^2
\end{eqnarray}

So $n_0 = 1$ and $C = 27$.

\subsection{Growth Example II}
$5n+7 = o(n^2)$. For every $c$, find $n_0 > 0$

\begin{equation}
5n+7 \leq 5n + n = 6n < cn^2
\end{equation}

For $n_0 = 7$, this holds. But now we have to prove $\forall c$. We do so via limit.

\subsection{Growth Class Example}
Prove $5n^2 + 2n =  \Theta(n^2)$. Just take big-$O$ and big-$\Omega$

\subsection{Growth Class Example}
Prove $5n^2 – 15n + 100 = \Omega(n^2)$

\begin{eqnarray}
& > & 5n^2 - 15n \\
& \geq & 5n^2 - n^2 \\
& = & 4n^2
\end{eqnarray}

\subsection{Master Theorem}
\end{document}