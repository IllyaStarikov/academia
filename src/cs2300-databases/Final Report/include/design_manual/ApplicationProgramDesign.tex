%-------------------- Create New User --------------------%
\subsection{Create a New User}
This function creates a new user, accessing only the \texttt{user} table. \\

\noindent
\begin{tabular}{l|p{9.5cm}}
\textsc{Input} & \texttt{Username, Height, Birthdate, Goal, Password, Gender}. \\
\br
\textsc{Steps} & \begin{enumerate}[topsep=0pt]
\item Check to see if the \texttt{username} is available. If available, proceed. If not, display appropriate message to notify the user.
\item Insert appropriate information to the User table. (Username to \texttt{username}, height to \texttt{height})
\item Generate a user ID, assign it to the \texttt{UiD} attribute.
\item Take a time stamp, assign it to the \texttt{JoinDate} attribute.
\item Calculate the age based on the \texttt{BirthDate} attribute.
\end{enumerate}
\\
\br
\textsc{Output} & A new \texttt{User} entity will be inserted into the table, with appropriate data into proper columns (along with computed properties and derived properties).\\
\br
\textsc{Assumptions} &\texttt{Username, Height, Birthdate, Goal, Password, Gender} are all correct (this will be validated in the sign up form).
\end{tabular}

%-------------------- Delete Group --------------------%
\subsection{Delete A Group}
This function deletes a group by updating information in \texttt{Forum}, and removing the \texttt{Group} and \texttt{Group Members} tables. \\

\noindent
\begin{tabular}{l|p{9.5cm}}
\textsc{Input} & The Group ID (\texttt{GiD}) that is to be deleted. \\
\br
\textsc{Steps} & \begin{enumerate}[topsep=0pt]
\item Check to see if the request is made by the moderator via the \texttt{Moderator} column in the \texttt{Groups} table. If true, approve the request. If not, cancel the request and notify the user.
\item Remove the row that has a matching \texttt{GiD} that was provided for deletion in the \texttt{Groups} table.
\item Query the \texttt{Group} Members table, removing any row that match the GiD provided for deletion.
\item Query the \texttt{Forum} table for any matching Group Ids (\texttt{GiD}), setting the \texttt{GiD} to \texttt{null} if matching.
\end{enumerate}
\\
\br
\textsc{Output} & The groups are deleted, the group members within that group are deleted, and any reference to the group is deallocated. \\
\br
\textsc{Assumptions} & None.
\end{tabular}

%-------------------- Modify User Statics --------------------%
\subsection{Modifying User Statistics}
Our user statistics have the ability to fluctuate. We would like to accommodate for this fluctuation by allowing users to update their respective statics; specifically, we would like to let users update \texttt{Username, Height, Goal, Password, Gender} in the \texttt{Users} table. \\

\noindent
\begin{tabular}{l|p{9.5cm}}
\textsc{Input} & The specific attribute(s) of the set \texttt{Username, Height, Goal, Password, Gender} that would like to be updated with the new value. \\
\br
\textsc{Steps} & \begin{enumerate}[topsep=0pt]
\item Ensure the data is valid (e.g. is not \texttt{null} when applicable, in the proper domain). If it is valid, continue. If not, prompt the user with an error message and try again.
\item Modify the attribute to reflect the new value.
\item Repeat for any additional attributes provided.
\end{enumerate}
\\
\br
\textsc{Output} & The attribute(s) should now reflect the new value provided. \\
\br
\textsc{Assumptions} & The data is within a proper range (will not overflow).
\end{tabular}


%-------------------- Query Other Users --------------------%
\subsection{Query Other Users}
This function allows for users to query other users; this can be done via \texttt{Username} or \texttt{FirstName, MiddleName}, and \texttt{LastName} from the \texttt{Users} table. \\

\noindent
\begin{tabular}{l|p{9.5cm}}
\textsc{Input} & Either a \texttt{Username} xor any subset of \texttt{FirstName, MiddleName}, or \texttt{LastName}.\\
\br
\textsc{Steps} & \begin{enumerate}[topsep=0pt]
\item Check to see if input is valid. If is, proceed. If not, display error message to the user.
\item Query the \texttt{Users} table to see if the user exists. If the user exists, proceed. If not, display appropriate message to the user.
\item Project the profile.
\end{enumerate} \\
\br
\textsc{Output} & Either the search user will be projected or an error message is the user does not exist. \\
\br
\textsc{Assumptions} & The first, middle and last name are all provided. The names are unique (solely for the testing purposes).
\end{tabular}

%-------------------- User Leader --------------------%
\subsection{User Leaderboards}
Generate the leaderboard based on the workouts accomplished; specifically aggregating data from the \texttt{Strength} and \texttt{Cardio} table. \textit{Note this is the function that requires multiple tables.}\\

\noindent
\begin{tabular}{l|p{9.5cm}}
\textsc{Input} & None. \\
\br
\textsc{Steps} & \begin{enumerate}[topsep=0pt]
\item Merge the \texttt{User} and \texttt{Workouts} table, call the new table \texttt{Merged}.
\item Add up the total duration (call the new property \texttt{TotalDuration}) in the \texttt{Merged} table based on the \texttt{UiD} attribute, making a new table named \texttt{Sums}.
\item Sort the \texttt{Sums} by the \texttt{TotalDuration} attribute.
\item Display the top 10 on the sorted \texttt{Sums} table to the user.
\item Display the user their current rank.
\end{enumerate} \\
\br
\textsc{Output} & An eleven-row table displaying the top 10 leaderboards and the user’s current rank. \\
\br
\textsc{Assumptions} & There is a bare minimum of eleven users.
\end{tabular}

% \subsection{}
% \\

% \noindent
% \begin{tabular}{l|p{9.5cm}}
% \textsc{Input} & - \\
% \br
% \textsc{Steps} & \begin{enumerate}[topsep=0pt]
% \end{enumerate} \\
% \br
% \textsc{Output} & - \\
% \end{tabular}