ForFit has two main components. The first part is the users themselves. Users sign up for the service and then begin recording their workouts. Workouts can be logged as many times as liked, from once a month to several times a day. The more workouts logged, the better data and the more progress toward your exercise goals.




Users can view other users and the workouts they've done. This leads into the second component of ForFit, which is the encouragement of community. Users can create groups and join existing groups. These groups can help users focus on specific areas. For example, a user wanting to run a marathon might join a group of other runners training for a marathon. This way the user can keep in touch and stay motivated through the group and also have a resource to ask questions or get other workouts.



Groups allow users to make forums, for in-depth discussion, and posts, for quick updates. These groups are the main area of community creation that ForFit focuses on in the design. Users can find other users and join groups with users they have been exercising with before. This creates a network of community interaction.