We know doing tasks such as creating the database, generating a new set of random inputs, or even running all the function files to makes sure all of the function are created can tedious. Forunatly, we have a solution --- that solution is \href{http://www.freeos.com/guides/lsst/}{Shell scripting}.

We quickly realized that repetitively running the same commands\footnote{Like running \texttt{`/Applications/Postgres.app/Contents/Versions/9.5/bin'/psql -p5432 -f} to create every file!} could get tedious, so we have provided multiple shell scripts that automate most of the monotonous tasks.

To run a shell script, first you must give it privileges to be able to run \texttt{Bash} commands. This is accomplished by \shellcmd{chmod -x /path/to/file.sh}. This give read/write access to the shell script, which then allows the shell script to be run via \shellcmd{./path/to/file.sh}. This then allows the user to do tasks like setting up the entirety of the database (which spans multiple files!) one file path away.

Below are the shell scripts we provide.